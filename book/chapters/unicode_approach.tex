% ==========================
\chapter{The Unicode approach}
\label{the-unicode-approach}
% ==========================

\section{Background}

The conceptualization and terminology of writing systems was rejuvenated by
the development of the Unicode Standard, with major input from Mark Davis,
co-founder and long-term president of the Unicode Consortium. For many years,
``exotic'' writing systems and phonetic transcription systems on personal
computers were constrained by the American Standard Code for Information
Interchange (ASCII) character encoding scheme, based on the Latin script, which
only allowed for a strongly limited number of different symbols to be encoded.
This implied that users could either use and adopt the (extended) Latin alphabet
or they could assign new symbols to the small number of code points in the ASCII
encoding scheme to be rendered by a specifically designed font
\citep{BirdSimons2003}. In this situation, it was necessary to specify the font
together with each document to ensure the rightful display of its content. To
alleviate this problem of assigning different symbols to the same code points,
in the late 80s and early 90s the Unicode Consortium set itself the ambitious
goal of developing a single universal character encoding to provide a unique
number, a code point, for every character in the world's writing systems.
Nowadays, the Unicode Standard is the default encoding of the technologies that
support the World Wide Web and for all modern operating systems, software and
programming languages.

\section{The Unicode Standard}

The Unicode Standard represents a massive step forward because it aims to
eradicate the distinction between universal (ASCII) versus language-particular
(font) by adding as much language-specific information as possible into the
universal standard. However, there are still language/resource-specific
specifications necessary for the proper usage of Unicode, as will be discussed
below. Within the Unicode structure many of these specifications can be captured
by so-called \textsc{Unicode Locales}, so we are moving to a new distinction
of universal (Unicode Standard) versus language-particular (Unicode Locale).
The major gain is much larger compatibility on the universal level (because
Unicode standardizes a much greater portion of writing system diversity), and
much better possibilities for automated processing on the language-particular
level (because Unicode Locales are machine-readable specifications).

Each version of the Unicode Standard (\citealp{Unicode2018}, as of writing at
version 11) consists of a set of specifications and guidelines that include (i) a
core specification, (ii) code charts, (iii) standard annexes and (iv) a
character database.\footnote{All documents of the Unicode Standard are available
at \url{http://www.unicode.org/versions/latest/}. For a quick survey of the use
of terminology inside the Unicode Standard, their glossary is particularly
useful, available at \url{http://www.unicode.org/glossary/}. For a general
introduction to the principles of Unicode, Chapter 2 of the core specification,
called \textsc{general structure}, is particularly insightful. Unlike 
many other documents in the Unicode Standard, this general introduction is
relatively easy to read and illustrated with many interesting examples from
various orthographic traditions from all over the world.} The \textsc{core
specification} is a book aimed at human readers that describes the formal
standard for encoding multilingual text. The \textsc{code charts} provide a
human-readable online reference to the character contents of the Unicode
Standard in the form of PDF files. The \textsc{Unicode Standard Annexes (UAX)}
are a set of technical standards that describe the implementation of the Unicode
Standard for software development, Web standards, and programming languages.
The \textsc{Unicode Character Database (UCD)} is a set of computer-readable text
files that describe the character properties, including a set of rich character
and writing system semantics, for each character in the Unicode Standard. In
this section, we introduce the basic Unicode concepts, but we will leave out
many details. Please consult the above-mentioned full documentation for a more
detailed discussion. Further note that the Unicode Standard is exactly that,
namely a standard. It normatively describes notions and rules to be followed. In
the actual practice of applying this standard in a computational setting, a
specific implementation is necessary. The most widely used implementation of the
Unicode Standard is the \textsc{International Components for Unicode (ICU)},
which offers C/C++ and Java libraries implementing the Unicode
Standard.\footnote{More information about the ICU is available here:
\url{http://icu-project.org}.}

\section{Character encoding system}
\label{character-encoding-system}

The Unicode Standard is a \textsc{character encoding system} whose
goal is to support the interchange and processing of written characters and
text in a computational setting.\footnote{An insightful reviewer notes that the term \textsc{encoding} is used for both sequences of code points and text encoded as bit patterns. Hence a Unicode-aware programmer might prefer to say that UTF-8, UTF-16, etc., are Unicode encoding systems. The issue is that the Unicode Standard introduces a layer of indirection between characters and bit patterns, i.e.\ the code point, which can be encoded differently by different encoding systems.} Underlyingly, the character encoding is
represented by a range of numerical values called a \textsc{code space}, which
is used to encode a set of characters. A \textsc{code point} is a unique
non-negative integer within a code space (i.e.~within a certain numerical
range). In the Unicode Standard character encoding system, an \textsc{abstract
character}, for example the \textsc{latin small letter p}, is mapped to a
particular code point, in this case the decimal value 112, normally represented in
hexadecimal, which then looks in Unicode parlance as
\uni{0070}.
%\footnote{Hexadecimal (base-16) 0070 is equivalent to decimal
%(base-10) 112, which can be calculated by considering that $(0\cdot16^3) +
%(0\cdot16^2) + (7\cdot16^1) + (0\cdot16^0) = 7\cdot16 = 112$. Underlyingly,
%computers may treat this code point as an 8-bit binary (base-2) sequence (11100000), as
%can be seen by calculating that $(1\cdot2^7) + (1\cdot2^6) + (1\cdot2^5) +
%(0\cdot2^4) + (0\cdot2^3) + (0\cdot2^2) + (0\cdot2^1) + (0\cdot2^0) = 64 + 32 +
%16 = 112$.} 
That encoded abstract character is rendered on a computer screen (or
printed page) as a \textsc{glyph}, e.g.\ <p>, depending on the \textsc{font} and
the context in which that character appears.

In Unicode Standard terminology, an (abstract) \textsc{character} is the basic
encoding unit. The term \textsc{character} can be quite confusing due to its
alternative definitions across different scientific disciplines and because in
general the word \textsc{character} means many different things to different
people. It is therefore often preferable to refer to Unicode characters simply
as \textsc{code points}, because there is a one-to-one mapping between Unicode
characters and their numeric representation. In the Unicode approach, a
character refers to the abstract meaning and/or general shape, rather than a
specific shape, though in code tables some form of visual representation is
essential for the reader's understanding. Unicode defines characters as
abstractions of orthographic symbols, and it does not define visualizations for
these characters (although it does present examples). In contrast, a
\textsc{glyph} is a concrete graphical representation of a character as it
appears when rendered (or rasterized) and displayed on an electronic device or
on printed paper. For example, <g {\large \textit{g}} \textbf{g}
{\fontspec{ArialMT} {\small g} \textit{g} \textbf{g}}> are different glyphs of the
same character, i.e.~they may be rendered differently depending on the
typography being used, but they all share the same code point. From the
perspective of Unicode they are \textit{the same thing}. In this approach, a
\textsc{font} is then simply a collection of glyphs connected to code points.
Allography is not specified in Unicode (barring a few exceptional cases, due
to legacy encoding issues), but can be specified in a font as a
\textsc{contextual variant} (aka presentation form).

Each code point in the Unicode Standard is associated with a set of
\textsc{character properties} as defined by the Unicode character property
model.\footnote{The character property model is described in
\url{http://www.unicode.org/reports/tr23/}, but the actual properties are
described in \url{http://www.unicode.org/reports/tr44/}. A simplified overview
of the properties is available at
\url{http://userguide.icu-project.org/strings/properties}. The actual code
tables listing all properties for all Unicode code points are available at
\url{http://www.unicode.org/Public/UCD/latest/ucd/}.} Basically, those
properties are just a long list of values for each character. For example, code
point \uni{0047} has the following properties (among many others): 
\begin{itemize}
	\item Name: LATIN CAPITAL LETTER G 
	\item Alphabetic: YES 
	\item Uppercase: YES 
	\item Script: LATIN 
	\item Extender: NO 
	\item Simple\_Lowercase\_Mapping: 0067 
\end{itemize}

These properties contain the basic information of the Unicode Standard and they
are necessary to define the correct behavior and conformance required for
interoperability in and across different software implementations (as defined in
the Unicode Standard Annexes). The character properties assigned to each code
point are based on each character's behavior in real-world writing
traditions. For example, the corresponding lowercase character to \uni{0047} is
\uni{0067}.\footnote{Note that the relation between uppercase and lowercase is in
many situations much more complex than this, and Unicode has further
specifications for those cases.} Another use of properties is to define the
script of a character.\footnote{The Glossary of Unicode Terms defines the term \textsc{script} as
a ``collection of letters and other written signs used to represent textual
information in one or more writing systems. For example, Russian is written with
a subset of the Cyrillic script; Ukrainian is written with a different subset.
The Japanese writing system uses several scripts.''} In practice, script is
simply defined for each character as the explicit \textsc{script} property in
the Unicode Character Database.

One frequently referenced property is the \textsc{block} property, which is
often used in software applications to impose some structure on the large number
of Unicode characters. Each character in Unicode belongs to a specific block.
These blocks are basically an organizational structure to alleviate the
administrative burden of keeping Unicode up-to-date. Blocks consist of
characters that in some way belong together, so that characters are easier to
find. Some blocks are connected with a specific script, like the Hebrew block or
the Gujarati block. However, blocks are predefined ranges of code points, and
often there will come a point after which the range is completely filled. Any
extra characters will have to be assigned somewhere else. There is, for example,
a block \textsc{Arabic}, which contains most Arabic symbols. However, there is
also a block \textsc{Arabic Supplement}, \textsc{Arabic Presentation Forms-A}
and \textsc{Arabic Presentation Forms-B}. The situation with Latin symbols is
even more extreme. In general, the names of blocks should not be taken as a
definitional statement. For example, many IPA symbols are not located in the
aptly-named block \textsc{IPA extensions}, but in other blocks
(see Section~\ref{pitfall-no-complete-ipa-block}).

\section{Grapheme clusters}

There are many cases in which a sequence of characters (i.e.~a sequence of more
than one code point) represents what a user perceives as an individual unit in a
particular orthographic writing system. For this reason the Unicode Standard
differentiates between \textsc{abstract character} and \textsc{user-perceived
character}. Sequences of multiple code points that correspond to a single
user-perceived characters are called \textsc{grapheme clusters} in Unicode parlance.
Grapheme clusters come in two flavors: (default) grapheme clusters and tailored
grapheme clusters.

The (default) \textsc{grapheme clusters} are locale-independent graphemes,
i.e.~they always apply when a particular combination of characters occurs
independent of the writing system in which they are used. These character
combinations are defined in the Unicode Standard as functioning as one
\textsc{text element}.\footnote{The Glossary of Unicode Terms defines \textsc{text element} as:
``A minimum unit of text in relation to a particular text process, in the
context of a given writing system. In general, the mapping between text elements
and code points is many-to-many.''} The simplest example of a grapheme cluster
is a base character followed by a letter modifier character. For example, the
sequence <n> + <\dia{0303}> (i.e.~\textsc{latin small letter n} at \uni{006E}, followed
by \textsc{combining tilde} at \uni{0303}) combines visually into <ñ>, a
user-perceived character in writing systems like that of Spanish. In effect, what the
user perceives as a single character actually involves a multi-code-point
sequence. Note that this specific sequence can also be represented with a single
so-called \textsc{precomposed code point}, the \textsc{latin small letter n with
tilde} at \uni{00F1}, but this is not the case for all multi-code-point
character sequences. A solution to the problem of multiple encodings for the
same text element was developed early on in the Unicode Standard. It is called 
\textsc{canonical equivalence}, e.g.~for <ñ>, the sequence \uni{006E} \uni{0303} 
should in all situations be treated identically to the precomposed \uni{00F1}. By doing so, 
Unicode can also support special or precomposed characters in legacy character sets. 
To determine canonical equivalence, the Unicode Standard offers different kinds of normalization to either 
decompose precomposed characters (called \textsc{NFD} for \textsc{normalization form
canonical decomposition}) or to combine sequences of code points into precomposed 
characters (called \textsc{NFC} for \textsc{normalization form canonical composition}).\footnote{See the 
Unicode Standard Annex \#15, Unicode Normalization Forms (\url{http://unicode.org/reports/tr15/}), 
which provides a detailed description of normalization algorithms and illustrated examples.} 
In current software development practice, NFC seems to be preferred in most situations
and is widely proposed as the preferred canonical form. We discuss Unicode normalization 
in detail in Section \ref{pitfall-canonical-equivalence}.

More difficult for text processing, because less standardized, is what the
Unicode Standard terms \textsc{tailored grapheme clusters}.\footnote{\url{http://unicode.org/reports/tr29/}} 
Tailored grapheme clusters are locale-dependent graphemes, i.e.~such combination of characters do
not function as text elements in all situations. Examples include the sequence
<c>~+~<h> for the Slovak digraph <ch> and the sequence <ky> in the Sisaala
practical orthography (pronounced as IPA /tʃ/,~\cite{Moran2006}). These grapheme
clusters are \textsc{tailored} in the sense that they must be specified on a
language-by-language or writing-system-by-writing-system basis. They are also 
grapheme clusters in these orthographies for processes such as collation (i.e.\ sorting).\footnote{\url{https://www.unicode.org/glossary/\#collation}}

The Unicode Standard provides technical specifications for creating locale specific data
in so-called \textsc{Unicode Locales}, i.e.~specifications 
that define a set of language-specific elements (e.g.~tailored grapheme
clusters, collation order, capitalization-equivalence), as well as other special
information, like how to format numbers, dates, or currencies. Locale
descriptions are saved in the \textsc{Common Locale Data Repository
(CLDR)},\footnote{More information about the CLDR can be found here:
\url{http://cldr.unicode.org/}.} a repository of
language-specific definitions of writing system properties, each of which
describes specific usages of characters. Each locale can be encoded in a
document using the \textsc{Locale Data Markup Language (LDML)}. LDML is an XML
format and vocabulary for the exchange of structured locale data. Unicode Locale
Descriptions allow users to define language- or even resource-specific writing
systems or orthographies.\footnote{The Glossary of Unicode Terms defines \textsc{writing
system} only very loosely, as it is not a central concept in the Unicode
Standard. A writing system is, ``A set of rules for using one or more scripts to
write a particular language. Examples include the American English writing
system, the British English writing system, the French writing system, and the
Japanese writing system.''} However, Unicode Locales have various drawbacks 
for the daily practice of scientific linguistic research in a multilingual setting.
