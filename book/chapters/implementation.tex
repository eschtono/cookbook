\chapter{Implementation}
\label{implementation}

\section{Overview}
To illustrate the practical applications of orthography profiles, we have implemented two versions of the specifications presented in Chapter~\ref{orthography-profiles}: one in Python\footnote{\url{https://pypi.python.org/pypi/segments}} and one in R.\footnote{\url{https://github.com/cysouw/qlcData}} In this chapter, we introduce these two software libraries and provide practical step-by-step guidelines for installing and using them. Various simple and sometimes somewhat abstract examples will be discussed to show the different options available, and to illustrate the intended usage of orthography profiles in general. 

Note that our two libraries have rather different implementation histories, thus they may not give the same results in all situations (as discussed in Chapter~\ref{orthography-profiles}). However, we do provide extensive test suites for each implementation that follow standard practices to make sure that results are correct. Users should refer to these tests and to the documentation in each release for specifics about each implementation. Note that due to the different naming convention practices in Python and R, function names differ between the two libraries. Also, the performance with larger datasets may not be comparable between the Python and R implementations. In sum, our two libraries should be considered as proofs of concept and not as the final word on the practical application of the specifications discussed in the previous chapter. In our experience, the current versions are sufficiently fast and stable to be useful for academic practice (e.g.\ checking data consistency, or analyzing and transliterating small to medium sized data sets), but they should probably not be used for full-scale industry applications without adaptation.

First, in Section \ref{installing-python-and-r} we explain how to install Python\footnote{\url{https://www.python.org/}} and R.\footnote{\url{https://www.r-project.org/}} Then in Sections \ref{python-implementations} and \ref{r-implementation}, we discuss our Python and R software packages, respectively. In addition to the material presented here to get users started, we maintain several case studies online that illustrate how to use orthography profiles in action. For convenience, we make these recipes available as Jupyter Notebooks\footnote{\url{http://jupyter.org/}} in our GitHub repository.\footnote{\url{https://github.com/unicode-cookbook/}} In the final section in this chapter, we also briefly describe a few recipes that we do not go into detail in this book.

\section{How to install Python and R}
\label{installing-python-and-r}
When one encounters problems installing software, or bugs in programming code, search engines are your friend! Installation problems and incomprehensible error messages have typically been encountered and solved by other users. Try simply copying and pasting the output of an error message into a search engine; the solution is often already somewhere online. We are fans of Stack Exchange\footnote{\url{https://stackexchange.com/}} -- a network of question-and-answer websites -- which are extremely helpful in solving issues regarding software installation, bugs in code, etc.

Searching the Web for ``install r and python'' returns numerous tutorials on how to set up your machine for scientific data analysis. Note that there is no single correct setup for a particular computer or operating system. Both Python and R are available for Windows, Mac, and Unix operating systems from the Python and R project websites. Another option is to use a so-called package manager, i.e.\ a software program that allows the user to manage software packages and their dependencies. On Mac, we use Homebrew,\footnote{\url{https://brew.sh/}} a simple-to-install (via the Terminal App) free and open source package management system. Follow the instructions on the Homebrew website and then use Homebrew to install R and Python (as well as other software packages such as Git and Jupyter Notebooks). 

Alternatively for R, RStudio\footnote{\url{https://www.rstudio.com/}} provides a free and open source integrated development environment (IDE). This application can be downloaded and installed (for Mac, Windows and Unix) and it includes its own R installation and R libraries package manager. For developing in Python, we recommend the free community version of PyCharm,\footnote{\url{https://www.jetbrains.com/pycharm/}} an IDE which is available for Mac, Windows, and Unix. 

Once you have R or Python (or both) installed on your computer, you are ready to use the orthography profiles software libraries presented in the next two sections. As noted above, we make this material available online on GitHub,\footnote{\url{https://github.com/}} a web-based version control system for source code management. GitHub repositories can be cloned or downloaded,\footnote{\url{https://help.github.com/articles/cloning-a-repository/}} so that you can work through the examples on your local machine. Use your favorite search engine to figure out how to install Git on your computer and learn more about using Git.\footnote{\url{https://git-scm.com/}} In our GitHub repository, we make the material presented below (and more use cases described briefly in Section \ref{use-cases}) available as Jupyter Notebooks. Jupyter Notebooks provide an interface where you can run and develop source code using the browser as an interface. These notebooks are easily viewed in our GitHub repository of use cases.\footnote{\url{https://github.com/unicode-cookbook/recipes}}



\section{Python package: segments}
\label{python-implementations}

The Python package \texttt{segments} is available both as a command line interface (CLI) and as an application programming interface (API).

\subsection*{Installation}

To install the Python package \texttt{segments} from the Python Package Index (PyPI) run:

\begin{lstlisting}[language=bash]
  $ pip install segments
\end{lstlisting}

\noindent on the command line. This will give you access to both the CLI and programmatic functionality in Python scripts, when you import the \texttt{segments} library.

You can also install the \texttt{segments} package from the GitHub repository:\footnote{\url{https://github.com/cldf/segments}}

\begin{lstlisting}[language=bash]
  $ git clone https://github.com/cldf/segments
  $ cd segments
  $ python setup.py develop
\end{lstlisting}

\subsection*{Application programming interface}
The \texttt{segments} API can be accessed by importing the package into Python or by writing a Python script. Here is an example of how to import the libraries, create a tokenizer object, tokenize a string, and create an orthography profile.

\begin{lstlisting}
>>> from segments.tokenizer import Tokenizer
\end{lstlisting}

\noindent The \texttt{characters} function will segment a string at Unicode code points.

% \lstset{extendedchars=false, escapeinside=**}
\begin{lstlisting}[extendedchars=false, escapeinside={(*@}{@*)}]
>>> t = Tokenizer()
>>> result = t.characters('(*@ĉháɾã̌ctʼɛ↗ʐː| k͡p@*)')
>>> print(result)
>>> '(*@c ̂ h a ́ ɾ a ̃ ̌ c t ʼ ɛ ↗ ʐ ː | # k ͡ p@*)'
\end{lstlisting}

\noindent The \texttt{grapheme\_clusters} function will segment text at the Unicode Extended Grapheme Cluster boundaries.\footnote{\url{http://www.unicode.org/reports/tr18/tr18-19.html#Default_Grapheme_Clusters}}

\begin{lstlisting}[extendedchars=false, escapeinside={(*@}{@*)}]
>>> result = t.grapheme_clusters('(*@ĉháɾã̌ctʼɛ↗ʐː| k͡p@*)')
>>> print(result)
>>> '(*@ĉ h á ɾ ã̌ c t ʼ ɛ ↗ ʐ ː | # k͡ p@*)'
\end{lstlisting}

\noindent The \texttt{grapheme\_clusters} function is the default segmentation algorithm for the \texttt{segments.Tokenizer}. It is useful when you encounter a text that you want to tokenize to identify orthographic or transcription elements.

\begin{lstlisting}[extendedchars=false, escapeinside={(*@}{@*)}]
>>> result = t('(*@ĉháɾã̌ctʼɛ↗ʐː| k͡p@*)')
>>> print(result)
>>> '(*@ĉ h á ɾ ã̌ c t ʼ ɛ ↗ ʐ ː | # k͡ p@*)'
\end{lstlisting}

\noindent The \texttt{ipa} parameter forces grapheme segmentation for IPA strings.

\begin{lstlisting}[extendedchars=false, escapeinside={(*@}{@*)}]
>>> result = t('(*@ĉháɾã̌ctʼɛ↗ʐː| k͡p@*)', ipa=True)
>>> print(result)
>>> '(*@ĉ h á ɾ ã̌ c tʼ ɛ ↗ ʐː | # k͡p@*)'
\end{lstlisting}

\noindent We can also load an orthography profile and tokenize an input string with it. In the data directory, we've placed an example orthography profile. Let's have a look at it using \texttt{more} on the command line.


\begin{lstlisting}[extendedchars=false, escapeinside={(*@}{@*)}]
  $ more data/orthography-profile.tsv
  Grapheme  IPA   XSAMPA  COMMENT
  a         a     a
  aa        (*@aː@*)    (*@a:@*)
  b         b     b
  c         c     c
  ch        (*@tʃ@*)    tS
  -         NULL  NULL    "comment with   tab"
  on        (*@õ@*)     o~
  n         n     n
  ih        (*@í@*)     i_H
  inh       (*@ĩ́@*)     i~_H
\end{lstlisting}


\noindent An orthography profile is a delimited UTF-8 text file (here we use tab as a delimiter for reading ease). The first column must be labelled \texttt{Grapheme}. Each row in the \texttt{Grapheme} column specifies graphemes that may be found in the orthography of the input text. In this example, we provide additional columns IPA and XSAMPA, which are mappings from our graphemes to their IPA and XSAMPA transliterations. The final column \texttt{COMMENT} is for comments; if you want to use a tab ``quote that     string''!

Let's load the orthography profile with our tokenizer.

\begin{lstlisting}
>>> from segments.tokenizer import Profile
>>> t = Tokenizer('data/orthography-profile.tsv')
\end{lstlisting}

\noindent Now let's segment the graphemes in some input strings with our orthography profile. The output is segmented given the definition of graphemes in our orthography profile, e.g.\ we specified the sequence of two <a a> should be a single unit <aa>, and so should the sequences <c h>, <o n> and <i h>.


\begin{lstlisting}[extendedchars=false, escapeinside={(*@}{@*)}]
>>> t('(*@aabchonn-ih@*)')
>>> '(*@aa b ch on n - ih@*)'
\end{lstlisting}

\noindent This example shows how we can tokenize input text into our orthographic specification. We can also segment graphemes and transliterate them into other formats, which is useful when you have sources with different orthographies, but you want to be able to compare them using a single representation like IPA.

\begin{lstlisting}[extendedchars=false, escapeinside={(*@}{@*)}]
>>> t.transform('(*@aabchonn-ih@*)', 'IPA')
>>> '(*@aː b tʃ õ n í@*)'
\end{lstlisting}

% For some reason that fails me, this does not work:
% \begin{lstlisting}[extendedchars=false, escapeinside={(*@}{@*)}]
% >>> t.transform('aabchonn-ih', 'XSAMPA')
% >>> '(*@a: b tS o~ n i_H@*)'
% \end{lstlisting}

\begin{lstlisting}
>>> t.transform('aabchonn-ih', 'XSAMPA')
>>> 'a: b tS o~ n i_H'
\end{lstlisting}


\noindent It is also useful to know which characters in your input string are not in your orthography profile. Use the function \texttt{find\_missing\_characters}.

\begin{lstlisting}[extendedchars=false, escapeinside={(*@}{@*)}]
>>> t.find_missing_characters('(*@aa b ch on n - ih x y z@*)')
>>> '(*@aa b ch on n - ih � � �@*)'
\end{lstlisting}


\noindent We set the default as the Unicode replacement character \uni{fffd}, \texttt{replacement character}.\footnote{\url{http://www.fileformat.info/info/unicode/char/fffd/index.htm}} But you can simply change this by specifying the replacement character when you load the orthography profile with the tokenizer.


\begin{lstlisting}
>>> t = Tokenizer('data/orthography-profile.tsv', 
	errors_replace=lambda c: '?')
>>> t.find_missing_characters("aa b ch on n - ih x y z")
>>> 'aa b ch on n - ih ? ? ?'
\end{lstlisting}

\begin{lstlisting}
>>> t = Tokenizer('data/orthography-profile.tsv', 
	errors_replace=lambda c: '<{0}>'.format(c))
>>> t.find_missing_characters("aa b ch on n - ih x y z")
>>> 'aa b ch on n - ih <x> <y> <z>'
\end{lstlisting}

\noindent Perhaps you want to create an initial orthography profile that also contains those graphemes x, y, z? Note that the space character and its frequency are also captured in this initial profile.

\begin{lstlisting}
>>> profile = Profile.from_text("aa b ch on n - ih x y z")
>>> print(profile)
\end{lstlisting}


\begin{lstlisting}[language=bash,texcl=true]
  Grapheme  frequency  mapping
            9
  a         2          a
  h         2          h
  n         2          n
  b         1          b
  c         1          c
  o         1          o
  -         1          -
  i         1          i
  x         1          x
  y         1          y
  z         1          z
\end{lstlisting}


\subsection*{Command line interface}

From the command line, access \texttt{segments} and its 
various arguments. For help, run:

\begin{lstlisting}[language=bash]
  $ segments -h

usage: segments [-h] [--verbosity VERBOSITY] 
                     [--encoding ENCODING]
                     [--profile PROFILE]
                     [--mapping MAPPING]
                     command ...

Main command line interface of the segments package.

positional arguments:
  command               tokenize | profile
  args

optional arguments:
  -h, --help            show this help message and exit
  --verbosity VERBOSITY
                        increase output verbosity
  --encoding ENCODING   input encoding
  --profile PROFILE     path to an orthography profile
  --mapping MAPPING     column name in ortho profile to map 
                        graphemes

Use 'segments help <cmd>' to get help about individual commands.  
\end{lstlisting}

\noindent We have created some test data\footnote{\url{https://github.com/unicode-cookbook/recipes/tree/master/Basics/sources}} with the German word \textit{Schächtelchen}, which is the diminutive form of \textit{Schachtel}, meaning `box', `packet', or `carton' in English.

\begin{lstlisting}[language=bash]
  $ more sources/german.txt

  Schächtelchen
\end{lstlisting}

\noindent We can create an initial orthography profile of the German text by passing it to the \texttt{segments profile} command. The initial profile tokenizes the text on Unicode grapheme clusters, lists the frequency of each grapheme, and provides an initial mapping column by default.

\begin{lstlisting}[language=bash, extendedchars=false, escapeinside={(*@}{@*)}]
  $ cat sources/german.txt | segments profile

  Grapheme  frequency  mapping
  c         3          c
  h         3          h
  e         2          e
  S         1          S
  (*@ä@*)         1          (*@ä@*)
  t         1          t
  l         1          l
  n         1          n
\end{lstlisting}

\noindent Next, we know a bit about German orthography and which characters combine to form German graphemes. We can use the information from our initial orthography profile to hand-curate a more precise German orthography profile that takes into account capitalization (German orthography obligatorily capitalizes nouns) and grapheme clusters, such as <sch> and <ch>. We can use the initial orthography profile above as a starting point (note that, in large texts, the frequency column may signal errors in the input, such as typos, if a grapheme occurs with very low frequency). The initial orthography profile can be edited with a text editor or spreadsheet program. As per the orthography profile specifications (see Chapter \ref{orthography-profiles}), we can adjust rows in the \texttt{Grapheme} column and then add additional columns for transliterations or comments.


\begin{lstlisting}[language=bash, extendedchars=false, escapeinside={(*@}{@*)}]
  $ more more data/german-orthography-profile.tsv

  Grapheme  IPA  XSAMPA  COMMENT                     
  Sch       (*@ʃ@*)    S       German nouns are capitalized
  (*@ä@*)         (*@ɛː@*)   E:                                  
  ch        (*@ç@*)    C                                   
  t         t    t                                   
  e         e    e                                   
  l         l    l                                   
  n         n    n                                    
\end{lstlisting}

\noindent Using the command line \texttt{segments} function and passing it our orthography profile, we can now segment our German text example into graphemes.

\begin{lstlisting}[language=bash, extendedchars=false, escapeinside={(*@}{@*)}]
  $ cat sources/german.txt | segments 
    --profile=data/german-orthography-profile.tsv tokenize

  '(*@Sch ä ch t e l ch e n@*)'
\end{lstlisting}

\noindent By providing \texttt{segments} a column for transliteration, we can convert the text into IPA.

\begin{lstlisting}[language=bash, extendedchars=false, escapeinside={(*@}{@*)}]
  $ cat sources/german.txt | segments --mapping=IPA 
	--profile=data/german-orthography-profile.tsv tokenize

  '(*@ʃ ɛː ç t e l ç e n@*)'
\end{lstlisting}

\noindent And we can transliterate to XSAMPA.

\begin{lstlisting}[language=bash]
  $ cat sources/german.txt | segments --mapping=XSAMPA 
	--profile=data/german-orthography-profile.tsv tokenize

  'S E: C t e l C e n'
\end{lstlisting}

\noindent More examples are available online.\footnote{\url{https://github.com/unicode-cookbook/recipes}}

% Old example that reviewer didn't like!
\begin{comment}
\begin{lstlisting}[language=bash]
  $ more text.txt
  aäaaöaaüaa
\end{lstlisting}

\noindent Here is an example of how to create and use an orthography 
profile for segmentation. Create a text file:

\begin{lstlisting}[language=bash]
  $ more text.txt
  aäaaöaaüaa
\end{lstlisting}

\noindent Now look at the profile:

\begin{lstlisting}[language=bash,texcl=true]
  $ cat text.txt | segments profile
  Grapheme frequency mapping
  a        7         a
  ä        1         ä
  ü        1         ü
  ö        1         ö
\end{lstlisting}

\noindent Write the profile to a file:

\begin{lstlisting}[language=bash]
  $ cat text.txt | segments profile > profile.prf
\end{lstlisting}

\noindent Edit the profile:

\begin{lstlisting}[language=bash]
  $ more profile.prf
  Grapheme frequency mapping
  aa       0         x
  a        7         a
  ä        1         ä
  ü        1         ü
  ö        1         ö
\end{lstlisting}

\noindent Now tokenize the text without profile:

\begin{lstlisting}[language=bash]
  $ cat text.txt | segments tokenize
  a ä a a ö a a ü a a	
\end{lstlisting}

\noindent And with profile:
\begin{lstlisting}[language=bash]
  $ cat text.txt | segments --profile=profile.prf tokenize
  a ä aa ö aa ü aa

  $ cat text.txt | segments --mapping=mapping 
    --profile=profile.prf tokenize
  a ä x ö x ü x
\end{lstlisting}
\end{comment}
% \end old example that reviewer didn't like!



\section{R library: qlcData}
\label{r-implementation}

% \subsection*{Installing the R implementation}
\subsection*{Installation}

The R implementation is available in the package \texttt{qlcData}, which is 
directly available from the central R repository CRAN (Comprehensive R Archive 
Network). The R software environment itself has to be downloaded from its 
website.\footnote{\url{https://www.r-project.org}} After starting the included 
R program, the \texttt{qlcData} package for dealing with orthography profiles can be 
simply installed as follows:

<<eval=FALSE>>=
# download and install the qlcData software
install.packages('qlcData') 
# load the software, so it can be used
library(qlcData) 
@

The version available through CRAN is the latest stable version.
To obtain the most recent bug-fixes and experimental additions, please use the
development version, which is available on
GitHub.\footnote{\url{http://github.com/cysouw/qlcData}} This development
version can be easily installed using the github-install helper software from the
\texttt{devtools} package.

<<eval=FALSE>>=
# download and install helper software
install.packages('devtools') 
# install the qlcData package from GitHub
devtools::install_github('cysouw/qlcData', build_vignettes = TRUE)
# load the software, so it can be used 
library(qlcData) 
@

Inside the \texttt{qlcData} package, there are two functions for
orthography processing, \texttt{write.profile} and \texttt{tokenize}. The package includes
help files with illustrative examples, and also a so-called vignette with
explanations and examples.

<<eval=FALSE>>=
# view help files
help(write.profile)
help(tokenize)
# view vignette with explanation and examples
vignette('orthography_processing')
@

Basically, the idea is to use \texttt{write.profile} to produce a
basic orthography profile from some data and then \texttt{tokenize} to apply the
(possibly edited) profile on some data, as exemplified in the next section. This
can of course be performed though R, but additionally there are two more
interfaces to the R code supplied in the \texttt{qlcData} package: (i) bash
executables and (ii) \texttt{Shiny} webapps.

The bash executables are little files providing an interface to the R code that
can be used in a shell on a UNIX-like machine. The exact location of these
executables is best found after installation of R the packages. The
location can be found by the following command in R. 

<<eval=FALSE>>=
# show the path to the bash executables
file.path(find.package('qlcData'), 'exec')
@

These executables can be 
used in the resulting file path, or they can be linked and/or copied to any location as wanted. 
For example, a good way to use the executables in a terminal is to
make softlinks (using \texttt{ln}) from the executables to a directory in your
PATH, e.g.\ to \texttt{/usr/local/bin/}. The two executables are named
\texttt{tokenize} and \texttt{writeprofile}, and the links can be made directly 
by using Rscript to get the paths to the executables within the terminal.

<<eval=FALSE, tidy=FALSE, size='scriptsize', engine='bash'>>=
# get the paths to the R executables in bash
pathT=`Rscript -e 'cat(file.path(find.package("qlcData"), 
  "exec", "tokenize"))'`
pathW=`Rscript -e 'cat(file.path(find.package("qlcData"), 
  "exec", "writeprofile"))'`

# make softlinks to the R executables in /usr/local/bin
# you will have to enter your user's password!
sudo ln -is $pathT $pathW /usr/local/bin
@

After inserting this softlink it should be possible to access the
\texttt{tokenize} function from the shell. Try \texttt{tokenize --help} to test
the functionality.

% TODO:
% <<size='scriptsize', engine='bash', tidy=FALSE>>=
% tokenize --help
% @

% TODO
% To make the functionality even more accessible, we have prepared webapps with 
% the \texttt{shiny} framework for the R functions. These webapps are available 
% online at \url{TODO}. The webapps are also included inside the \texttt{qlcData} 
% package and can be started with the following helper function:

To make the functionality even more accessible, we have prepared webapps with 
the \texttt{Shiny} framework for the R functions. The webapps are 
included inside the \texttt{qlcData} package and can be started with the 
helper function (in R): \texttt{launch\_shiny('tokenize')}.

% <<eval=FALSE>>=
% launch_shiny('tokenize')
% @



\subsection*{Profiles and error reporting}
\label{error-reporting}

The first example of how to use these functions concerns finding errors in the
encoding of texts. In the following example, it looks as if we have two
identical strings, \texttt{AABB}. However, this is just a surface-impression
delivered by the current font, which renders Latin and Cyrillic capitals
identically. We can identify this problem when we produce an orthography profile
from the strings. Using the R implementation of orthography profiles, we
first assign the two strings to a variable \texttt{test}, and then produce an
orthography profile with the function \texttt{write.profile}. As it turns out,
some of the letters are Cyrillic.

<<>>=
( test <- c('AABB', '\u0041\u0410\u0042\u0412') )
write.profile(test)
@

The function of error-message reporting can also nicely be illustrated
with this example. Suppose we made an orthography profile with just the two
Latin letters <A> and <B> as possible graphemes, then this profile would not be
sufficient to tokenize the strings. There are graphemes in the data that are not
in the profile, so the tokenization produces an error, which can be used to fix
the encoding (or the profile). In the example below, we can see that the
Cyrillic encoding is found in the second string of the \texttt{test} input.

<<>>=
test <- c('AABB', '\u0041\u0410\u0042\u0412')
tokenize( test, profile = c('A', 'B') )
@

\subsection*{Different ways to write a profile}
\label{write-profile}

The function \texttt{write.profile} can be used to prepare a skeleton for an
orthography profile from some data. The preparation of an orthography profile
from some data might sound like a trivial problem, but actually there are
various different ways in which strings can be separated into graphemes by
\texttt{write.profile}. Consider the following string of characters called
\texttt{example} below. The default settings of \texttt{write.profile} separates
the string into Unicode graphemes according to grapheme clusters (called user-perceived characters; see Chapter~\ref{the-unicode-approach} for an explanation). The results are shown 
in Table \ref {tab:profile1}. As it 
turns out, some of these graphemes are single code points, others are combinations
of two code points (see Section~\ref{pitfall-characters-are-not-glyphs}).

<<>>=
example <- '\u00d9\u00da\u00db\u0055\u0300\u0055\u0301\u0055\u0302'
profile_1 <- write.profile(example)
@

<<results = 'asis', echo = FALSE>>=
colnames(profile_1) <- c('Gr.', 'Freq.', 'Codepoint', 'Unicode Name')
print(xtable(profile_1
        , caption = 'Profile 1 (default settings, splitting grapheme clusters)'
        , label = 'tab:profile1'
        )
  , include.rownames = FALSE
  , booktabs = TRUE
  , size = 'scriptsize'
  , table.placement = 'H'
  )
@

By specifying the splitting separator as the empty string
\texttt{sep~=~""}, it is possible to split the string into Unicode code points,
thus separating the combining diacritics. The idea behind this option
\texttt{sep} is that separating by a character allows for user-determined
separation. The most extreme choice here is the empty string \texttt{sep~=~""},
which is interpreted as separation everywhere. The other extreme is the default
setting \texttt{sep~=~NULL}, which means that the separation is not
user-defined, but relegated to the Unicode grapheme definitions. The result is 
shown in Table \ref{tab:profile2}.

<<>>=
profile_2 <- write.profile(example, sep = "")
@

<<results = 'asis', echo = FALSE>>=
colnames(profile_2) <- c('Grapheme', 'Frequency', 'Codepoint', 'Unicode Name')
print(xtable(profile_2
        , caption = 'Profile 2 (splitting by code points)'
        , label = 'tab:profile2'
        )
  , include.rownames = FALSE
  , booktabs = TRUE
  , size = 'scriptsize'
  , table.placement = 'H'
  )
@

Some characters look identical, although they are encoded differently.
Unicode offers different ways of normalization (see
Section~\ref{pitfall-canonical-equivalence}), which can be invoked here as well
using the option \texttt{normalize}. NFC normalization turns everything into the
precomposed characters, while NFD normalization separates everything into base
characters with combining diacritics. Splitting by code points (i.e.\ \texttt{sep~=~""}) 
shows the results of these two normalizations in Tables \ref{tab:profile3} \& \ref{tab:profile4}.

<<>>=
# after NFC normalization Unicode code points have changed
profile_3 <- write.profile(example, normalize = "NFC", sep = "")
# NFD normalization gives another structure of the code points
profile_4 <- write.profile(example, normalize = "NFD", sep = "")
@

<<results = 'asis', echo = FALSE>>=
colnames(profile_3) <- c('Grapheme', 'Frequency', 'Codepoint', 'Unicode Name')
print(xtable(profile_3
        , caption = 'Profile 3 (splitting by NFC code points)'
        , label = 'tab:profile3'
        )
  , include.rownames = FALSE
  , booktabs = TRUE
  , size = 'scriptsize'
  , table.placement = 'H'
  )
@

<<results = 'asis', echo = FALSE>>=
colnames(profile_4) <- c('Grapheme', 'Frequency', 'Codepoint', 'Unicode Name')
print(xtable(profile_4
        , caption = 'Profile 4 (splitting by NFD code points)'
        , label = 'tab:profile4'
        )
  , include.rownames = FALSE
  , booktabs = TRUE
  , size = 'scriptsize'
  , table.placement = 'H'
  )
@

It is important to realize that for Unicode grapheme definitions, NFC
and NFD normalization are equivalent. This can be shown by normalizing the
example in either NFD or NFC, as shown in Tables \ref{tab:profile5} \& \ref{tab:profile6}, 
by using the default separation in
\texttt{write.profile}. To be precise, default separation means setting
\texttt{sep~=~NULL}, but that has not be added explicitly below.

<<tidy = FALSE>>=
# note that NFC and NFD normalization are identical
# for Unicode grapheme definitions
profile_5 <- write.profile(example, normalize = "NFD")
profile_6 <- write.profile(example, normalize = "NFC")
@

<<results = 'asis', echo = FALSE>>=
colnames(profile_5) <- c('Gr.', 'Freq.', 'Codepoint', 'Unicode Name')
print(xtable(profile_5
        , caption = 'Profile 5 (splitting by graphemes after NFD)'
        , label = 'tab:profile5'
        )
  , include.rownames = FALSE
  , booktabs = TRUE
  , size = 'scriptsize'
  , table.placement = 'H'
  )
@

<<results = 'asis', echo = FALSE>>=
colnames(profile_6) <- c('Gr.', 'Freq.', 'Codepoint', 'Unicode Name')
print(xtable(profile_6
        , caption = 'Profile 6 (splitting by graphemes after NFC)'
        , label = 'tab:profile6'
        )
  , include.rownames = FALSE
  , booktabs = TRUE
  , size = 'scriptsize'
  , table.placement = 'H'
  )
@

These different profiles can also be produced using the bash
executable \texttt{writeprofile} (see above for how to install the bash executable). 
This example is also included in the help file of the executable.

% TODO: fix
% <<size='scriptsize', engine='bash'>>=
% writeprofile --help
% @


\subsection*{Using an orthography profile skeleton}
\label{profile-skeleton}

A common workflow to use these functions is to first make a skeleton for an
orthography profile and then edit this profile by hand. For example, Table
\ref{tab:profile_skeleton1} shows the profile skeleton after a few graphemes have
been added to the file. Note that in this example, the profile is written to the
desktop, and this file has to be edited manually. We simply add a few
multigraphs to the column \texttt{Grapheme} and leave the other columns empty.
These new graphemes are then included in the graphemic parsing.

<<eval=FALSE>>=
# a few words to be graphemically parsed
example <- c("mishmash", "mishap", "mischief", "scheme")
# write a profile skeleton to a file
write.profile(example, file = "~/Desktop/profile_skeleton.txt")
# edit the profile, and then use the edited profile to tokenize
tokenize(example, profile = "~/Desktop/profile_skeleton.txt")$strings
@

<<echo=FALSE>>=
example <- c("shampoo", "mishap", "mischief", "scheme")
profile_skeleton <- write.profile(example)
profile_skeleton <- rbind( c('sh','','','')
                         , c('ch','','','')
                         , c('sch','','','')
                         , c('ie','','','')
                         , c('oo','','','')
                         , profile_skeleton
                         )
tokenize(example, profile = profile_skeleton)$strings
@

<<results='asis', echo=FALSE>>=
print(xtable(profile_skeleton
        , caption = 'Manually edited profile skeleton'
        , label = 'tab:profile_skeleton1'
        )
  , include.rownames = FALSE
  , booktabs = TRUE
  , size = 'scriptsize'
  , table.placement = 'htb'
  )
@

To leave out the Unicode information in
the profile skeleton, use the option \texttt{info = FALSE}. It is also
possible not to use a separate file at all, but process everything within R. In
simple situations this is often useful (see below), but in general we prefer to
handle everything through a separately saved orthography profile. This profile
often contains highly useful information that is nicely coded and saved inside
this one file, and can thus be easily distributed and shared. Doing the same as
above completely within R might look as follows:

<<>>=
# make a profile, just select the column 'Grapheme'
profile <- write.profile(example)[,"Grapheme"]
# extend the profile with multigraphs
profile <- c("sh", "ch", "sch", "ie", "oo", profile)
# use the profile to tokenize
tokenize(example, profile)$strings
@

\subsection*{Rule ordering}
\label{rule-ordering}

Everything is not yet correct with the graphemic parsing of the example discussed
previously. The sequence <sh> in `mishap' should not be a digraph, and
conversely the sequence <sch> in `mischief' should of course be separated into
<s> and <ch>. One of the important issues to get the graphemic parsing right is
the order in which graphemes are parsed. For example, currently the grapheme
<sch> is parsed before the grapheme <ch>, leading to <m\ i\ sch\ ie\ f> instead
of the intended <m\ i\ s\ ch\ ie\ f>. The reason that <sch> is parsed before
<ch> is that by default longer graphemes are parsed before shorter ones. Our
experience is that in most cases this is expected behavior. You can change the
ordering by specifying the option \texttt{ordering}. Setting this option to
\texttt{NULL} results in no preferential ordering, i.e.\ the graphemes are parsed
in the order of the profile, from top to bottom. Now `mischief' is parsed
correctly, but `scheme' is wrong. So this ordering is not the solution in this
case.

<<eval=FALSE, tidy=FALSE>>= 
# do not reorder the profile
# just apply the graphemes from top to bottom
tokenize( example
         , profile = "~/Desktop/profile_skeleton.txt"
         , ordering = NULL
        )$strings
@

<<echo=FALSE>>=
tokenize( example, profile = profile_skeleton, ordering = NULL)$strings
@

There are various additional options for rule ordering implemented. Please check
the help description in R, i.e.\ \texttt{help(tokenize)}, for more details on the
possible rule ordering specifications. In summary, there are four different 
ordering options, that can also be combined:

\begin{itemize}
  
   \item \textsc{size}\\
         This option orders the lines in the profile by the size of the
         grapheme, largest first. Size is measured by number of Unicode
         characters after normalization as specified in the option
         \texttt{normalize}. For example, <é> has a size of 1 with
         \texttt{normalize = "NFC"}, but a size of 2 with
         \texttt{normalize = "NFD"}.

   \item \textsc{context}\\ This option orders the lines by whether they have
           any context specified (see next section). Lines with context will
           then be used first. Note that this only works when the option
           \texttt{regex = TRUE} is also chosen (otherwise context
           specifications are not used).

   \item \textsc{reverse}\\ This option orders the lines from bottom to top.
         Reversing order can be useful because hand-written profiles tend to put
         general rules before specific rules, which mostly should be applied in
         reverse order.

  \item \textsc{frequency}\\
         This option orders the lines by the frequency with which they
         match in the specified strings before tokenization, least frequent
         coming first. This frequency of course depends crucially on the
         available strings, so it will lead to different orderings when applied
         to different data. Also note that this frequency is (necessarily)
         measured before graphemes are identified, so these ordering frequencies
         are not the same as the final frequencies shown in the output.
         Frequency of course also strongly differs on whether context is used
         for the matching through \texttt{regex = TRUE}.
  
\end{itemize}

By specifying more than one ordering, these orderings are used to break ties,
e.g.\ the default setting \texttt{ordering = c("size", "context", "reverse")}
will first order by size, and for those with the same size, it will order by
whether there is any context specified or not. For lines that are still tied
(i.e.\ have the same size and both/neither have context) the order will be
reversed compared to the order as attested in the profile, because most
hand-written specifications of graphemes will first write the general rule,
followed by more specific regularities. To get the right tokenization, these 
rules should in most cases be applied in reverse order.

Note that different ordering of the rules does not result in 
feeding and bleeding effects found with finite-state rewrite
rules.\footnote{Bleeding is the effect that the application of a rule changes
the string, so as to prevent a following rule from applying. Feeding is the opposite: a
specific rule will only be applied because a previous rule changed the string
already. The interaction of rules with such feeding and bleeding effects is
extremely difficult to predict.} The graphemic parsing advocated here is 
crucially different from rewrite rules in that there is nothing being rewritten:
each line in an orthography profile specifies a grapheme to be captured in the 
string. All lines in the profile are processed in a specified order (as determined
by the option \texttt{ordering}). At the processing of a specific line, all 
matching graphemes in the data are marked as captured, but not changed. 
Captured parts cannot be captured again, but they can still be used to match 
contexts of other lines in the profile. Only when all lines are processed the 
captured graphemes are separated (and possibly transliterated). In this way the 
result of the applied rules is rather easy to predict.

To document a specific case of graphemic parsing, it is highly useful to save
all results of the tokenization to file by using the option \texttt{file.out},
for example as follows: 

<<eval=FALSE, tidy=FALSE>>= 
# save the results to various files
tokenize( example
         , profile = "~/Desktop/profile_skeleton.txt"
         , file.out = "~/Desktop/result"
        )
@

This will lead to the following four files being written. Crucially, a
new profile is produced with the re-ordered orthography profile. To reproduce
the tokenization, this re-ordered profile can be used with the option
\texttt{ordering~=~NULL}.

\begin{itemize}
  
   \item \textsc{result\_strings.tsv}:\\ A tab-separated file with the original
         and the tokenized/transliterated strings.

   \item \textsc{result\_profile.tsv}:\\ A tab-separated file with the
         graphemes with added frequencies of occurrence in the data. The lines
         in the file are re-ordered according to the order that resulted from the
         ordering specifications (see Section~\ref{rule-ordering}).

   \item \textsc{result\_errors.tsv}:\\ A tab-separated file with all original
         strings that contain unmatched parts. Unmatched parts are indicated
         with the character as specified with the option \texttt{missing}. By
         default the character \textsc{double question mark} <⁇> at
         \uni{2047} is used. When there are no errors, this file is 
         absent.

    \item \textsc{result\_missing.tsv}:\\ A tab-separated file with the graphemes
          that are missing from the original orthography profile, as indicated in
          the errors. When there are no errors, then this file is absent.
          
\end{itemize}

\subsection*{Contextually specified graphemes}
\label{contextual-specification}

To refine a profile, it is also possible to add graphemes with contextual
specifications. An orthography profile can have columns called \texttt{Left} and
\texttt{Right} to specify the context in which the grapheme is to be
separated.\footnote{The column names \textttf{Left}, \textttf{Right} and
\textttf{Grapheme} are currently hard-coded, so these exact column names
should be used for these effects to take place. The position of the columns in
the profile is unimportant. So the column \textttf{Left} can occur anywhere.}
For example, we are adding an extra line to the profile from above, resulting in
the profile shown in Table~\ref{tab:profile_skeleton2}. The extra line specifies
that <s> is a grapheme when it occurs after <mi>. Such contextually-specified
graphemes are based on regular expressions so you can also use regular
expressions in the description of the context. For such contextually specified
graphemes to be included in the graphemic parsing we have to specify the option
\texttt{regex = TRUE}. This contextually specified grapheme should actually be
handled first, so we could try \texttt{ordering = NULL}. However, we can also
explicitly specify that rules with contextual information should be applied
first by using \texttt{ordering = "context"}. That gives the right results for
this toy example, as shown in Table \ref{tab:profile_skeleton2}.

<<eval=FALSE, tidy=FALSE>>=
# add a contextual grapheme, and then use the edited 
# profile to tokenize
tokenize( example
         , profile = "~/Desktop/profile_skeleton.txt"
         , regex = TRUE
         , ordering = "context"
        )$strings
@

<<echo=FALSE>>=
Left <- rep('', times = nrow(profile_skeleton))
profile_skeleton <- as.matrix(cbind(Left, profile_skeleton))
profile_skeleton <- rbind(c('mi','s','','',''), profile_skeleton)
tokenize(example, profile = profile_skeleton, regex = TRUE, order='context')$strings
@

<<results='asis', echo=FALSE>>=
print(xtable(profile_skeleton
        , caption = 'Orthography profile with contextual specification for <s>'
        , label = 'tab:profile_skeleton2'
        )
  , include.rownames = FALSE
  , booktabs = TRUE
  , size = 'scriptsize'
  , table.placement = 'htb'
  )
@

Note that with the option \texttt{regex = TRUE} all
content in the profile is treated as regular expressions, so the characters with
special meaning in regular expressions should be either omitted or escaped (by
putting a <\ \backslash\ > \textsc{reverse solidus} at \uni{005C} before the
character). Specifically, this concerns the following characters:

\begin{itemize}
  
  \item[] <-> \textsc{hyphen-minus} at \uni{002D}
  \item[] <!> \textsc{exclamation mark} at \uni{0021}
  \item[] <?> \textsc{question mark} at \uni{003F}
  \item[] <.> \textsc{full stop} at \uni{002E}
  \item[] <(> \textsc{left parenthesis} at \uni{0028}
  \item[] <)> \textsc{right parenthesis} at \uni{0029}
  \item[] <[> \textsc{left square bracket} at \uni{005B}
  \item[] <]> \textsc{right square bracket} at \uni{005D}
  \item[] <\{> \textsc{left curly bracket} at \uni{007B}
  \item[] <\}> \textsc{right curly bracket} at \uni{007D}
  \item[] <|> \textsc{vertical line} at \uni{007C}
  \item[] <*> \textsc{asterisk} at \uni{002A}
  \item[] <\backslash> \textsc{reverse solidus} at \uni{005C}
  \item[] <ˆ> \textsc{circumflex accent} at \uni{005E}
  \item[] <+> \textsc{plus sign} at \uni{002B}
  \item[] <\$> \textsc{dollar sign} at \uni{0024}
  
\end{itemize}

\subsection*{Profile skeleton with columns for editing}
\label{profile-editing}

When it is expected that context might be important for a profile, then the
profile skeleton can be created with columns prepared for the contextual
specifications. This is done by using the option \texttt{editing = TRUE}~(cf.\
Table~\ref{tab:profile_editing_1} for a toy profile of some Italian words).

<<eval=FALSE, tidy=FALSE>>=
example <- c('cane', 'cena', 'cine')
write.profile(example
              , file = "~/Desktop/profile_skeleton.txt"
              , editing = TRUE
              , info = FALSE
              )
@

<<results='asis', echo=FALSE>>=
example <- c('cane', 'cena', 'cine')
profile_editing <- write.profile(example, editing=T, info=F)
print(xtable(profile_editing
        , caption = 'Orthography profile with empty columns for editing contexts'
        , label = 'tab:profile_editing_1'
        )
  , include.rownames = FALSE
  , booktabs = TRUE
  , size = 'scriptsize'
  , table.placement = 'htb'
  )
@

Besides the columns \texttt{Left}, \texttt{Grapheme}, and \texttt{Right} as
discussed in the previous sections, there are also columns \texttt{Class} and
\texttt{Replacement}. The column \texttt{Class} can be used to specify classes
of graphemes that can then be used in the contextual specification. The column
\texttt{Replacement} is just a copy of the column \textt{Grapheme}, providing a
skeleton to specify transliteration. The name of the column
\texttt{Replacement} is not fixed -- there can actually be multiple columns with 
different kinds of transliterations in a single profile.

To achieve contextually determined replacements it is possible to use a regular
expression in the contextual column. For example, consider the edited toy
profile for Italian in Table~\ref{tab:profile_editing_2} (where <c> becomes /k/
except before <i,e>, then it becomes /tʃ/). 

<<results='asis', echo=FALSE>>=
colnames(profile_editing)[5] <- 'IPA'
profile_editing[2,5] <- 'k'
profile_editing <- rbind(c('', 'c', '[ie]', '', 'tʃ'), profile_editing)
profile_editing <- profile_editing[c(1,2,6,3,4,5),]
print(xtable(profile_editing
        , caption = 'Orthography profile with regex as context'
        , label = 'tab:profile_editing_2'
        )
  , include.rownames = FALSE
  , booktabs = TRUE
  , size = 'scriptsize'
  , table.placement = 'htb'
  )
@

To use this profile, you have to add the option \texttt{regex = TRUE}. Also note
that we have changed the name of the transliteration column, so we have to tell
the tokenization process to use this column to transliterate. This is done by
adding the option \texttt{transliterate = "IPA"}.

<<eval=FALSE, tidy=FALSE>>=
# add a contextual grapheme, and then use the edited 
# profile to tokenize
tokenize( example
         , profile = "~/Desktop/profile_skeleton.txt"
         , regex = TRUE
         , transliterate = "IPA"
        )$strings
@

<<echo=FALSE>>=
tokenize(example
  , profile = profile_editing
  , regex = TRUE
  , transliterate = 'IPA'
  )$strings
@

Another equivalent possibility is to use a column \texttt{Class} to specify a
class of graphemes, and then use this class in the specification of context.
This is useful to keep track of recurrent classes in larger profiles. You are
free to use any class-name you like, as long as it does not clash with the rest
of the profile. The example shown in Table~\ref{tab:profile_editing_3} should 
give the same result as obtained previously by using a regular expression.

<<results='asis', echo=FALSE>>=
profile_editing[1,3] <- 'Vfront'
profile_editing[5,4] <- 'Vfront'
profile_editing[6,4] <- 'Vfront'
print(xtable(profile_editing
        , caption = 'Orthography profile with Class as context'
        , label = 'tab:profile_editing_3'
        )
  , include.rownames = FALSE
  , booktabs = TRUE
  , size = 'scriptsize'
  , table.placement = 'htb'
  )
@

<<eval=FALSE, tidy=FALSE>>=
# add a class, and then use the edited profile to tokenize
tokenize( example
         , profile = "~/Desktop/profile_skeleton.txt"
         , regex = TRUE
         , transliterate = "IPA"
        )$strings
@

<<echo=FALSE>>=
tokenize(example
  , profile = profile_editing
  , regex = TRUE
  , transliterate = 'IPA'
  )$strings
@

\subsection*{Formatting grapheme separation}
\label{formattingseparation}

In all examples above we have used the default formatting for grapheme
separation using space as a separator, which is obtained by the default setting
\texttt{sep~=~"~"}. It is possible to specify any other separator here,
including the empty string, i.e.\ \texttt{sep = ""}. This will not show the
graphemic tokenization anymore (although it has of course been used in the
background).

<<eval=FALSE, tidy=FALSE>>=
# Use the empty string as separator
tokenize( example
         , profile = "~/Desktop/profile_skeleton.txt"
         , regex = TRUE
         , transliterate = "IPA"
         , sep = ""
        )$strings
@

<<echo=FALSE>>=
tokenize(example
  , profile = profile_editing
  , regex = TRUE
  , transliterate = 'IPA'
  , sep = ''
  )$strings
@

Normally, the separator specified should not occur in the data. If it does,
unexpected things might happen, so consider removing the chosen separator from
your strings first. However, there is also an option \texttt{sep.replace} to
replace the separator with something else. When \code{sep.replace} is specified,
this mark is inserted in the string at those places where the separator occurs.
Typical usage in linguistics would be \texttt{sep = " ", sep.replace = "\#"} adding
spaces between graphemes and replacing spaces in the input string by hashes in
the output string.

<<tidy=FALSE>>=
# Replace separator in string to be tokenized
tokenize( "test test test"
         , sep = " "
         , sep.replace = "#"
        )$strings$tokenized
@

\subsection*{Remaining issues}

Given a set of graphemes, there are at least two different methods to tokenize
strings. The first is called \texttt{method = "global"}. This approach
takes the first grapheme in the profile, then matches this grapheme globally at
all places in the string, and then turns to process the next string in the profile. The 
other approach is called \texttt{method = "linear"}. This
approach walks through the string from left to right. At the first character it
looks through all graphemes whether there is any match, and then walks further
to the end of the match and starts again. This approach is more akin to
finite-state rewrite rules (though note that it still works differently from
such rewrite rules, as previously stated). The global method is used 
by default in the R implementation.

In some special cases these two tokenization methods can lead to different
results, but these special situations are very unlikely to happen in natural
language. The example below shows that a string \texttt{'abc'} can be parsed
differently in case of a very special profile with a very special ordering of
the graphemes.

<<tidy=FALSE>>=
# different parsing methods can lead to different results
# the global method first catches 'bc'
tokenize( "abc"
         , profile = c("bc","ab","a","c")
         , order = NULL
         , method = "global"
         )$strings
         
# the linear method catches the first grapheme, which is 'ab'
tokenize( "abc"
         , profile = c("bc","ab","a","c")
         , order = NULL
         , method = "linear"
         )$strings
@

Further, the current R implementation has a limitation when regular expressions
are used. The problem is that overlapping matches are not captured when using
regular expressions.\footnote{This restriction is an effect of the underlyingly
used ICU implementation of the Unicode Standard as implemented in R through the
package \textttf{stringi}.} Everything works as expected without regular
expressions, but there might be warnings/errors in case of \texttt{regex =
TRUE}. However, just as in the previous issue, this problem should only very
rarely (when at all) happen in natural language data.

The problem can be exemplified by a sequence <bbbb> in which a grapheme <bb>
should be matched. With the default \texttt{regex = FALSE} there are three
possible matches, but with \texttt{regex = TRUE} only the first two <b>'s or the
last two <b>'s are matched. The middle two <b>'s are not matched because they
overlap with the other matches. In the example below this leads to an error,
because the second <bb> is not matched. However, we have not been able to
produce a real example in any natural language in which this limitation might
lead to an error.

<<tidy=FALSE>>=
# Everything perfect without regular expressions
tokenize( "abbb"
        , profile = c("ab","bb")
        , order = NULL
        , regex = FALSE
        )$strings
        
# Matching with regular expressions does not catch overlap
tokenize( "abbb"
        , profile = c("ab","bb")
        , order = NULL
        , regex = TRUE
        )$strings
@


\section{Recipes online}
\label{use-cases}

We provide several use cases online -- what we refer to as \textit{recipes} -- that illustrate the applications of orthography profiles using our implementations in Python and R.\footnote{\url{https://github.com/unicode-cookbook/recipes}} Here we briefly describe these use cases and we encourage users to try them out using Git and Jupyter Notebooks.

First, as we discussed above, we provide a basic tutorial on how to use the Python \texttt{segments}\footnote{\url{https://pypi.python.org/pypi/segments}} and R \texttt{qlcData}\footnote{\url{https://github.com/cysouw/qlcData}} libraries. This recipe simply shows the basic functions of each library to get you started.\footnote{\url{https://github.com/unicode-cookbook/recipes/tree/master/Basics}}

The two recipes using the Python \textt{segments} package include a tutorial on how to segment graphemes in IPA text:

\begin{itemize}
	\item https://github.com/unicode-cookbook/recipes/tree/master/JIPA
\end{itemize}
	
\noindent and an example of how to create an orthography profile to tokenize fieldwork data from a large comparative wordlist.

\begin{itemize}
	\item https://github.com/unicode-cookbook/recipes/tree/master/Dogon
\end{itemize}

\noindent The JIPA recipes uses excerpts from \textit{The North Wind and the Sun} passages from the Illustrations of the IPA published in the Journal of the International Phonetic Alphabet. Thus the recipe shows how a user might tokenize IPA proper. The Dogon recipe uses fieldwork data from the Dogon languages of Mali language documentation project.\footnote{\url{http://dogonlanguages.org/}} This recipe illustrates how a user might tokenize fieldwork data from numerous linguists using different transcription practices by defining these practices with an orthography profile to make the output unified and comparable.

The two recipes using the R \texttt{qlcData} library include a use case for tokenizing wordlist data from the Automated Similarity Judgment Program (ASJP):\footnote{\url{http://asjp.clld.org/}}

\begin{itemize}
	\item https://github.com/unicode-cookbook/recipes/tree/master/ASJP
\end{itemize}

\noindent and for tokenizing a corpus of text in Dutch orthography:

\begin{itemize}
	\item https://github.com/unicode-cookbook/recipes/tree/master/Dutch
\end{itemize}

\noindent The ASJP use case shows how to download the full set of ASJP wordlists, to combine them into a single large CSV file, and to tokenize the ASJP orthography. The Dutch use case takes as input the 10K corpus for Dutch (``nld'') from the Leipzig Corpora Collection,\footnote{\url{http://wortschatz.uni-leipzig.de/en/download/}} which is then cleaned and tokenized with an orthography profile that captures the intricacies of Dutch orthography.

% In closing, using GitHub to share code and data provides a platform for sharing scientific results and it also promotes a means for scientific replicability of results. Moreover, we find that in cases where the scientists are building the to tools for analysis, open repositories and data help to ensure that what you see is what you get.

