\chapter{Preface}
\label{preface}

% ==========================
% \section{Preface}
% \label{preface}
% ==========================

% Early phoneticians: Richard Lepsius, Isaac Pitman, Alexander J. Ellis, and Alexander Melville Bell
% Later ones: Henry Sweet, L'abbé Jean-Pierre Rousselot, Eduard Sievers, Paul Passy, and Jens Otto Harry Jesperson

In the 1840s, several researchers and teachers began developing notation to accurately transcribe articulatory gestures and their acoustic correlates. Their goal was not only transcription, they wanted to use their notation for spelling reform (especially English), for teaching literacy and foreign languages, and in particular for developing writing systems for unwritten languages.

At first, the challenge of devising a universal phonetic notation seemed straightforward: normalize the relationship between graphemes and sounds (<c> in `cake' -> /k/; <c> in `cell' -> /s/) to remove inconsistencies introduced by the historical divergence of written and spoken forms (e.g.\ in English and French orthographies), and keep adding new symbol-sound mappings until all sounds are accounted for.

% These early transcription system pioneers found success with the first few languages that they encountered. Grapheme replacement and normalization of symbols across different orthographies seemed to work. But as they were confronted with more articulations and different notational devices, early phoneticians encountered a wealth of potentially important data that they had not expected.

This approach was successful for the first few languages investigated. But as the early phoneticians were confronted with more and more articulations, they encountered a wealth of data that they had not expected. The in-depth exploration of articulatory phonetics led to the discovery that the number of possible spoken sounds is infinite. \cite{Hockett1995} notes:

\begin{quote}
``But as they observed articulation more and more closely, and as they took more and more languages into consideration, the number of sounds they could discriminate and the number of symbols needed to denote them both increased seemingly without limit. In fact, as early as 1848, Ellis asserted that in principle the number of different sounds is endless. With the development of instrumental phonetics, largely by L'abbe Rousselt, it soon turned out that, given sufficiently accurate measuring devices, that is in fact the case.''
\end{quote}

% Hockett 1995: "And so the phoneticians began to develop notations that would show only the constant and distinctive features and let the transient and predictable ones be provided for by cover statements"

% The phonemic principle emerged and the model for it was alphabetic writing.

The first instrumental measurements in the 1870-80s showed unequivocally that spoken language is comprised of sounds waves; it is dynamic and not discrete. Given the fluid nature of the speech stream, sounds differ from one utterance to the next and the same sound may differ in different phonetic environments. However, many of these alterations are predictable in the context of their environments and they need not necessarily form distinctive lexical contrasts.

These findings left the early phoneticians in a conundrum given the aims of developing precise phonetic notion, but also practically-minded alphabetic-based writing. It is in this vein that the phonetic and phonemic distinction was introduced very early into transcription. The etic vs emic principle took hold, in essence, even before the dichotomy was widely accepted and before terminology for linguistic distinctiveness was introduced and adopted.

A transcription system that showed both contrastive and non-contrastive elements emerged, taking form notably in the International Phonetic Alphabet (which itself has been updated from time-to-time over the last nearly 120 years). The association between phonetic and phonemic would be maintained in the International Phonetic Alphabet and through the various instantiations of phonological theory, which developed on the work by early phoneticians.

The model for phonemic contrast was alphabetic writing, in particular shallow orthography, i.e.\ one-to-one mappings between sound and symbol. Traditional alphabetic orthography, however, did not scale to precise phonetic notation, so the former was sprinkled with diacritics to denote non-contrastive phenomena. This dichotomy between phonemic and phonetic and its implementation in writing and consistent encoding causes many issues -- linguistic, theoretic, and most recently technological -- which we bring to light in this book.
