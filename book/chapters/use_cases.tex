\chapter{Use cases}
\label{use-cases}

\section{Introduction}

We now present several use cases that have motivated the development of
orthography profiles. These include: 

\begin{itemize}
	\item tokenization and error checking
	\item normalization of orthographic systems to attain interoperability across different source documents, 
	\item cognate identification for detecting the similarity between words from different languages 
	\item ...
\end{itemize}

An important assumption of our work is that input sources are encoded in Unicode
(UTF-8 to be precise). Anything that the Unicode Standard is unable to capture,
cannot be captured by an orthography profile. We also use Unicode normalization
form NFD to decompose all incoming text input into normalized and (Unicode)
logically ordered strings. This process is of fundamental importance when
working with text data in the Unicode Standard.

% ==========================
\section{Tokenization and error checking}
\label{usecase-tokenization}
% ==========================


The most basic use case is tokenization of language data. Tokenization comes in three types:

\begin{itemize}
	\item Unicode character tokenization
	\item grapheme tokenization
	\item tailored grapheme tokenization
\end{itemize}
	
The simplest form of tokenization provided by any Unicode Standard-compliant
software will split an input text stream on Unicode character code points. The
input text is split on the character bit sequences as they have been encoded by
the user. This may be in various Unicode Normalization Forms (see Section
\ref{}) and tokenization split function returns a byte stream sequence given a
particular encoding (see Section \ref{}).

There is increasing support for the regular expression match, commonly ``\\X'',
to identify Unicode graphemes (see Section \ref{}) and perform tokenization on
these sets of Unicode characters. Examples are given in Section \ref{}.

The orthography profile provides the mechanism for tokenizing an input stream on
tailored grapheme clusters. The orthography's formal specification is given in
Section \ref{}. To sum, an orthography profile is a list of tailored grapheme
clusters and / or orthographic rules that specify how a specific resource of
text input should be tokenized. 

Once text has been normalized, an additional and straightforward process is
Unicode grapheme tokenization. More recently this regular expression, often
available as short cut ``\textbackslash{X}'', identifies all sequences of
``base'' characters followed by 0 or more combining diacritics. For example, the
sequence, % <ŋ͡mṍraː> would tokenize as a space-delimited sequence as: <ŋ͡ m ṍ r
a ː>. For a first pass at identifying grapheme clusters, this is a
straightforward tokenization. However, the linguist will note that both the tie
bar (COMBINING DOUBLE INVERTED BREVE, a Unicode Combining Modifier) and the
length marker (MODIFIER LETTER TRIANGULAR COLON. a Unicode Letter Modifier) do
not appear ``correctly'', i.e.~the tie bar is grouped with the first character
in the sequence and the length marker appears singly. This is necessary, as
defined by the Unicode Standard, because of these character's semantic
properties -- with Unicode we cannot know if, for example, the MODIFIER LETTER
TRIANGULAR COLON should appear before or after the base character that it
modifies, cf.~the IPA aspiration marker, , the MODIFIER LETTER SMALL H, which is
also a Unicode Letter Modifier and for linguists a diacritic that be be used for
pre- or post-aspiration. These ambiguities of position in tokenization are not
uncommon in IPA, thus orthographic (or source) normalization and tokenization is
needed.

\section{Cross-orthographic analysis of writing systems}

Given that orthography profiles are stored in a standard CSV format, we can use tools for converting and working with CSV. One such tool is the command line utility \textsc{csvkit}.\footnote{\url{https://csvkit.readthedocs.org}}

% insert some examples of creating a table of language | grapheme, language | orthographic contexts
% any measure of writing systems complexity?

\begin{comment}

\section{Cross-orthography **normalization*}

Orthographic tokenization requires a specification of the tailored grapheme
clusters (linguists: graphemes) to separate orthographic units within an input
source. In the simplest case, the Unicode graphemic tokenization with regular
expression ``\textbackslash{X}'' captures the grapheme clusters, i.e.~the input
data is graphemically tokenized and there are no additional string sequences
required to capture each grapheme, e.g.~no specification. This scenario is the
most unlikely.

The creation of an orthography typically requires that someone describes
knowledge about the input data source's writing system. For example, when
orthographically tokenizing a dictionary as input, the sequences of grapheme
clusters need to be explicitly stated or a regular expression rule can be
specified to transform the input. We illustrate these methods with examples from
Thiesen 1998, a bilingual dictionary of Diccionario Bora-Castellano, and with
Huber \& Huber 1992, a comparative vocabulary of sixty-nine indigenous languages
of Columbia.

Huber \& Huber 1992

% ({[}a\textbar{}á\textbar{}e\textbar{}é\textbar{}i\textbar{}í\textbar{}o\textbar{}ó\textbar{}u\textbar{}ú{]})(n)(\s)({[}a\textbar{}á\textbar{}e\textbar{}é\textbar{}i\textbar{}í\textbar{}o\textbar{}ó\textbar{}u\textbar{}ú{]}), \1 \2 \4

A preliminary orthography profile can be generated with code available
online.\footnote{\url{https://github.com/bambooforest/tokenizer}} The Python
script creates a unigram model for individual Unicode characters or Unicode
grapheme clusters and their frequency in the source input. We find that this
gives a broad overview of the input at the character and grapheme levels and it
often finds typographic errors in the source input by identifying rarely
occurring or one-off characters in the input. The models also provides the basis
for the ``quick'' development of an initial orthography profile by saving the
user keyboarding time.

Note that these symbols are still phonetically unspecified, i.e.~symbol may or
may not represent, say, a front open unrounded vowel.

The orthography profile is designed so that specified strings of characters are
captured for tokenization.

\% how the tokenization works

The graphemic cluster specifications in the orthography profile are read into a
trie data structure, which is used to tokenize the incoming input source by
identifying sentinels (tokenization boundaries) in the input stream, which we
then separate grapheme clusters by spaces.

An illustration is given in Figure N.

Comparative wordlists often start as the compilation of different sources with
different writing systems. The Dogon comparative wordlist is an example of a
compilation of slightly divergent transcription practices, which we will use to
illustrate our point and to orthographically normalize an input source by
transforming different transcriptions into a single IPA-like transcription.

*Orthographic normalization / cross-source **normalization*

Once we have normalized and tokenized our text elements, we can apply various
sorts of source normalization across resources to create interoperable and
comparable data.

Sound-based normalization is practical because \ldots{}

We use orthography profiles in our work to describe cross-linguistic difference
between writing systems.

\emph{Cognate detection}

The identification of cognates is a non-trivial task, which involves identifying
the semantic and phonetic similarity of words.

use grapheme correspondences to approximate cognacy and sound correspondences

to identify cognates and sound changes

Examples

Orthographic similarity != Phonetic similarity != Regular sound change

Cognate identification is further complicated by the fact that languages are
written differently, and even more so for the low resource languages that we
work with, linguists' phonetic transcriptions are heterogeneous and typically
document-specific, i.e.~two linguists working on the same language variety most
often develop different practical orthographies by using different symbols and
different orthographic rules. Thus the essence of language documentation and
description is diverse and ephemeral in nature. The encoding of linguistic data
and linguistic data models is diverse and abstracting away from individual
analytical preferences requires a document-by-document approach and a mechanism
for standardization of writing to form an interlingual pivot so that different
descriptions can be normalized into one encoding system to undertake analysis.

\% Orthography profile implementation in Lingpy / generic tokenizer

\% describe Tokenizer's function

\% describe algorithm for tokenizing IPA (Unicode grapheme cluster tokenization
and then alignment of Letter Modifiers and Tie Bars)

\% insert an example of orthographic tokenization from PAD

\section{move Letter Modifiers left}\label{move-letter-modifiers-left}

(\s)(\p{Lm}), \2

which catches:

\section{Diacritics: Unicode letter modifiers (Lm)}
\label{diacritics-unicode-letter-modifiers-lm}

\section{({[}ː\textbar{}ʰ\textbar{}ʼ\textbar{}ʿ\textbar{}ʾ{]})}\label{ux2d0ux2b0ux2bcux2bfux2be}

from the set of this:

\section{Diacritics: all}\label{diacritics-all}

\section{({[}ː\textbar{}ʰ\textbar{}ʼ\textbar{}ʿ\textbar{}ʾ\textbar{}̜\textbar{}̠\textbar{}̟\textbar{}͡\textbar{}̪\textbar{}̰\textbar{}̣\textbar{}̥\textbar{}̩\textbar{}‿\textbar{}.\textbar{}→\textbar{}˻\textbar{}+\textbar{}̃\textbar{}̫\textbar{}̤{]})}\label{ux2d0ux2b0ux2bcux2bfux2be.}

and allows us to with one regular expression do this:

PAD examples

A side effect of character tokenization is identification of outliers, e.g.

Some examples in Python

Some examples in R

{[}UNICODE USE-CASES{]}

The most basic overall Unicode character property is the \textsc{General
Category}, which categorizes Unicode characters into: \emph{Letters,
Punctuation, Symbols, Marks, Numbers, Separators, }and* Other*. Each Unicode
character property also has a character property value. So each General Category
is denoted by a property value by a single letter abbreviation, e.g. ``L'' for
Letter. Each General Category property also has subcategories that are also
properties with property values. For example, ``Letters'' is broken down into
Uppercase Letter (Lu), Lowercase Letter (Ll), Titlecase Letter (Lt), Modifier
Letter (Lm) and Other Letter (Lo).\footnote{The subcategory ``Other Letter''
includes characters such as uncased letters as in the Chinese writing system.}
By defining character properties for each code point, algorithmic
implementations such as text processing applications or regular expression
engines that are conformant to the Unicode Standard can be used to recognize
whole (sub)categories of characters or ranges of characters, e.g.~to match
proper names in English in title case or more generally to match character
types, such as scripts, or symbols and integers in email addresses, phone
numbers, etc.

Script information is useful for algorithmic processes, such as collation,
searching and for multilingual text processing. For example, if a regular
expression engine includes functionality to match characters at the level of
script, then it provides the ability to identify characters, e.g. \match{Greek},
which tests whether letters in a text belong to the Greek script or not. This
functionality can be used to determine the boundaries between different writing
systems that appear in the same document. In Section 6 (Use cases of grapheme
tokenization) we provide an example of a regular expression match at the level
of character property.

Unicode-aware regular expression engines may contain a meta-character,
``\textbackslash{X}'', that matches this level of grapheme cluster, i.e.~a base
character followed by some number of accents or diacritical marks. Grapheme
clusters are important to determine word or line boundaries, for collation
(`ordering') and for user interface interaction (e.g.~mouse selection,
backspacing, cursor movement).

\end{comment}

\section{PHOIBLE}

As new languages' phonological inventories are added to the 
database,\footnote{\url{http://phoible.org/}} updates and additions to the 
conventions are published online by~\cite{MoranMcCloy2014}.\footnote{
\url{http://phoible.github.io/conventions/}} , like 
introducing several symbols for widely attested sounds that lack representation 
in the current version of the IPA.\@ Whenever possible, additions were drawn from 
the extIPA symbols for disordered speech.\footnote{\url{https://www.internationalphoneticassociation.org/sites/default/files/extIPAChart2008.pdf}}

\cite{MoranMcCloy2014} explicitly define all character sequences so that the
vast variety of phonemes found in a large sample of descriptions of the world's
language were normalized into consistent character sequences, e.g.~if one
language description uses <kʲʼ> and another <kʼʲ>, when both are intended to be
phonetically equivalent, then a decision to normalize to one form was taken. 

Character sequences with diacritics above the base character were not
problematic in \citet{Moran2012} because they include only the centralized,
mid-centralized and nasalized combining characters. 

\citet{Moran2012;MoranMcCloy2014} mark tones as singletons with Space Modifier
Letters, e.g. <˦> for a phonemic high tone, instead of accent diacritics,
alleviating potential conflicts. 

\citet{Moran2012;MoranMcCloy2014} represent clicks as a combination of anterior and 
posterior articulations, where the set < k q ɡ ɢ ŋ ɴ > denote place of articulation 
and phonation type (voiced, voiceless, nasal). The anterior articulation click characters 
are < ʘ ǀ ǁ ǃ ǂ > with the addition of non-standard IPA < ‼ > \textsc{DOUBLE EXCLAMATION MARK)} 
\uni{203C} to represent the retroflex click. Note that the (post)alveolar click < ǃ > at \uni{01C3} 
is called incorrectly \textsc{LATIN LETTER RETROFLEX CLICK} (it is also a different character than 
the keyboard < ! > \textsc{EXCLAMATION MARK} at \uni{0021}). All sources that contain clicks in 
PHOIBLE are standardized to contain both the anterior and posterior characters and diacritics 
are added to the posterior glyph, e.g.\ < ŋ̊ǃ >.\footnote{http://phoible.github.io/conventions/}

Building on these works...



