%%%%%%%%%%%%%%%%%%%%%%%%%%%%%%%%%%%%%%%%%%%%%%%%%%%%
%%%                                              %%%
%%%     Language Science Press Master File       %%%
%%%         follow the instructions below        %%%
%%%                                              %%%
%%%%%%%%%%%%%%%%%%%%%%%%%%%%%%%%%%%%%%%%%%%%%%%%%%%%

% Everything following a % is ignored
% Some lines start with %. Remove the % to include them

\documentclass[output=inprep,
%  long|short|inprep
%  ,blackandwhite
%  ,smallfont
%  ,draftmode  
		biblatex
		]{LSP/langsci}\usepackage[]{graphicx}\usepackage[]{color}
%% maxwidth is the original width if it is less than linewidth
%% otherwise use linewidth (to make sure the graphics do not exceed the margin)
\makeatletter
\def\maxwidth{ %
  \ifdim\Gin@nat@width>\linewidth
    \linewidth
  \else
    \Gin@nat@width
  \fi
}
\makeatother

\definecolor{fgcolor}{rgb}{0.345, 0.345, 0.345}
\newcommand{\hlnum}[1]{\textcolor[rgb]{0.686,0.059,0.569}{#1}}%
\newcommand{\hlstr}[1]{\textcolor[rgb]{0.192,0.494,0.8}{#1}}%
\newcommand{\hlcom}[1]{\textcolor[rgb]{0.678,0.584,0.686}{\textit{#1}}}%
\newcommand{\hlopt}[1]{\textcolor[rgb]{0,0,0}{#1}}%
\newcommand{\hlstd}[1]{\textcolor[rgb]{0.345,0.345,0.345}{#1}}%
\newcommand{\hlkwa}[1]{\textcolor[rgb]{0.161,0.373,0.58}{\textbf{#1}}}%
\newcommand{\hlkwb}[1]{\textcolor[rgb]{0.69,0.353,0.396}{#1}}%
\newcommand{\hlkwc}[1]{\textcolor[rgb]{0.333,0.667,0.333}{#1}}%
\newcommand{\hlkwd}[1]{\textcolor[rgb]{0.737,0.353,0.396}{\textbf{#1}}}%

\usepackage{framed}
\makeatletter
\newenvironment{kframe}{%
 \def\at@end@of@kframe{}%
 \ifinner\ifhmode%
  \def\at@end@of@kframe{\end{minipage}}%
  \begin{minipage}{\columnwidth}%
 \fi\fi%
 \def\FrameCommand##1{\hskip\@totalleftmargin \hskip-\fboxsep
 \colorbox{shadecolor}{##1}\hskip-\fboxsep
     % There is no \\@totalrightmargin, so:
     \hskip-\linewidth \hskip-\@totalleftmargin \hskip\columnwidth}%
 \MakeFramed {\advance\hsize-\width
   \@totalleftmargin\z@ \linewidth\hsize
   \@setminipage}}%
 {\par\unskip\endMakeFramed%
 \at@end@of@kframe}
\makeatother

\definecolor{shadecolor}{rgb}{.97, .97, .97}
\definecolor{messagecolor}{rgb}{0, 0, 0}
\definecolor{warningcolor}{rgb}{1, 0, 1}
\definecolor{errorcolor}{rgb}{1, 0, 0}
\newenvironment{knitrout}{}{} % an empty environment to be redefined in TeX

\usepackage{alltt}

%%%%%%%%%%%%%%%%%%%%%%%%%%%%%%%%%%%%%%%%%%%%%%%%%%%%
%%%                                              %%%
%%%          additional packages                 %%%
%%%                                              %%%
%%%%%%%%%%%%%%%%%%%%%%%%%%%%%%%%%%%%%%%%%%%%%%%%%%%%

% put all additional commands you need in the 
% following files. If you do not know what this might 
% mean, you can safely ignore this section
\usepackage{verbatim}

%%%%%%%%%%%%%%%%%%%%%%%%%%%%%%%%%%%%%%%%%%%%%%%%%%%%
%%%                                              %%%
%%%                 Metadata                     %%%
%%%          fill in as appropriate              %%%
%%%                                              %%%
%%%%%%%%%%%%%%%%%%%%%%%%%%%%%%%%%%%%%%%%%%%%%%%%%%%%

\title{The Unicode Cookbook \linebreak for Linguists}
\subtitle{Managing writing systems using orthography profiles}
% \BackTitle{Change backtitle in  localmetadata.tex}
\BackBody{This text is a practical guide for linguists, and programmers, who work with data in multilingual computational environments. We introduce the basic concepts needed to understand how writing systems and character encodings function, and how they work together at the intersection between the Unicode Standard and the International Phonetic Alphabet. Although these standards are often met with frustration by users, they nevertheless provide language researchers and programmers with a consistent computational architecture needed to process, publish and analyze lexical data from the world's languages. Thus we bring to light common, but not always transparent, pitfalls which researchers face when working with Unicode and IPA. Having identified and overcome these pitfalls involved in making writing systems and character encodings syntactically and semantically interoperable (to the extent that they can be), we created a suite of open-source Python and R tools to work with languages using orthography profiles that describe author- or document-specific orthographic conventions. In this cookbook we describe a formal specification of orthography profiles and provide recipes using open source tools to show how users can segment text, analyze it, identify errors, and to transform it into different written forms for comparative linguistics research.}
% \dedication{Change dedication in localmetadata.tex}
\typesetter{Michael Cysouw, Steven Moran}
\proofreader{Sandra Auderset, Aleksandrs Berdičevskis, Rosey Billington, Varun deCastro-Arrazola, Simon Cozens, Aniefon Daniel, Bev Erasmus, Amir Ghorbanpour, Linda Leembruggen, David Lukeš, Antonio Machicao y Priemer, Hugh Paterson III, Stephen Pepper, Katja Politt, Felix Rau, Lea Schäfer, Benedikt Singpiel, James Tauber, Luigi Talamo, Jeroen van de Weijer, Viola Wiegand, Alena Witzlack-Makarevich, and Esther Yap}
\author{Steven Moran\lastand Michael Cysouw}
\renewcommand{\lsISBNdigital}{978-3-96110-090-3 }                     
\renewcommand{\lsISBNhardcover}{978-3-96110-091-0}                     
\renewcommand{\lsSeries}{tmnlp}  
\renewcommand{\lsSeriesNumber}{10}  
\renewcommand{\lsID}{176}  
\renewcommand{\lsBookDOI}{10.5281/zenodo.773250}
% add all extra packages you need to load to this file  

\usepackage{amsmath} 
\usepackage{unicode-math}
\setmathfont{Asana-Math.otf} % this looks much better for formulas
\usepackage[libertine]{newtxmath}

\usepackage{enumitem} % some additional possibilities for enumerations
\setitemize{noitemsep}
\setenumerate{noitemsep}

\usepackage{tabularx}
\usepackage{booktabs} % nice lines in tables
\usepackage{xtab} % xtabular for better multipage tables
\xentrystretch{-0.16} % squeeze more lines on a page
\usepackage{array} % for better handling of columns
\newcolumntype{L}[1]{>{\footnotesize\raggedright\let\newline\\\arraybackslash\hspace{0pt}}p{#1}}


\widowpenalty=10000
\clubpenalty=10000

% I have tried to use monospaced numbers in the TOC, but this does not work...
%\settocstylefeature{\fontspec[Numbers=Monospaced]{LinLibertineO}}
%\settocfeature{\fontspec[Numbers=Monospaced]{LinLibertineO}}

% other ideas that also don't give the desired results
%\usepackage{tocloft}
%\renewcommand{\cftXpagefont}{\fontspec[Numbers=Monospaced]{LinLibertineO}}


%%%%%%%%%%%%%%%%%%%%%%%%%%%%%%%%%%%%%%%%%%%%%%%%%%%%
%%%                                              %%%
%%%           Examples                           %%%
%%%                                              %%%
%%%%%%%%%%%%%%%%%%%%%%%%%%%%%%%%%%%%%%%%%%%%%%%%%%%% 

\usepackage{lsp-gb4e} 

%% to add additional information to the right of examples, uncomment the following line
% \usepackage{jambox}
%% if you want the source line of examples to be in italics, uncomment the following line
% \renewcommand{\exfont}{\itshape}

\usepackage{listings}

\lstset{ %
  backgroundcolor=\color{white},   % choose the background color; you must add \usepackage{color} or \usepackage{xcolor}
  basicstyle=\footnotesize\ttfamily,        % the size of the fonts that are used for the code 
  keywordstyle=\color{blue!60!black},       % keyword style
  language=XML,                 % the language of the code 
  stringstyle=\color{green!60!black},     % string literal style 
  morekeywords={token,xlink:href, Action, Value, Cursor,LogEvent}
} 
 
%% hyphenation points for line breaks
%% Normally, automatic hyphenation in LaTeX is very good
%% If a word is mis-hyphenated, add it to this file
%%
%% add information to TeX file before \begin{document} with:
%% %% hyphenation points for line breaks
%% Normally, automatic hyphenation in LaTeX is very good
%% If a word is mis-hyphenated, add it to this file
%%
%% add information to TeX file before \begin{document} with:
%% %% hyphenation points for line breaks
%% Normally, automatic hyphenation in LaTeX is very good
%% If a word is mis-hyphenated, add it to this file
%%
%% add information to TeX file before \begin{document} with:
%% \include{localhyphenation}
\hyphenation{
affri-ca-te
affri-ca-tes
com-ple-ments
graph-eme
graph-emes
}
\hyphenation{
affri-ca-te
affri-ca-tes
com-ple-ments
graph-eme
graph-emes
}
\hyphenation{
affri-ca-te
affri-ca-tes
com-ple-ments
graph-eme
graph-emes
}
%add all your local new commands to this file

\newcommand{\smiley}{:\)}

% to get the U+Hexdecimal abbreviations looking good
\newcommand{\uni}[1]
{\@{\small \fontspec[Letters=Uppercase]{LinLibertineO}U+#1}\@}

\newcommand{\unif}[1]
{\@{\footnotesize \fontspec[Letters=Uppercase]{LinLibertineO}U+#1}\@}

% use charisSIL in suitable size to fit other text
\newcommand{\charis}[1]
{{\small \fontspec{CharisSIL}#1}}

% mark lonely diacritics with a dotted circle
\newcommand{\dia}[1]
{{\fontspec{CharisSIL}{\large ◌}\symbol{"#1}}} % circle before diacritics

\newcommand{\diareverse}[1]
{{\fontspec{CharisSIL}\symbol{"#1}{\large ◌}}} % circle after diacritic

\newcommand{\diaf}[1]
{{\footnotesize \fontspec{CharisSIL}{\small ◌}\symbol{"#1}}} % small for footnote

% manually specifying space before and after knitr chunks
\renewenvironment{knitrout}
  {\vspace{-0.5em} } % what happens before the code chunk
  {\vspace{-1em} } % what happens after the code chunk

\bibliography{localbibliography}

%%%%%%%%%%%%%%%%%%%%%%%%%%%%%%%%%%%%%%%%%%%%%%%%%%%%
%%%                                              %%%
%%%             Frontmatter                      %%%
%%%                                              %%%
%%%%%%%%%%%%%%%%%%%%%%%%%%%%%%%%%%%%%%%%%%%%%%%%%%%%
\IfFileExists{upquote.sty}{\usepackage{upquote}}{}
\begin{document}
\maketitle
\frontmatter
% %% uncomment if you have preface and/or acknowledgements
% \addchap{Preface}
\begin{refsection}

%content goes here

\printbibliography[heading=subbibliography]
\end{refsection}


% \addchap{Acknowledgments}
\begin{refsection}

%content goes here

\printbibliography[heading=subbibliography]
\end{refsection}


% \addchap{List of abbreviations}
\begin{refsection}

%content goes here

\printbibliography[heading=subbibliography]
\end{refsection}


\tableofcontents
\mainmatter%

%%%%%%%%%%%%%%%%%%%%%%%%%%%%%%%%%%%%%%%%%%%%%%%%%%%%
%%%                                              %%%
%%%             knitr settings                   %%%
%%%                                              %%%
%%%%%%%%%%%%%%%%%%%%%%%%%%%%%%%%%%%%%%%%%%%%%%%%%%%%

% Here are settings for knitr, which will be removed in the .tex file
% spacing of code-chunks is set with the "knitrout" environment in localcommands.tex



%%%%%%%%%%%%%%%%%%%%%%%%%%%%%%%%%%%%%%%%%%%%%%%%%%%%
%%%                                              %%%
%%%             Chapters                         %%%
%%%                                              %%%
%%%%%%%%%%%%%%%%%%%%%%%%%%%%%%%%%%%%%%%%%%%%%%%%%%%%

\addchap{Preface}
\begin{refsection}

%content goes here

\printbibliography[heading=subbibliography]
\end{refsection}


\chapter{Writing Systems}
\label{writing_systems}

% \section{Introduction}
% \label{introduction}

Writing systems arise and develop in a complex mixture of cultural, technological and practical pressures. They tend to be highly conservative, in that people who have learned to read and write in a specific way (however impractical or tedious) are mostly unwilling to change their habits, e.g.~they tend to resist spelling reforms. In all literate societies there exists a strong socio-political mainstream that tries to force unification of writing (for example by strongly enforcing ``right'' from ``wrong'' writing in schools). However, there is also a large community of users who take as many liberties in their writing as they can get away with.

For example, the writing of tone diacritics in Yoruba is often proclaimed to be the right way to write, although many users of Yoruba writing seem to be perfectly fine with leaving them out. As pointed out by the proponents of the official rules, there are some homographs when leaving out the tone diacritics \citet[44]{Olumuyiw2013}. However, writing systems (and the languages they represent) are normally full of homophones, which is normally not a problem at all for speakers of the language. More importantly, writing is not just a purely functional tool, but just as importantly it is a mechanism to signal social affiliation. By showing that you \textit{know the rules} of expressing yourself in writing, others will more easily accept you as a worthy participant in their group. And that just as well holds for obeying to the official rules when writing a job application, as for obeying to the informal rules when writing an SMS to classmates in school. The case of Yoruba writing is an exemplary case, as even after more than a century of efforts to standardize the writing systems, there is still a wide range of variation in daily use \citet{Olumuyiw2013}.

The sometimes cumbersome and sometimes illogical structure, and the enormous variability of existing writing systems is a fact of life scholars have to accept and should try to adapt to as good as possible. Our plea here is a proposal for a formalization to do exactly that.

When considering the worldwide linguistic diversity, including all lesser-studied and endangered languages, there exist numerous different orthographies using symbols from the same scripts. For example, there are hundreds of orthographies using Latin-based alphabetic scripts. All of these orthographies use the same symbols, but these symbols differ in meaning and usage throughout the various orthographies. To be able to computationally use and compare different orthographies, we need a way to specify all orthographic idiosyncrasies in a computer-readable format (a process called \textsc{tailoring} in Unicode parlance). We call such specifications \textsc{orthography profiles}. Ideally, these specifications have to be integrated into so-called Unicode locale descriptions, though we will argue that in practice this is often not the most useful solution for the kind of problems arising in the daily practice of linguistics. Consequently, a central goal of this paper is to flesh out the linguistic challenges for locale descriptions, and work out suggestions to improve their structure for usage in a linguistic context. Conversely, we also aim to improve linguists' understanding and appreciation for the accomplishments of the Unicode Consortium in the development of the Unicode Standard.

The necessity to computationally use and compare different orthographies most forcefully arises in the context of language comparison. Concretely, in our current research our goal is to develop quantitative methods for language comparison and historical analysis in order to investigate worldwide linguistic variation and to model the historical and areal processes that underlie linguistic diversity, cf.~\citet{Steiner_etal2011,List2012,List2012a,ListMoran2013,MoranProkic2013}. In this work, it is crucial to be able to flexibly process across numerous resources with different orthographies. In many cases even different resources on the \textit{same} language use different orthographic conventions. Another orthographic challenge that we encounter regularly in our linguistic practice is electronic resources on a particular language that claim to follow a specific orthographic convention (often a resource-specific convention), but on closer inspection such resources are almost always not consistently encoded. Thus a second goal of our orthography profiles is to allow for an easy specification of orthographic conventions, and use such profiles to check consistency and to report errors to be corrected.

A central step in our proposed solution to this problem is the tailored grapheme separation of strings of symbols, a process we call \textsc{grapheme tokenization}. Basically, given some strings of symbols (e.g.~morphemes, words, sentences) in a specific source, our first processing step is to specify how these strings have to be separated into graphemes, considering the specific orthographic conventions used in a particular source document. Our experience is that such a graphemic tokenization can be performed without extensive in-depth knowledge about the phonetic and phonological details of the language in question. For example, the specification that $<$ou$>$ is a grapheme of English is a much easier task than to specify what exactly the phonetic values of this grapheme are in any specific occurrence in English words. Grapheme separation is a task that can be performed relatively reliably and with limited availability of time and resources (compare, for example, the task of creating a complete phonetic or phonological normalization).

Although grapheme tokenization is only one part of the solution, it is an important and highly fruitful processing step. Given a grapheme tokenization, various subsequent tasks become easier, like (a) temporarily reducing the orthography in a processing pipeline, e.g.~only distinguishing high versus low vowels; (b) normalizing orthographies across sources (often including temporary reduction of oppositions), e.g.~specifying an (approximate) mapping to the International Phonetic Alphabet; (c) using co-occurrence statistics across different languages (or different sources in the same language) to estimate the probability of grapheme matches, e.g.~with the goal to find regular sound changes between related languages or transliterations between different sources in the same language.

Before we deal with these proposals, in the first part of this paper (Sections \ref{encoding} through \ref{ipa-meets-unicode}) we give an extended introduction to the notion of encoding (Section \ref{encoding}) and writing systems, both from a linguistic perspective and from the perspective of the Unicode Consortium (Section \ref{terminology}). We consider the Unicode Standard to be a breakthrough (and ongoing) development that fundamentally changed the way we look at writing systems, and we aim to provide here a slightly more in-depth survey of the many techniques that are available in the standard. A good appreciation for the solutions that the Unicode Standard also allows for a thorough understanding of possible pitfalls that one might encounter when using it (Section \ref{unicode-pitfalls}). As an example of the current state-of-the-art, we discuss the rather problematic marriage of the International Phonetic Alphabet (IPA) with the Unicode Standard (Section \ref{ipa-meets-unicode}).

The second part of the paper (Sections \ref{orthography-profiles} and \ref{use-cases}) describes our proposals for how to deal with the Unicode Standard in the daily practice of (comparative) linguists. First, we discuss the challenges of characterizing a writing system. To solve these problems, we propose the notions of orthography profiles, closely related to Unicode locale descriptions (Section \ref{orthography-profiles}). Finally, we discuss practical issues with actual examples (Section \ref{use-cases}). We provide reference implementation of our proposals in R and in Python, available as open-source libraries.

The following conventions are followed in this paper. All phonemic and phonetic representations are given in the International Phonetic Alphabet (IPA), unless noted otherwise \citep{IPA2005}. Standard conventions are used for distinguishing between graphemic < >, phonemic / / and phonetic [ ] representations. For character descriptions, we follow the notational conventions of the Unicode Standard \citep{Unicode2014}. Character names are represented in small capital letters (e.g.~\textsc{latin small letter schwa}) and code points are expressed as U\emph{+n} where \emph{n} is a four to six digit hexadecimal number, e.g.~U+0256, which can be rendered as the glyph <ə>.
\chapter{Unicode pitfalls}
\label{unicode-pitfalls}

% ==========================
\section{Wrong it is not}
\label{wrong-it-is-not}
% ==========================

In this chapter we describe some of the most common pitfalls that we have
encountered when using the Unicode Standard in our own work, or in discussion
with other linguists. This section is not meant as a criticism of the decisions
made by the Unicode Consortium; on the contrary, we aim to highlight where the
technological aspects of the Unicode Standard diverge from many users'
intuitions. What have sometimes been referred to as problems or inconsistencies
in the Unicode Standard are mostly due to legacy compatibility issues, which can
lead to unexpected behavior by linguists using the standard. However, there are
also some cases in which the Unicode Standard has made decisions that
theoretically could have been taken differently, but for some reason or another
(mostly very good reasons) were accepted as they are now. We call behavior that
executes without error but does something different than the user
expected---often unknowingly---a \textsc{pitfall}.

In this context, it is important to realize that the Unicode Standard was not
developed to solve linguistic problems per se, but to offer a consistent
computational environment for written language. In those cases in which the
Unicode Standard behaves differently as expected, we think it is important not
to dismiss Unicode as ``wrong'' or ``deficient'', because our
experience is that in almost all cases the behavior of the Unicode Standard has
been particularly well thought through. The Unicode Consortium has a more
wide-ranging view of matters and often examines important practical use-cases
that from a linguistic point of view are normally not considered. Our general
guideline for dealing with the Unicode Standard is to accept it as it is, and
not to battle windmills. Alternatively, of course, it is possible to actively
engage in the development of the standard itself, an effort that is highly
appreciated by the Unicode Consortium.

% ==========================
\section{Pitfall: Characters are not glyphs}
\label{pitfall-characters-are-not-glyphs}
% ==========================

A central principle of Unicode is the distinction between character and glyph. A
character is the abstract notion of a symbol in a writing system, while a glyph
is the concrete drawing of such a symbol. In practice, there is a complex
interaction between characters and glyphs. A single Unicode character may of
course be rendered as a single glyph. However, a character may also be a piece
of a glyph, or vice-versa. Actually, all possible relations between glyphs and
characters are attested.

First, a single character may have different contextually determined glyphs. For
example, characters in writing systems like Hebrew and Arabic have different
glyphs depending on where they appear in a word. Some letters in Hebrew change
their form at the end of the word, and in Arabic, primary letters have four
contextually-sensitive variants (isolated, word initial, medial and final).
Second, a single character may be rendered as a sequence of multiple glyphs. For
example, in Tamil one Unicode character may result in a combination of a
consonant and vowel, which are rendered as two adjacent glyphs by fonts that
supports Tamil. Third, a single glyph may be a combination of multiple
characters. For example, the ligature <fi>, a single glyph, is the result of two
characters, <f> and <i>, that have undergone glyph substitution by font
rendering (see also Section~\ref{pitfall-faulty-rendering}). Like
contextually-determined glyphs, ligatures are (intended) artifacts of text
processing instructions. Finally, a single glyph may be a part of a
character, as exemplified by diacritics.

Further, the rendering of a glyph is dependent on the font being used. For
example, the Unicode character \textsc{latin small letter g} appears as <g> and
<{\fontspec{Courier}g}> in the Linux Libertine and Courier fonts, respectively,
because their typefaces are designed differently. Furthermore, font face may
change the visual appearance of a character, for example Times New Roman
two-story <{\fontspec{Times New Roman}a}> changes to a single-story glyph in italics
<\emph{\fontspec{Times New Roman}a}>. This becomes a real problem for some
phonetic typesetting (see Section~\ref{pitfall-ipa-homoglyphs}).

In sum, character-to-glyph mappings are complex technological issues that the
Unicode Consortium has had to address in the development of the Unicode
Standard, but for the lay user they can be utterly confusing because visual
rendering does not (necessarily) indicate logical encoding.

% ==========================
\section{Pitfall: Characters are not graphemes}
\label{pitfall-characters-are-not-graphemes}
% ==========================

The Unicode Standard is a character encoding system, and not a writing system
encoding system. This most forcefully becomes clear with the notion of grapheme.
From a linguistic point of view, graphemes are the basic building blocks of a
writing system (see Section~\ref{linguistic-terminology}). It is extremely
common, up to the point of being universally attested, that writing systems use
combinations of multiple symbols as a single grapheme, like <sch>, <th> or <ei>.
There is no possibility to encode such complex graphemes using the Unicode Standard.

The Unicode Standard deals with complex graphemes only inasmuch they consist of
base characters with diacritics (see
Section~\ref{pitfall-different-notions-of-diacritics} for a discussion of the
notion of diacritic). The Unicode Standard calls such combination ``grapheme
clusters''. Complex graphemes consisting of multiple base characters,
like <sch>, are called ``tailored grapheme clusters'' in Unicode parlance (see
Section~\ref{the-unicode-approach}).

Inspecting the Unicode Standard, there appear to be special Unicode characters
to ``glue'' together characters into larger tailored grapheme clusters,
specifically the \textsc{zero width joiner} at \uni{200D} and the
\textsc{combining grapheme joiner} at \uni{034F}. However, these characters are
confusingly named (cf.~Section~\ref{pitfall-names}). Both codepoints actually do
not join characters, but explicitly separate them. The zero-width joiner (ZWJ)
can be used to solve special problems related to ordering (called ``collation''
in Unicode parlance). The combining grapheme joiner (CGJ) can be used to
separate characters that are not supposed to form ligatures. 

To solve the issue of tailored grapheme clusters, Unicode offers some assistance
in the form of the Unicode Locale Descriptions. However, in the practice of
linguistic research, this is not a real solution. For that reason we propose to
use orthography profiles (see Chapter~\ref{orthography-profiles}). Basically,
both orthography profiles and locale descriptions offer a way to specify
tailored grapheme clusters. For example, for English one could specify that <sh>
is such a cluster. Consequently, this sequence of characters is then always
interpreted as a complex grapheme. For cases in which this is not the right
decision, like in the English word \textit{mishap}, the \textsc{zero width
joiner} at \uni{200D} has to be entered between <s> and <h>.

% ==========================
\section{Pitfall: Missing glyphs}
\label{pitfall-missing-glyphs}
% ==========================

The Unicode Standard is often praised (and deservedly so) for solving many of
the perennial problems with the interchange and display of the world's writing
systems. However, a common complaint from users is that, while the praise may be
true, they mostly just see some boxes on their screen instead of those promised
symbols. The problem of course is that users' computers do not have any glyphs
installed matching the Unicode code points in the file they are trying to
inspect. It is important to realize that internally in the computer everything
still works as expected: any handling of Unicode code points works independently
of how they are displayed on the screen. So, although a user might only see
boxes being displayed, this user should be assured that everything is still in
order.

The central problem behind the missing glyphs is that designing actual glyphs
includes a lot of different considerations and it is a time-consuming process.
Many traditional expectations of how specific characters should look like have
to be taken into account when designing glyphs. Those expectations are often not
well documented, and it is mostly up to the knowledge and experience of the font
designer to try and conform to them as good as possible. Therefore, most
designers produce fonts only including glyphs for certain parts of the Unicode
Standard, namely for those characters they feel comfortable with. At the same
time, the number of characters defined by the Unicode Standard is growing with
each new version, so it is neigh impossible for any designer to produce glyphs
for all characters. The result of this is that, almost necessarily, each font
only includes glyphs for a subset of the characters in the Unicode Standard.

The simple solution to missing glyphs is thus to install additional fonts
providing additional glyphs. For the more exotic characters there is often not
much choice. There are a few particularly large fonts that might be considered.
First, there is the \textsc{Everson Mono} font made by Michael Everson, which
currently includes 9,756 different glyphs (not including Chinese) updated up to
Unicode 7.0.\footnote{Everson Mono is available as shareware at
\url{http://www.evertype.com/emono/}.} Already a bit older is the \textsc{Titus
Cyberbit Basic} font made by Jost Gippert and Carl-Martin Bunz, which includes
10,044 different glyphs (not including Chinese), but not including newer
characters added after Unicode 4.0.\footnote{Titus Cyberbit Basic is available
at \url{http://titus.fkidg1.uni-frankfurt.de/unicode/tituut.asp}.}

Further, we suggest to always install at least one so-called \textsc{fall-back
font}, which provides glyphs that at least show the user some information about
the underlying encoded character. Apple Macintoshes have such a font (which is
invisible to the user), which is designed by Michael Everson and made available
for other systems through the Unicode Consortium.\footnote{The Apple/Everson
fallback font is available for non-Macintosh users at \newline
\url{http://www.unicode.org/policies/lastresortfont\_eula.html}.} Further, the
\textsc{GNU Unifont} is a clever way to produce bitmaps approximating the
intended glyph of each available character, updated to Unicode 7.0.\footnote{The
GNU Unifont is available at \url{http://unifoundry.com/unifont.html}.} Finally,
the Summer Institute of Linguistics provides a \textsc{SIL Unicode BMP Fallback
Font}, currently available up to Unicode version 6.1. This font does not even
attempt to show a real glyph, but only shows the hexadecimal code inside a box
for each character, so a user can at least see the Unicode codepoint of the
character to be displayed.\footnote{The SIL Unicode BMP Fallback Font is
available at \newline \url{http://scripts.sil.org/UnicodeBMPFallbackFont}.}

% ==========================
\section{Pitfall: Faulty rendering}
\label{pitfall-faulty-rendering}
% ==========================

A similar complaint to missing glyphs, discussed previously, is that while there
might be a glyph being displayed, it does not look right. There are two
reasons for unexpected visual display, namely automatic font substitution and
faulty rendering. Like missing glyphs, any such problems are independent from
the Unicode Standard. The Unicode Standard only includes very general
information about characters and leaves the specific visual display to others to
decide on. Any faulty display is thus not to be blamed on the Unicode
Consortium, but on a complex interplay of different mechanisms happening in a
computer to turn Unicode codepoints into visual symbols. We will only sketch a
few aspects of this complex interplay here.

Most modern software applications (like Microsoft Word) offer some approach to
\textsc{automatic font substitution}. This means that when a text is written in
a specific font (e.g.~Times New Roman) and an inserted Unicode character does not
have a glyph within this font, then the software application will automatically
search for another font to display the glyph. The result will be that this
specific glyph will look slightly different from the others. This mechanism
works differently depending on the software application, and mostly only limited
user influence is expected and little feedback is given, which might be rather
frustrating to font-aware users.\footnote{For example, Apple Pages does not give
any feedback that a font is being replaced, and the user does not seem to have
any influence on the choice of replacement (except by manually marking all
occurrences). In contrast, Microsoft Word does indicate the font replacement by
showing the name in the font menu of the font replacement. However, Word simply
changes the font completely, so any text written after the replacement is written in a
different font as before. Both behaviors leave much to be desired.}

The other problem with visual display is related to the so-called \textsc{font
rendering}. Font rendering refers to the process of the actual positioning of
Unicode characters on a page of written text. This positioning is actually a
highly complex problem, and many things can go wrong in the process. Well-known
rendering problems, like proportional glyph size or ligatures are reasonably
well understood. In contrast, the positioning of multiple diacritics relative to
a base character is still a widespread problem, even within the Latin script.
Especially when more than one diacritic is supposed to be placed above (or
below) each other, this often leads to unexpected effects in many modern
software applications. The problems arising in Arabic and in many southeast
Asian scripts (like Devanagari or Burmese) are even more complex. 

To understand where any problems arise it is important to realize that there are
basically three different approaches to font rendering. The most widespread is
Adobe's and Microsoft's \textsc{OpenType} system. This approach makes it
relatively easy for font developers, as the font itself does not include all
details about the precise placement of individual characters. For those details,
additional script-descriptions are necessary. All of those systems can lead to
unexpected behavior.\footnote{For more details about OpenType, see
\url{http://www.adobe.com/products/type/opentype.html} and
\url{http://www.microsoft.com/typography/otspec/}. Additional systems for
complex text layout are, among others, Microsoft's DirectWrite
\url{https://msdn.microsoft.com/library/dd368038.aspx} and the open-source
project HarfBuzz \url{http://www.freedesktop.org/wiki/Software/HarfBuzz/}.}
Alternative systems are \textsc{Apple Advanced Typography} (AAT) and the
open-source \textsc{Graphite} system from the Summer Institute of Linguistics
(SIL).\footnote{More information about AAT can be found at
\url{https://developer.apple.com/fonts/}. \newline SIL's Graphite is described
in detail at
\url{http://scripts.sil.org/cms/scripts/page.php?site\_id=projects\&item\_id=graphite\_home}.}
In both of these systems, a larger burden is placed on the description inside
the font.

There is mostly no real solution to problems arising from faulty font rendering.
Switching to another software application that offers better handling is the
only real alternative, but this is normally not an option for daily work. The 
experience with rendering on the side of the software industry is developing 
quickly, so we can expect the situation only to get better. In the meantime one 
can try to correct faulty layout by tweaking baseline and/or kerning (when such 
option are available).

% ==========================
\section{Pitfall: Blocks}
\label{pitfall-blocks}
% ==========================

The Unicode code space is subdivided into blocks of contiguous code points. For
example, the block called \textsc{Cyrillic} runs from \uni{0400} till
\uni{04FF}. These blocks arose as an attempt at ordering the enormous amount of
characters in Unicode, but the ideas of blocks very quickly ran into problems.
First, the size of a block is fixed, so when a block is full, a new block will
have to be instantiated somewhere further in the code space. For example, this
led to the blocks \textsc{Cyrillic Supplement}, \textsc{Cyrillic Extended-A}
(both of which are also already full) and \textsc{Cyrillic Extended-B}. Second,
when a specific character already exists, then it is not duplicated in another
block, although the name of the block might indicate that a specific symbol
should be available there. In general, names of blocks are just an approximate
indication of the kind of characters that will be in the block.

The problem with blocks arises because finding the right character among the
thousands of Unicode characters is not easy. Many software applications present
blocks as a primary search mechanism, because the block names suggest where to
look for a particular character. However, when a user searches for an IPA
character in the block \textsc{IPA Extensions}, then many IPA characters will not
be found there. For example, the velar nasal <ŋ> is not part of the block
\textsc{IPA Extensions} because it was already included as \textsc{latin small letter
eng} at \uni{014B} in the block \textsc{Latin Extensions-A}.

In general, finding a specific character in the Unicode Standard is often not
trivial. The names of the blocks can help, but they are not (and never were supposed
to be) a foolproof structure. It is not the goal nor aim of the Unicode
Consortium to provide a user interface to the Unicode Standard. If one often
encounters the problem of needing to find a suitable character, there are
various other useful services for end-users available.\footnote{The Unicode
website offers a basic interface to the code charts at
\url{http://www.unicode.org/charts/index.html}. As a more flexible interface, we
particularly like PopChar from Macility, available for both Macintosh and
Windows. There are also various free websites that offer search interfaces
to the Unicode code tables, like \url{http://unicode-search.net} or
\url{http://unicode-search.net}. A further useful approach for searching characters
using shape matching is \url{http://shapecatcher.com}.}

% ==========================
\section{Pitfall: Names}
\label{pitfall-names}
% ==========================

The names of characters in the Unicode Standard are sometimes misnomers and
should not be misinterpreted as definitions. For example, the \textsc{combining
grapheme joiner} at \uni{034F} does not join characters into larger graphemes
(see Section~\ref{pitfall-characters-are-not-graphemes}) and the \textsc{latin
letter retroflex click} \uni{01C3} is actually not the IPA symbol for a
retroflex click, but for an alveolar click (see
Section~\ref{pitfall-ipa-homoglyphs}). In a sense, these names can be seen as
``errors.'' However, it is probably better to realize that such names are just
convenience labels that are not going to be changed. Just like the block names
(Section~\ref{pitfall-blocks}), the character names are often helpful, but they
are not supposed to be definitions.

The actual intended ``meaning'' of a Unicode codepoint is a combination of the
name, the block and the character properties (see
Section~\ref{the-unicode-approach}). Further details about the underlying intentions 
with which a character should be used
are only accessible by perusing the actual decisions of the Unicode Consortium.
All proposals, discussions and decisions of the Unicode Consortium are publicly
available. Unfortunately there is not (yet) any way to easily find everything
that is ever proposed, discussed and decided in relation to a specific
codepoint of interest, so many of the details are often somewhat
hidden.\footnote{All proposals and other documents that are the basis of Unicode
decisions are avaialbe at \url{http://www.unicode.org/L2/all-docs.html}. The
actual decisions that make up the Unicode Standard are documented in the minutes
of the Unicode Technical Committee, available at
\url{http://www.unicode.org/consortium/utc-minutes.html}.}

% ==========================
\section{Pitfall: Homoglyphs}
\label{pitfall-homoglyphs}
% ==========================

Homoglyphs are visually indistinguishable glyphs (or highly similar glyphs) that
have different code points in the Unicode Standard and thus different character
semantics. As a principle, the Unicode Standard does not specify how a character
appears visually on the page or the screen. So in most cases, a different
appearance is caused by the specific design of a font, or by user-settings like
size or boldface. Taking an example already discussed in
Section~\ref{pitfall-homoglyphs}, the following symbols <g {\large \textit{g}}
\textbf{g} {\fontspec{ArialMT} {\small g} \textit{g} \textbf{g}}> are different
glyphs of the same character, i.e.~they may be rendered differently depending on
the typography being used, but they all share the same code point (viz.
\textsc{latin small letter g} at \uni{0067}). In contrast, the symbols
<{\fontspec{EversonMono}AАΑᎪᗅᴀꓮ𐊠𝖠𝙰}> are all different code points,
although they look highly similar---in some cases even sharing exactly the same
glyph in some fonts. All these different A-like characters include the following
code points in the Unicode Standard:

\begin{itemize}
	\item[] <{\fontspec{EversonMono}A}> \textsc{latin capital letter a}, at \uni{0041} 
	\item[] <{\fontspec{EversonMono}А}> \textsc{cyrillic capital letter a}, at \uni{0410} 
	\item[] <{\fontspec{EversonMono}Α}> \textsc{greek capital letter alpha}, at \uni{0391} 
	\item[] <{\fontspec{EversonMono}Ꭺ}> \textsc{cherokee letter go}, at \uni{13AA} 
	\item[] <{\fontspec{EversonMono}ᗅ}> \textsc{canadian syllabics carrier gho}, at \uni{15C5} 
	\item[] <{\fontspec{EversonMono}ᴀ}> \textsc{latin small letter capital a}, at \uni{1D00} 
	\item[] <{\fontspec{EversonMono}ꓮ}> \textsc{lisu letter a}, at \uni{A4EE} 
%	\item[] <{\fontspec{EversonMono}A}> \textsc{fullwidth latin capital letter a}, at \uni{FF21} 
	\item[] <{\fontspec{EversonMono}𐊠}> \textsc{carian letter a}, at \uni{102A0} 
%	\item[] <{\fontspec{EversonMono}𐌀}> \textsc{old italic letter a}, at \uni{10300} 
	\item[] <{\fontspec{EversonMono}𝖠}> \textsc{mathematical sans-serif capital a}, \uni{1D5A0} 
	\item[] <{\fontspec{EversonMono}𝙰}> \textsc{mathematical monospace capital a}, at \uni{1D670} 
\end{itemize}

The existence of such homoglyphs is partly due to legacy compatibility, but for
the most part these characters are simply different characters that happen to
look similar.\footnote{A particularly nice interface to look for homoglyphs is
\url{http://shapecatcher.com}, based on the principle of recognizing shapes
\citep{Belongie2002}.} Yet, they are supposed to behave different from the
perspective of a font designer. For example, when designing a Cyrillic font, the
<A> will have different aesthetics and different traditional expectation
compared to a Latin <A>.

Such homoglyphs are a widespread problem for consistent encoding. Although for
most users it looks like the words <voces> and <νοсеѕ> are almost identical, in
actual fact they do not even share a single code point.\footnote{The first words
consists completely of Latin characters, namely \unif{0076}, \unif{006F},
\unif{0063}, \unif{0065} and \unif{0073}, while the second is a mix of Cyrillic
and Greek characters, namely \unif{03BD}, \unif{03BF}, \unif{0041}, \unif{0435}
and \unif{0455}.} For computers these two words are completely different
entities. Commonly, when users with Cyrillic or Greek keyboards have to type
some Latin-based orthography, they mix similar looking Cyrillic or Greek
characters into their text, because those characters are so much easier to type.
Similarly, when users want to enter an unusual symbol, they normally search by
visual impression in their favorite software application, and just pick
something that looks reasonably alike to what they expect the glyph to look
like.

It is really easy to make errors at text entry and add characters that are 
not supposed to be included. Our proposals for orthography profiles (see
Chapter~\ref{orthography-profiles}) are a method for checking the consistency of 
any text. In situations in which interoperability is important, we consider it 
crucial to add such checks in any workflow.

% ==========================
\section{Pitfall: Canonical equivalence}
\label{pitfall-canonical-equivalence}
% ==========================

For some characters, there is more than one possible encoding in the Unicode
Standard. This is a possible pitfall, as this would mean that for the computer
there exist multiple different entities that for a user are the same. This
would, for example, lead to problems with searching, as the computer would
search for specific encodings, and not find all expected characters. As a
solution, the Unicode Standard includes a notion of \textsc{canonical
equivalence}. Different encodings are explicitly declared as equivalent in the
Unicode Standard code tables. Further, to harmonize all encodings in a specific
piece of text, the Unicode Standard proposes a mechanism of
\textsc{normalization}.

Consider for example the characters and following Unicode code points:
\begin{itemize}
	\def\labelenumi{\arabic{enumi}.} 
	\item <Å> \textsc{latin capital letter a with ring above} \uni{00C5} 
	\item <Å> \textsc{angstrom sign} \uni{212B}
	\item <Å> \textsc{latin capital letter a} \uni{0041}
	+ \textsc{combining ring above} \uni{030A}
\end{itemize}

The character, represented here by glyph <Å>, is encoded in the Unicode Standard
in the first two examples by a single-character sequence; each is assigned a
different code point. In the third example, the glyph is encoded in a
multiple-character sequence that is composed of two character code points. All
three sequences are \textsc{canonically equivalent}, i.e.~they are strings that
represent the same abstract character and because they are not distinguishable
by the user, the Unicode Standard requires them to be treated the same in
regards to their behavior and appearance. Nevertheless, they are encoded
differently. For example, if one were to search an electronic text (with
software that does not apply Unicode Standard normalization) for
\textsc{angstrom sign} (\uni{212B}), then the instances of \textsc{latin 
capital letter a with ring above} (\uni{00C5}) would not be found.

In other words, there are equivalent sequences of Unicode characters that should
be normalized, i.e.~transformed into a unique Unicode-sanctioned representation
of a character sequence called a \textsc{normalization form}. Unicode provides a
Unicode Normalization Algorithm, which essentially puts combining marks
into a specific logical order and it defines decomposition and composition
transformation rules to convert each string into one of four normalization
forms. We will discuss here the two most relevant normalization forms: NFC and
NFD.\@

The first of the three characters above is considered the \textsc{Normalization
Form C (NFC)}, where \textsc{C} stands for composition. When the process of NFC
normalization is applied to the character sequences in 2 and 3, both sequences
are normalized into the \textsc{pre-composed} character sequence in 1. Thus all
three canonical character sequences are standardized into one composition form
in NFC.\@The other central Unicode normalization form is the
\textsc{Normalization Form D (NFD)}, where \textsc{D} stands for decomposition.
When NFD is applied to the three examples above, all three, including
importantly the single-character sequences in 1 and 2, are normalized into the
\textsc{decomposed} multiple-sequence of characters in 3. Again, all three are
then logically equivalent and therefore comparable and syntactically
interoperable.

As illustrated, some characters in the Unicode Standard have alternative
representations (in fact, many do), but the Unicode Normalization Algorithm can
be used to transform certain sequences of characters into canonical
forms to test for equivalency. To determine equivalence, each
character in the Unicode Standard is associated with a combining class, which is
formally defined as a character property called \textsc{canonical combining
class} which is specified in the Unicode Character Database. The combining class
assigned to each code point is a numeric value between 0 and 254 and is used by
the Unicode Canonical Ordering Algorithm to determine which sequences of
characters are canonically equivalent. Normalization forms, as very briefly
described above, can be used to ensure character equivalence by ordering
character sequences so that they can be faithfully compared.

It is very important to note that any software applications that is Unicode
Standard compliant is free to change the character stream from one
representation to another. This means that a software application may compose,
decompose or reorder characters as its developers desire; as long as the
resultant strings are canonically equivalent to the original. This might lead to
unexpected behavior for users. Various players, like the Unicode Consortium, the
W{\large 3}C, or the TEI recommend NFC in most user-directed situations, and
some software applications that we tested indeed seem to automatically convert
strings into NFC.\footnote{See the summary of various recommendation here:
\url{http://www.win.tue.nl/~aeb/linux/uc/nfc_vs_nfd.html}.} This means in
practice that if a user, for example, enters <a> and <\dia{0300}>, i.e.~\textsc{latin
small letter a} at \uni{0061} and \textsc{combining grave accent} at \uni{0300},
this might be automatically converted into <à>, i.e.~\textsc{latin small letter
a with grave} at \uni{00E0}.\footnote{The behavior of software applications can
be quite erratic in this respect. For example, Apple's TextEdit does not do any
conversion on text entry. However, when you copy and paste some text inside the
same document in rich text mode (i.e.~RTF-format), it will be transformed into
NFC on paste. Saving a document does not do any conversion to the glyphs on
screen, but it will save the characters in NFC.}

% ==========================
\section{Pitfall: Absence of canonical equivalence}
\label{pitfall-absence-of-equivalence}
% ==========================

Although in most cases canonical equivalence will take care of alternative
encodings of the same character, there are a some cases in which the Unicode
Standard decided against equivalence. This leads to identical characters that
are not equivalent, like <ø> \textsc{latin small letter o with stroke} at
\uni{00F8} and <o̷> a combination of \textsc{latin small letter o} at \uni{006F}
with \textsc{combining short solidus overlay} at \uni{0037}.
The general rule followed is that extensions of Latin characters that are
connected to the base character are not separated as combining diacritics. For
example, characters like <ŋ ɲ ɳ> or <ɖ ɗ> are obviously derived from <n> and <d>
respectively, but they are treated like new separate characters in the Unicode
Standard. Likewise, characters like <ø> and <ƈ> are not separated into a base 
character <o> and <c> with an attached combining diacritic.

Interestingly, and somewhat illogically, there are three elements, which are
directly attached to their base characters, but which are still treated as
separable in the Unicode Standard. Such characters are decomposed (in NFD
normalization) in a base character with a combining diacritic. However, it is
these cases that should be considered the exceptions to the rule. These three 
elements are the following:

\begin{itemize}

  \item <\dia{0327}>: the \textsc{combining cedilla} at \uni{0327} \newline 
        This diacritic is
        for example attested in the precomposed character <ç> \textsc{latin
        small letter c with cedilla} at \uni{00E7}. This <ç> will thus be
        decomposed in NFC normalization.
  \item <\dia{0328}>: the \textsc{combining ogonek} at \uni{0328} \newline 
        This diacritic is
        for example attested in precomposed <ą> \textsc{latin small letter a
        with ogonek} at \uni{0105}. This <ą> will thus be decomposed in NFC
        normalization.
  \item <\dia{031B}>: the \textsc{combining horn} at \uni{031B} \newline 
        This diacritic is for
        example attested in precomposed <ơ> \textsc{latin small letter o with
        horn} at \uni{01A1}. This <ơ> will thus be decomposed in NFC
        normalization. 

\end{itemize}

There are further combinations that deserve specific care because it is actually
possible to produce identical characters in different ways without them being
canonically equivalent. In these situations, the general rule holds, namely that
characters with attached extras are not decomposed. However, in the following
cases the ``extras'' actually exist as combining diacritics, so there is also 
the possibility to construct a character by using a base character with those 
combining diacritics.

\begin{itemize}
  
  \item First, there are the combining characters designated as ``combining
        overlay'' in the Unicode Standard, like <\dia{0334}>
        \textsc{combining tilde overlay} at \uni{0334} or <\dia{0335}>
        \textsc{combining short stroke overlay} at \uni{0335}. There are many
        characters that look like they are precomposed with such an overlay,
        for example <\charis{ɫ~ᵬ~ᵭ~ᵱ}> or <\charis{ƚ~ɨ~ɉ~ɍ}>, or also the
        example of <ø> given at the start of this section. However, they are 
        not decomposed in NFD normalization.
  \item Second, the same situation also occurs with combining characters
        designated as ``combining hook'', like 
        <{\fontspec{CharisSIL}{\large ◌}}\symbol{"0321}> \textsc{combining
        palatalized hook below} at \uni{0321}. This element seems to occur in
        precomposed characters like <\charis{ᶀ~ᶁ~ᶂ~ᶄ}>. However, they are 
        not decomposed in NFD normalization.
        
\end{itemize}

To harmonize the encoding in these cases it is not sufficient to use Unicode 
normalization. Additional checks are necessary, for example by using orthography 
profiles (see Chapter~\ref{orthography-profiles}).

% ==========================
\section{Pitfall: File formats}
\label{pitfall-file-formats}
% ==========================

Unicode is a character encoding standard, but these characters of course
actually appear inside some kind of computer file. The most basic Unicode-based file
format is pure line-based text, i.e.~strings of Unicode-encoded characters
separated by line breaks (note that these line breaks are what for most people
intuitively corresponds to paragraph breaks). Unfortunately, even within this
apparently basic setting there exist a multitude of variants. In general, these
different possibilities are well-understood in the software industry, and
nowadays they normally do not lead to any problems for the end user. However,
there are some situations in which a user is suddenly confronted with cryptic
questions in the user interface involving abbreviations like LF, CR, BE, LE or
BOM.\@ Most prominently this occurs with exporting or importing data in several
software applications from Microsoft. Basically, there are two different issues
involved. First, the encoding of line breaks and, second, the encoding of the
Unicode characters into code units and the related issue of endianness.

\subsubsection*{Line breaks}

The issue with \textsc{line breaks} originated with the instructions necessary
to direct a printing head of a physical printer to a new line. This involves two
movements, known as \textsc{carriage return} (CR, returning the printing head to
the start of the line on the page) and \textsc{line feed} (LF, moving the
printing head to the next line on the page). Physically, these are two different
events, but conceptually together they form one action. In the history of
computing, various encodings of line breaks have been used (e.g.~CR+LF, LF+CR,
only LF, or only CR). Currently, all Unix and Unix-derived systems use only LF
as code for a line break, while software from Microsoft still uses a combination
of CR+LF.\@ Today, most software applications recognize both options, and
are able to deal with either encoding of line breaks (until rather recently this
was not the case, and using the wrong line breaks would lead to unexpected
errors). Our impression is that there is a strong tendency in software
development to standardize on the simpler ``only LF'' encoding for line
breaks, and we suggest that everybody use this encoding whenever possible.

\subsubsection*{Code units}

The issue with \textsc{code units} stems from the question how to separate a
stream of binary ones and zero, i.e.~bits, into chunks representing Unicode
characters. A code unit is the sequence of bits used to encode a single
character in an encoding. Depending on different use cases, the Unicode Standard
offers three different approaches, called UTF-32, UTF-16 and UTF-8.\footnote{The
letters UTF stand for \textsc{Unicode Transformation Format}, but the notion of
``transformation'' is a legacy notion that does not have meaning anymore.
Nevertheless, the designation UTF (in capitals) has become an official
standard designation, but should probably best be read as simply ``Unicode
Format.''} The details of this issue is extensively explained in section 2.5 of
the Unicode Core Specification~\citet{Unicode2014}. 

Basically, \textsc{UTF-32} encodes each character in 32 bits (32 \textit{bi}nary
uni\textit{ts}, i.e.~32 zeros or ones) and is the most disk-space-consuming
variant of the three. However, it is the most efficient encoding
processing-wise, because the computer simply has to separate each character
after 32 bits. 

In contrast, \textsc{UTF-16} uses only 16 bits per character, which is
sufficient for the large majority of Unicode characters, but not for all of
them. A special system of \textsc{surrogates} is defined within the Unicode Standard to
deal with these additional characters. The effect is a more disk-space efficient
encoding (approximately half the size), while adding a limited computational
overhead to manage the surrogates. 

Finally, \textsc{UTF-8} is a more complex system that dynamically encodes each
character with the minimally necessary number of bits, choosing either 8, 16 or
32 bits depending on the character. This represents again a strong reduction in
space (particularly due to the high frequency of data using erstwhile ASCII
characters, which need only 8 bits) at the expense of even more computation
necessary to process such strings. However, because of the ever growing
computational power of modern machines, the processing overhead is in most
practical situations a non-issue, while saving on space is still useful,
particularly for sending texts over the Internet. As an effect, UTF-8 has become
the dominant encoding on the World Wide Web. We suggest that everybody uses
UTF-8 as their default encoding.

A related problem is a general issue about how to store information in computer
memory, which is known as \textsc{endianness}. The details of this issue go
beyond the scope of this book. It suffices to realize that there is a
difference between \textsc{big-endian} (BE) storage and \textsc{little-endian}
(LE) storage. The Unicode Standard offers a possibility to explicitly indicate
what kind of storage is used by starting a file with a so-called \textsc{byte order
mark} (BOM). However, the Unicode Standard does not require the usage of BOM,
preferring other non-Unicode methods to signal to computers which kind of
endianness is used. This issue only arises with UTF-32 and UTF-16 encodings.
When using the preferred UTF-8, using a BOM is theoretically possible, but
strongly dispreferred according to the Unicode Standard. We suggest that
everyone tries to prevent the inclusion of BOM in your data.

% ==========================
\section{Recommendations}
\label{recommendations}
% ==========================

Summarizing the pitfalls discussed in this chapter, we propose the following 
recommendations:

\begin{itemize}
   \item To prevent strange boxes instead of nice glyphs, always install a few
         fonts with a large glyph collection and at least one fall-back font
         (see Section~\ref{pitfall-missing-glyphs}).
   \item Unexpected visual impressions of symbols does not necessarily mean that
         the actual encoding is wrong. It is mostly a problem of faulty
         rendering (see Section~\ref{pitfall-faulty-rendering}).
   \item Do not trust the names of codepoints as a definition of the character
         (see Section~\ref{pitfall-names}). Also do not trust Unicode blocks as
         a strategy to find specific characters (see
         Section~\ref{pitfall-blocks})
   \item To ensure consistent encoding of texts, apply Unicode normalization
         (NFC or NFD, see Section~\ref{pitfall-canonical-equivalence}).
   \item To prevent remaining inconsistencies after normalization, for example 
         stemming from homoglyphs (see Section~\ref{pitfall-homoglyphs}) 
         or from missing canonical equivalence in the Unicode Standard
         (see Section~\ref{pitfall-absence-of-equivalence})
         use orthography profiles (see Chapter~\ref{orthography-profiles}).
   \item To deal with ``tailored'' grapheme clusters
         (Section~\ref{pitfall-characters-are-not-graphemes}), use Unicode Locale 
         Descriptions, or orthography profiles 
         (see Chapter~\ref{orthography-profiles}).
   \item As a preferred file format, use Unicode Format UTF-8 in 
         Normalization Form Composition (NFC) with LF line endings, 
         but without byte order mark (BOM), whenever possible (see 
         Section~\ref{pitfall-file-formats}). This last nicely cryptic 
         recommendation has T-shirt potential:
  
\end{itemize}

\begin{center}
  I prefer it
  
  \textbf{UTF-8 NFC LF no BOM}
\end{center}



% ==========================
\chapter{The International Phonetic Alphabet}
\label{the-international-phonetic-alphabet}
% ==========================

\section{Brief history}

Established in 1886, the International Phonetic Association (henceforth
Association) has long maintained a standard alphabet, the International Phonetic
Alphabet (IPA), which is a common standard in linguistics to transcribe sounds
of spoken languages. It was first published in 1888 as an international system
of phonetic transcription for oral languages and for pedagogical purposes. It
contained phonetic values for English, French and German. Diacritics for length
and nasalization were already present in this first version, and the same
symbols are still used today. 
%\footnote{Also referred to as API, for \textit{Association Phonétique Internationale}.} 

Originally, the IPA alphabet was a list of symbols with pronunciation examples
from words in different languages. In 1900 the symbols were first organized into
a chart and were given phonetic feature labels for consonants (e.g.\ for manner
of articulation among others `plosives', `nasales', `fricatives' and for place
of articulation among others `bronchiales', `laryngales', `labiales') and for
vowels (e.g.\ `fermées', `mi-fermées', `mi-ouvertes', `ouvertes'). Throughout
the last century, the structure of the chart has changed with increases in
phonetic knowledge. Thus, similar to notational systems in other scientific
disciplines, the IPA reflects facts and theories of phonetic knowledge that have
developed over time. It is natural then that the IPA is modified occasionally to
accommodate scientific innovations and discoveries. In fact, this is part of the
Association's mandate. These changes are captured in the recurring revisions to
the IPA.
%\footnote{For a detailed history see:
%\url{https://en.wikipedia.org/wiki/History\_of\_the\_International\_Phonetic_Alphabet}}

Over the years there have been several revisions, many minor. For example, the
articulation labels -- what often are call `features' even though the IPA
deliberately avoids this term -- have changed (e.g.\ terms like `lips', `throat'
or `rolled' are not used anymore). Phonetic symbol values have changed (e.g.\
voiceless is no longer marked by <h>). Symbols have been dropped (e.g.\ the
caret diacritic denoting `long and narrow' is no longer used). And many symbols
have been added to reflect contrastive sounds found in the world's very diverse
phonological systems.

Although the IPA began its life as a pedagogical tool, from its earliest days the 
Association aimed to provide ``a separate sign for each distinctive sound; 
that is, for each sound which, being used instead of another, in the same 
language, can change the meaning of a word'' \citep[27]{IPA1999}. Distinctive 
sounds became later known as \textsc{phonemes} and the IPA has developed historically 
into a notational devise with a strictly segmented phonemic view. A phoneme is an 
abstract theoretical notion derived from an acoustic signal as produced by speakers 
in the real world. Therefore the IPA contains a number of theoretical assumptions 
about speech and how to analyze speech in written form. 

Phonetic analysis is based on two premises: (i) that it is possible to describe
the acoustic speech signal (i.e.\ sound waves) in terms of sequentially ordered
discrete segments, and, (ii) that each segment can be characterized by an
articulatory target.\footnote{A purely phonetic description is only derivable
from instrumental data in high quality sound recordings. Once spoken language
data are segmented, phonological consideration inextricably play a roll in
transcription. In other words, phonetic observations beyond quantitative
acoustic analysis are always made in terms of some phonological framework.}
Today, the IPA chart reflects a linguistic theory grounded in principles of
phonological contrast and in knowledge about the attested linguistic variation.
This fact is stated explicitly in several places, including in the
\textit{Report on the 1989 Kiel convention} published in the \textit{Journal of
the International Phonetic Association} \citep[67-68]{International1989report}:

\begin{quote}
The IPA is intended to be a set of symbols for representing all the possible 
sounds of the world's languages. The representation of these sounds uses a set 
of phonetic categories which describe how each sound is made. These categories 
define a number of natural classes of sounds that operate in phonological rules 
and historical sound changes. The symbols of the IPA are shorthand ways of 
indicating certain intersections of these categories.
\end{quote}

\noindent and in the \textit{Handbook of the International Phonetic Association} \citep[18]{IPA1993}: 

\begin{quote}
% The general value of the symbols in the chart is listed below. In each case 
[...] a symbol can be regarded as a shorthand equivalent to a phonetic
description, and a way of representing the contrasting sounds that occur in a
language. Thus [m] is equivalent to `voiced bilabial nasal', and is also a way
of representing one of the contrasting nasal sounds that occur in English and
other languages. [...] When a symbol is said to be suitable for the
representation of sounds in two languages, it does not necessarily mean that the
sounds in the two languages are identical. \end{quote}

\noindent Although the IPA provides symbols to unambiguously represent phonemes,
it also aims to represent phonetic details. Since phonetic detail could
potentially include anything, e.g. something like `deep voice', the IPA
restricts phonetic detail to linguistically relevant aspects of speech. These
principles and conventions for using the IPA are outlined in the Association's
Handbook.


\section{Principles}

IPA transcription has essentially two parts. The first is a text containing IPA 
symbols and the second is a set of conventions (rules) for interpreting those 
symbols (and their combinations). The IPA is designed to meet practical linguistic 
needs and is used to transcribe the phonetic or phonological structure of languages. 
It is also used increasingly as a foreign language learning tool, as a standard 
pronunciation guide and as a tool for creating practical orthographies of 
previously unwritten languages. 

The current construction and use of the IPA are guided by principles outlined 
in the \textit{Handbook of the International Phonetic Association: A guide 
to the use of the International Phonetic Alphabet} \citep[159]{IPA1999}, henceforth 
HB. The use of the symbols to represent a language's phonological system is guided 
by the principle of contrast; where two words are distinguishable by phonetic contrast, 
those contrasts should be transcribed with different symbols (graphemes not diacritics). 
Allophonic distinction falls under the rubric of diacritically-distinguished symbols, 
e.g. [stop̚] vs [spʰot]. In sum:

\begin{itemize}
	\item different symbols (without diacritics) should be used whenever a language employs two contrastive sounds
	\item when two sounds in a language are not known to be contrastive, the same symbol should be used to represent these sounds; however, diacritics may be used to distinguish such sounds when necessary
	\item diacritics cannot be dispensed with entirely, so the Association recommends to limit their use to:
	\begin{itemize}
		\item denoting length, stress and pitch
		\item representing minute shades of sounds
		\item obviating the design of a (large) number of new symbols when a single diacritic suffices (e.g. nasalized vowels, aspirated stops)
	\end{itemize}
\end{itemize}	

\noindent Thus, an IPA transcription \textit{always} consists of ``a set of 
symbols and a set of conventions for their interpretation''. 

In systematic transcription (as opposed to impressionistic), there is a division 
between phonemic and allophonic transcription. The terms phoneme and allophone 
contain theoretical baggage, but the basic goal of a phonemic transcription is to 
distinguish all words in a language with the minimal number of transcription 
symbols \citep[19]{Abercrombie1964}. Allophonic transcription uses a broader 
set of distinct symbols to describe systematic allophonic differences in sounds 
in words. These two systematic transcriptions are related to each other by a 
set of conventions in the IPA tradition, so that they can be converted between 
one and another.

An IPA transcription is connected to a speech event by a set of conventions. 
A phonetic (or impressionistic) transcription may use the conventions implicit 
in the IPA chart, i.e. the transcriber can indicate that the phonetic value of 
<ŋ͡m> is a simultaneous labial and velar closure which is voiced and contains 
nasal airflow. 

A phonemic transcription includes the conventions of a particular language's 
phonological rules. These rules determine the realization of that language's 
phonemes. However, there can be different systems of phonemic transcription 
for the same variety of a language. The differences may result from the fact 
that more than one phonetic symbol may be appropriate for a phoneme (see Section 
\ref{}). Or the differences may be due to different phonemic analyses, e.g.\ 
Standard German's vowel system is arguably contrastive in length or tenseness.

An important principle of the IPA is that different representations resulting 
from different symbols and different analyses are in line with the IPA's aims. 
In other words, the IPA does not provide phonological analyses for specific 
languages and the IPA does not define a single ``correct'' transcription system. 
Rather, the IPA aims to provide a resource that allows users to express any 
analysis so that it is widely understood. Thus the IPA suits many linguists' 
needs because:

\begin{itemize}
	\item it is intended to be a set of symbols for representing all possible sounds in the world's (spoken) languages
	\item its chart has a linguistic basis (specifically a phonological bias) rather than a general phonetic notation scheme
	\item its symbols can be used to represent distinctive feature combinations\footnote{Although the chart uses traditional manner and place of articulation labels, the symbols can nevertheless represent any defined bundle of features, binary or otherwise, to define phonetic dimensions.}
	\item its chart provides a summary of linguists' agreed-upon phonetic knowledge (a common denominator of phonological ``facts'')
\end{itemize}

Several styles of transcription with IPA are possible and the HB illustrates 
these and notes that they are all valid (see the 29 languages and their transcriptions 
in the original and initial \textit{Illustrations of the IPA} \citep[41--154]{IPA2007}). 
Therefore, there are different but equivalent transcriptions, or as \cite[64]{Ladefoged1990a} 
captures it, ``Perhaps now that the Association has been explicit in its eclectic 
approach, outsiders to the Association will no longer speak of \textit{the} IPA 
transcription of a given phenomenon, as if there were only one approved style.''.

Clearly not all phoneticians agree (or will likely ever agree of course) on all 
aspects of the IPA or on transcription practices. As noted above, there have 
been several revisions in the IPA's long history, but the current version (2005) 
is strikingly similar to the 1926 version. In 1989 an IPA revision convention 
was held in Kiel, Germany. As per other revisions, there was expansion and revisions 
to phonetic symbols in the IPA chart. Notably the marking of tone was extended 
with the addition of a second system for marking linguistic tones (Chao tones). 
Importantly, however, this was the first revision to address issues of 
computational representations for the IPA -- the principles above of which 
have had several ramifications that make the interoperability of electronic 
linguistic data extremely difficult.


\section{Computational representation}

Prior to the Kiel Convention for the modern revision of the IPA in 1989, 
\cite{Wells1987} collected and published practical approaches to coded 
representations of the IPA, which dealt mainly with the assignment of 
characters on the keyboard. The process of assigning standardized ``computer 
codes'' to phonetic symbols was assigned to the Workgroup on Computer 
Coding (henceforth working group) at the Kiel Convention. This working 
group was tasked with \citep{Esling1990,EslingGaylord1993}: 

\begin{itemize}
	\item determining how to represent the IPA numerically
	\item developing a set of numbers to refer to the IPA symbols unambiguously
	\item providing each symbol a unique name (intended to provide a mnemonic description of that character's shape)
\end{itemize}

\noindent The identification of IPA symbols with unique identifiers was 
a first step in formalizing the IPA computationally because it would give 
each symbol an unambiguous numerical identifier called an \textsc{IPA Number}. 
The numbering system was to be comprehensive enough to support future revisions 
of the IPA, including symbol specifications and diacritic placement. The 
application of diacritics was also to be made explicit. 

Although the Association had never officially approved a set of names 
for the IPA symbols, each IPA symbol received a unique \textsc{IPA Name}. 
Many symbols already had an informal name (or two) used by linguists, but 
consensus on symbol names was growing due to the recent publication of the 
\textit{Phonetic Symbol Guide} \citep{PullumLadusaw1986}. Thus most of the 
IPA symbol names were taken from \cite{PullumLadusaw1986} \citep[31]{IPA1999}.

The working group insightfully decided that the computing-coding convention 
for the IPA should be independent of computer environments or formats (e.g.\ ASCII), 
i.e.\ the IPA Number was not meant to be implemented directly in a computer 
encoding. The working group report's declaration includes the explanatory 
remarks \citep[82]{International1989report}:

\begin{quote}
The recommendation of a 7-bit ASCII or 8-bit extended-ASCII coding system 
would be short-sighted in view of development towards 16-bit and 32-bit 
processors. In fact, any specific recommendations would tie the Association 
to a stage of technological development which is bound to be outdated long 
before the next revision of the handbook.
\end{quote}

\noindent Thus the coding convention was not meant to address the engineering 
aspects of the actual encoding in computers (cf.\ \cite{Anderson1984}). However, 
it was meant to serve as a basis for a communication-interchange standard for 
creating mapping tables from various computer encodings, fonts, phonetic-character-set 
software, etc., to common IPA Numbers, and thus symbols.\footnote{Remember, at 
this time in the late 1980s there was no stable multilingual computing environment. 
But some solution was needed because scholars were increasingly using personal 
computers for their research and many were quickly adopting electronic mail or 
discussion boards like Usenet as a medium for international exchanges. This was 
before the Internet as we know it today. Most of these systems ran on 8-bit 
hardware systems using a 7-bit ASCII character encoding.}

Furthermore, the assignment of computer codes to IPA symbols was meant to 
represent an unbiased formulation. The Association plays the role of an international 
advisory body and it stated that it should not recommend a particular existing 
system of encoding. In fact, during this time there were a number of coding 
systems used, but none of them had a dominant international position. The 
differences between systems were also either too great or too subtle to warrant 
an attempt at combining them \citep{International1989report}.

The working group assigned each IPA symbol to a unique three-digit number, 
i.e.\ an IPA Number. Encoded in this number scheme is information about the 
status of each symbol (see below). The IPA numbers are listed with the IPA 
symbols and they are also illustrated in IPA chart form (see \cite[84]{EslingGaylord1993} 
or \cite[App. 2]{IPA2007}). The numbers were assigned in linear order (e.g.\ 
[p] 101, [b] 102, [t] 103...) following the IPA revision of 1989 and its update in 1996.

The working group made the decision that no IPA symbol, past or present, 
could be ignored. The comprehensive inclusion of all IPA symbols was to 
anticipate the possibility that some symbols might be added, withdrawn, 
or reintroduced into current or future usage. For example, in the 1989 
revision voiceless implosives < ƥ, ƭ, ƈ, ƙ, ʠ > were added; in the 1993 
revision they were removed. Ligatures like < ʧ, ʤ > are included as formerly 
recognized IPA symbols; they are assigned to the ``200 series'' of IPA numbers 
as members of the group of symbols formerly recognized by the IPA. To ensure 
backwards compatibility, legacy IPA symbols would retain an IPA Number and 
an IPA Name for reference purposes. As we discuss below, this decision is 
later reflected in the Unicode Standard; many legacy IPA symbols reside in 
the \textsc{IPA Extensions} block.

The IPA Number is simply expressed as a ``three-digit number numerical 
directory of digit triples'' \cite{}.\footnote{For practical purposes, 
the IPA Number also served as a typesetter's guide to the IPA chart.} 
The numbering scheme specifies three-digit codes, the first digit of which 
indicates the symbol's category \cite{Esling1990,EslingGaylord1993}:

\begin{itemize}
	\item 100s for accepted IPA consonants
	\item 200s for former IPA consonants and non-IPA symbols
	\item 300s for vowels
	\item 400s for segmental diacritics
	\item 500s for suprasegmental symbols
	\item 600s-800s for future specifications
	\item 900s for escape sequences
\end{itemize}

After a symbol is categorized, it is assigned a number sequentially, 
e.g.\ [i] 301, [e] 302, [ɛ] 303. The system allows for the addition 
of new symbols within the various series by appending them, e.g. [ⱱ] 
184. Former or often used but non-IPA symbols for consonants, vowels 
and diacritics are numbered from x99 backwards. For example, the voiceless 
and voiced postalveolar affricates and fricatives < č, ǰ, š, ž > are 
assigned the IPA numbers 299, 298, 297 and 296, respectively, because 
they are not sanctioned IPA symbols.

The assignment of the IPA numbers to IPA symbols provided the basis for 
uniquely identifying the set of past and present IPA symbols as a type of 
computational representational standard of the IPA. Within each revision 
of the IPA, the coding defines a closed and clearly defined set of characters. 
The benefits of this standardization are clear in at least two ways: it is 
used in translation tables that reference ASCII representations of the IPA, 
and this early computational representation of the IPA became the basis for 
its inclusion into the Unicode Standard version 1.0.

\section{SAMPA and X-SAMPA}

True to the working group's aim, the IPA numbers provided a mechanism for 
a communication interchange standard for creating mapping tables to various 
computer encodings. For example, the IPA coding system was used as a mapping 
system in the creation of SAMPA \citep{Wells_etal1992}, an ASCII representation 
of the IPA symbols. 

For a long time, linguists, like all other computer users, were
limited to ASCII-encoded 7-bit characters, which only includes Latin characters,
numbers and some punctuation and symbols. Restricted to these standard character
sets that lacked IPA support or other language-specific graphemes that they
needed, linguists devised their own solutions.\footnote{Early work addressing
the need for a universal computing environment for writing systems and their
computational complexity is discussed in \citet{Simons1989}. A survey of
practical recommendations for language resources, including notes on encoding,
can be found in \citet{BirdSimons2003}} For example, some chose to represent
unavailable graphemes with substitutes, e.g.~the combination of <ng> to
represent <ŋ>. Tech-savvy linguists redefined selected characters from a
character encoding by mapping custom made fonts to specific code points.\footnote{For 
example, SIL's popular font SIL IPA 1990}. However,
one linguist's electronic text would not render properly on another linguist's
computer without access to the same font. Furthermore, if two character encodings
defined two character sets differently, then data could not be reliably and
correctly displayed. This is a commonly encountered example of the non-interoperability of
data and data formats.

One solution was the ASCII-ification of the IPA, which simply involved 
defining keyboard-able sequences as IPA symbol codings.\footnote{\cite{Wells1987} 
provides an in-depth description of IPA codings from country-to-country. 
Later ASCII-IPAs include Kirshenbaum (created in 1992 in a Usenet group and 
named after its lead developer who was at Hewlett-Packard Laboratories) and 
Worldbet (published in 1993 by \cite{Hieronymus1993}, who was at AT\&T Laboratories).} 
A successful effort was SAMPA (Speech Assessment
Methods Phonetic Alphabet), which was created between 1988--1991 in Europe to 
represent IPA symbols with ASCII
character sequences \citep{Wells1987,Wells_etal1992}, e.g. <p\textbackslash> 
for [ɸ]. SAMPA was developed by a group of speech scientists from nine countries 
in Europe and it constituted the ASCII-IPA symbols needed for phonemic transcription 
of the principal European Union languages \citep{Wells1995}. It is still widely 
used in language technology.

Two problems with SAMPA are that (i) it is only a partial encoding of the IPA and (ii) it encodes different
languages in separate data tables, instead of using a universal alphabet, like
IPA.\@ SAMPA tables were developed as part of a European Commission-funded project to
address technical problems like electronic mail exchange (what is now simply
called email). SAMPA is essentially a hack to work around displaying IPA
characters, but it provided speech technology and other fields a basis that has
been widely adopted and often still used in code. So, SAMPA is a collection of tables to be
compared, instead of a large universal table representing all languages. 

An extended version of SAMPA, called X-SAMPA, set out to include every symbol, 
including all diacritics, in the IPA chart \citep{Wells1995}. X-SAMPA is considered
more universally applicable because it consists of one table that encodes all
characters in IPA. In line with the principles of the IPA, SAMPA and X-SAMPA include a 
repertoire of symbols. These symbols are intended to represent phonemes rather than 
all allophonic distinctions. Additionally, both ASCII-ifications of IPA are useful because 
strings of SAMPA or X-SAMPA are (reportedly) uniquely parsable \citep{Wells1995}. However, 
like the IPA, X-SAMPA has different notations for encoding the same phonetic phenomena 
(see Pitfall \ref{}).

SAMPA and X-SAMPA have been widely used for speech technology and as an encoding system in
computational linguistics. In fact, they are stilled used in popular software packages 
that require ASCII input and some of which have been co-opted for linguistic analyses, 
e.g.~RuG/L04 and SplitsTree4.\footnote{See \url{http://www.let.rug.nl/kleiweg/L04/} and 
\url{http://www.splitstree.org/}, respectively}


\section{The need for a single multilingual environment}

During the 1980's, it became increasingly clear that an adequate solution 
to the problem of multilingual computing environments was needed. Linguists 
were on the forefront of addressing this issue because they faced these 
challenges head-on by wishing to publish and communicate electronic text 
with phonetic symbols which were not included in basic ASCII.\footnote{One 
only needs to look at facsimiles of older electronic documents to see exotic 
symbols written in by hand.}

Long familiar were linguists already with the distinction between function 
and form. Even in the context of the computer implementation of writing systems, 
the necessity to distinguish form and function had been made \citep{Becker1984}. 
The computer industry, on the other hand, did not consider, ignored, or simply 
did not encode this principle when creating operating systems like MS-DOS, which 
were limited to 256 code points (due to computer hardware architecture) and 
encoded with one-to-one mappings from character codes to graphemes.

Industry was starting to tackle the issues involved in developing a single 
multilingual computing environment on a variety of fronts, including the then 
new technology of bitmap fonts and the creation of Font Manager and Script 
Manager by Apple \citep{Apple1985,Apple1986,Apple1988}. As noted above, around 
this time linguists were developing work-arounds such as SAMPA, so that they 
could communicate IPA transcription and use ASCII-based software. Some linguists 
formalized the issues of multilingual text processing from a computational 
perspective \citep{Anderson1984,Becker1984,Simons1989}. The study of writing 
systems was also being invigorated by the computational challenges in making 
computers work in a multilingual environment.\footnote{\cite[11--15]{Sampson1985} 
urges linguists to view the study of writing systems as legitimate scientific enquiry.}

% This standard became the basis for a proposal to include the IPA in the first version of the Unicode Standard. Decisions by the Computer Coding working group and work they continued after the 1989 Kiel Convention were adopted by the International Phonetic Association. These decisions are directly reflected in the Unicode Standard's encoding of IPA, seeing as it was the Association who submitted the script proposal to the Unicode Consortium.

The second major benefit of the standardization of the IPA in a computational 
representation by the Kiel working group is that it provided the basis for a 
formal proposal to be submitted to various international standards organizations, 
several of which were trying to tackle (and in a sense `win') the multilingual 
computing environment problem. Basically, everyone from corporations to governments 
to language scientists (for lack of a better term) wanted a single unified multilingual 
character encoding set for all the world's writing systems, even if they did not 
understand or appreciate the challenges involved in creating and adopting a solution. 
Additionally, advancements in computer hardware were making the solutions easier 
to implement in software.

The engineering problems and solutions had been spelled out years before, e.g.\ 
a two-byte encoding for multilingual text \citep{Anderson1984}. 
Although languages vary to an astounding extent (cf.\ \cite{EvansLevinson2009}), 
writing systems are quite similar formally and the issue of formal representation 
of the world's orthographic systems had also been addressed \citep{Simons1989}. 

A major obstacle in creating a single encoding multilingual environment from 
the perspective of writing systems involves the distinction between function 
and form \citep{Becker1984} This distinction is so central to basic linguistic 
theory and that trained linguists and semiologists take it as second nature. 
A central challenge in developing a universal character set was to combine a 
technological solution with a formalization of writing systems proper.\footnote{Of 
course there were additional practical issues to overcome, e.g.\ funding, creating 
the formal and technological proposal, deciding which characters and writing systems 
to include initially, while setting precedence of how to add new ones in the future.}

In hindsight it is easy to lose sight of how impactful 30 years of technological 
development have been on linguistics, from theory development using quantitative 
means to pure data collection and dissemination (which as fed back into the former). 
But at the end of the 1970s, virtually no ordinary working linguist was using a 
person computer \cite{Simons1996}. Personal computer usage, however, dramatically 
increased throughout the 1980s. By 1990, dozens of character sets were in common use. 
They varied in their architecture and in their character repertoires, which 
made things a mess. There were two major players in the universal character 
set race: Unicode and the International Organization for Standardization (ISO).

	
\section{Unicode and ISO 10646}

In the late 1980s, a universal character set was being developed by what 
is now referred to as the Unicode Consortium.\footnote{The Unicode 
Consortium was officially incorporated in January 1991.} This consortium 
consisted largely, although not entirely, of major US corporations, with 
the aim of overcoming the inoperability of different coded character sets 
and their costly hinderance for developing multilingual software development 
and for internationalization efforts. Commercial importance of course drove 
the early inclusion of Latin, non-Latin, and some exotic scripts; see the 
table of commercial importance as measured by GDP of countries using certain 
writing systems \citep[2]{unicode88}.

The original Unicode manifesto is \cite{unicode88.pdf}.\footnote{http://www.unicode.org/history/unicode88.pdf} 
Its aim was for a reliable international multilingual text encoding standard 
that would encompass all scripts of the world, or in the author's own words, 
``a new, world-wide ASCII''. An in-depth history of Unicode, highlighting 
interesting facts like its first text prototypes at Apple and its incorporation 
into TrueType, is retold online.\footnote{\url{http://www.unicode.org/history/earlyyears.html}}

Unicode 88 provided the basic principles for the Unicode Standard's design -- 
pushing for 16 bit representations of characters with a clear distinction 
between characters and glyphs. Some of the contents of this status proposal 
of 1988 were reworked for inclusion in the early Unicode Standard pre-publication 
drafts and by August 1990, the proposal was in a (very) rough draft format. Its 
editors and the Unicode Working Group (the predecessor to the Unicode Technical 
Committee) worked together to lay out the the proposed standard's structure and 
content. At this time, the proposal contained no code charts nor block descriptions. 

% http://www.unicode.org/history/earlyyears.html
% During this period of time, in addition to his co-authoring of Apple KanjiTalk, Davis was involved in further discussions of a universal character set which were being prompted by the development of the Apple File Exchange.

The other major player in developing a universal character set was the ISO 
working group from the International Standards Organization (ISO), based 
in Europe, which was responsible for ISO/IEC 10646. This character set 
standard was composed in 1989 and a draft was published in 1990.\footnote{\url{http://www.iso.org/iso/catalogue_detail.htm?csnumber=56921}} 
The `Universal Multiple-Octet Coded Character Set' or simply UCS was the first 
officially standardized character encoding with the aim of including all characters from all writing systems.\footnote{\url{http://www.nada.kth.se/i18n/ucs/unicode-iso10646-oview.html}}

ISO/IEC 10646 is partly based on ISO/IEC 8859, a series of of ASCII-based 
standard character encodings published in 1987 that use a single bit 8-byte 
character set. Each part of the standard, e.g.\ 8859-1, 8859-5, 8859-6, 
encodes characters to support different languages' writing systems, e.g.\ 
Latin-1 Western European, Latin/Cyrillic, Latin/Arabic, respectively. Being 
a joint effort by the International Organization for Standardization (ISO) 
and the International Electrotechnical Commission (IEC), the aim of the 
standard is reliable information exchange. So, again, issues of phonetic 
symbol encoding, typography, etc., were ignored -- or perhaps more properly 
put, not commercially driven at this early stage.

Intended for the major Western European languages, ISO/IEC 8859 was an 
extension of the ASCII character encoding standard, which included the 
English alphabet, numerals and computer control characters (e.g.\ beep, 
space, carriage return). By extending ASCII's 7-bit system to 8-bit, the 
character repertoire of each of ISO/IEC 8859 character set was doubled 
from 128 to 256 characters. Each character set defined a mapping between 
digital bit patterns and characters, which are visually rendered on screen 
as graphic symbols. ASCII was shared between ISO/IEC 8859 character sets, 
but the characters in the extra bit patterns were different. Thus an aim 
of the ISO working group responsible for ISO/IEC 10646 was to bring all 
characters in all writings systems into a single unified encoding.

In 1991, the Unicode Consortium and the ISO Working Group for ISO/IEC 10646 
decided to create a single universal standard for encoding multilingual text.\footnote{http://unicode.org/book/appC.pdf} 
The two character sets converged, resulting in mutually acceptable changes 
to both, and each group keeps versions of their respective character codes 
and encoding forms synchronized.\footnote{http://www.unicode.org/versions/} 
Although each standard has its own form of reference and the terminology in 
each may differ slightly, the practical difference is that the Unicode Standard 
is a formal implementation of ISO/IEC 10646 and imposes additional constraints 
on its implementation. The Unicode Standard includes character data, algorithms 
and specifications, outside the scope of ISO/IEC 10646, which ensure, when 
properly implemented in software applications and platforms, that characters 
are treated uniformly. 

The incorporation of the Unicode Standard into the international encoding 
standard ISO 10646 was approved by ISO as an International Standard in June 
1992.\footnote{http://www.unicode.org/versions/Unicode1.0.0/Notice.pdf} 
The joint Unicode and ISO/IEC 10646 standard has become \textit{the} universal 
character set and it is a single multilingual environment for the majority 
of the world's written languages. Its formal implementation has also been 
vital to the rise of a multi-lingual Internet.


\section{IPA and Unicode}

It was a long journey, but the goal of achieving a single multilingual 
computing environment has largely been accomplished. We users, however, 
must cope with the pitfalls that were dug along the way (see Pitfall sections below). 
Some linguists, including your humble authors, are particularly sensitive 
to these issues. So we provide practical advice and approaches in the rest 
of this chapter. But first, we explain how the IPA became incorporated 
into the Unicode Standard via ISO/IEC 10646.

% "The set of IPA symbols and their numbers were used to draw up an entity set within SGML by the Text Encoding Initiative (TEl). The name of each entity is formed by 'IPA' preceding the number, e.g. IPA304 is the rEIentity name of lower-case A. These symbols can be processed as IPA symbols and represented on paper and screen with the appropriate local font by modifying the :entity replacement text. The advantage of the SGML entity set is that it is independent or the character set being used."
% "A TEl writing system declaration (wsd) has been drawn up for the IPA symbols."
% A TEl writing system declaration (wsd) has been drawn up for the IPA symbols. This document gives information about the symbol and its IPA function, as well as its encoding in the accompanying SGML document and in UnicodelUCS and in AFII. The writing system declaration can be read as a text d9cument or processed by machines in an SMGL process.

After the Kiel Convention in 1989, the Computer Coding of the IPA working 
group assisted the International Phonetic Association in representing the 
IPA to ISO and to the Text Encoding Initiative (TEI) \citep{EslingGaylord1993}. 
The working group's formalization of the IPA, i.e.\ a full listing of agreed 
upon computer codings for phonetic symbols, was used in developing writing 
systems descriptions which were at the time being solicited for scripts to 
be included in initiatives for new multilingual international character 
encoding standards. The working group for ISO/IEC 10646 and Unicode were 
two such initiatives.

In the historical context of the IPA being considered for inclusion in 
ISO/IEC 10646, it is important to realize that there were a variety of 
sources (i.e.\ not just from the Association) which submitted character 
proposals for phonetic alphabets. These proposals, including from the 
Association via the Kiel working group, were considered as a whole by 
the ISO working groups which were responsible for incorporating a phonetic 
script into the universal character set (UCS). The ISO working groups that 
were responsible for assigning a phonetic character set then made their 
own submissions as part of a review process by ISO for approval based on 
both ``informatic'' and phonetic criteria \citep[86]{EslingGaylord1993}. 

Character set ISO/IEC 10646 was approved by ISO, including the phonetic 
characters submitted to them in May 1993. The set of IPA characters were 
assigned UCS codes in 16 bit representation (in hexadecimal) and were 
published Tables 2 and 3 in \cite{EslingGaylord1993}, which include a 
graphical representation of the IPA symbol, its IPA Name, phonetic description, 
IPA Number, UCS Code and AFII Code.\footnote{The Association for Font 
Information Interchange (AFII) was an international database of glyphs 
created to promote the standardization of font data required to produce 
ISO/IEC 10646.} Because the character sets of ISO/IEC 10646 and the 
Unicode Standard converged, the IPA as submitted by the Association 
and reviewed and further submitted by the ISO working group, was included 
in the Unicode Standard Version 1.0 -- largely as we know it today.\footnote{The 
Association later made the foresightful remark, ``When this character set 
is in wide use, it will be the normal way to encode IPA symbols.'' \citep[164]{IPA1999}.}

With subsequent revisions to the IPA, one might expect the Unicode 
Consortium would update the Unicode Standard in a way that is inline 
with linguists' or other language scientists' intuitions. However, 
updates that go against the ISO's and the Unicode Standard's principles 
of maintaining backwards compatibility lose out, i.e.\ it is more important 
to deal with the pitfalls created along the way then it is to change the 
standard. Therefore, many of the pitfalls we encounter when using Unicode 
IPA are historic relics that we have to come to grips with.

% https://en.wikipedia.org/wiki/Uralic_Phonetic_Alphabet
% http://www.unicode.org/conference/bulldog.html
% http://www-01.sil.org/computing/computing_environment.html

\chapter{IPA meets Unicode}
\label{ipa-meets-unicode}

\section{The International Phonetic Alphabet (IPA)}
\label{the-international-phonetic-alphabet}

The International Phonetic Alphabet (IPA) is a common standard in linguistics to
transcribe sounds of spoken language into some Latin-based characters
\citep{IPA2005}. Although IPA is reasonably easily adhered to with pen and
paper, it is not trivial to encode IPA characters electronically. Early work
addressing the need for a universal computing environment for writing systems
and their computational complexity is discussed in Simons 1989. For a long time,
linguists (like all other computer users) were limited to ASCII-encoded 7-bit
characters, which only includes Latin characters, numbers and some punctuation
and symbols. Restricted to these standard character sets that lacked IPA support
or other language-specific graphemes that they needed, linguists devised their
own solutions \citep{BirdSimons2003}. For example, some chose to represent
unavailable graphemes with substitutes, e.g.~the combination of to represent .
Tech-savvy linguists redefined selected characters from a character encoding by
mapping custom made fonts to those code points. However, one linguist's
electronic text would not render properly on another linguist's computer without
access to the same font. Further, if two character encodings defined two
character sets differently, then data could not be reliably and correctly
displayed. This is a common example of the non-interoperability of data and data
formats.

To alleviate this problem, during the late 1980s, SAMPA (Speech Assessment
Methods Phonetic Alphabet) was created to represent IPA symbols with 7-bit
printable ASCII character sequences, e.g. <p\textbackslash> for [ɸ]. Two
problems with SAMPA are that (i) it is only a partial encoding of the IPA and
(ii) it encodes different languages in separate data tables, instead of a
universal alphabet, like IPA. SAMPA tables are derived from phonemes appearing
in several European languages that were developed as part of a European
Commission-funded project to address technical problems like electronic mail
exchange (what is now simply called email). SAMPA is essentially a hack to work
around displaying IPA characters, but it provided speech technology and other
fields a basis that has been widely adopted and used in code. So, SAMPA was a
collection of tables to be compared, instead of a large universal table
representing all languages. An extended version of SAMPA, called X-SAMPA, set
out to include every symbol in the IPA chart including all diacritics
\citep{WellsND}. X-SAMPA was considered more universally applicable because it
consisted of one table that encoded the set of characters that represented
phones/segments in IPA across languages. SAMPA and X-SAMPA have been widely used
for speech technology and as an encoding system in computational linguistics.
Eventually, ASCII-encoding of the IPA became depreciated through the advent of
the Unicode Standard. Note however that many popular software packages used for
linguistic analyses still require ASCII input,
e.g.~RuG/L04\footnote{\url{http://www.let.rug.nl/kleiweg/L04/}} and
SplitsTree4.\footnote{\url{http://www.splitstree.org/}}

Still, there are a few pitfalls to be aware of when using the Unicode Standard
to encode IPA. As we have said before, from a linguistic perspective it might
look like the Unicode Consortium is making incomprehensible decisions, but it is
important to realize that the consortium has tried and is continuing to try to
be as consistent as possible across a wide range of use cases, and it does place
linguistic traditions above other orthographic possibilities. In general, we
strongly suggest linguists not to complain about any decisions in the Unicode
Standard, but to try and understand the rationale of the Unicode Consortium
(which in our experience is almost always well-conceived) and devise ways to
work with any unexpected behavior. Many of the current problems derive from the
fact that the IPA is clearly historically based on the Latin script, but
different enough from most other Latin-based writing systems to warrant special
attention. This ambivalent status of the IPA glyphs (partly Latin, partly
special) is unfortunately also attested in the treatment of IPA in the Unicode
Standard. In retrospect, it might have been better to consider the IPA (and
other transcription systems) to be a special or new kind of script within the
Unicode Standard, and treat the obvious similarity to Latin glyphs as a
historical relic. All IPA glyphs would then have their own code points, instead
of the current situation in which some IPA glyphs have special code points,
while others are treated as being identical to the regular Latin characters.
Yet, the current situation, however unfortunate, is unlikely to change, so as
linguists we will have to learn to deal with the specific pitfalls of IPA within
the Unicode Standard. In this section, we will describe these pitfalls in some
detail.

\section{Pitfall: No complete IPA code block}
\label{pitfall-no-complete-ipa-block}

The ambivalent nature of IPA glyphs arises because, on the one hand, the IPA
uses Latin-based glyphs like <a>, <b> or <p>. From this perspective, the IPA
seems to be just another orthographic tradition using Latin characters, all of
which do not get a special treatment within the Unicode Standard (just like
e.g.~the French, German, or Danish orthographic traditions do not have a special
status). On the other hand, the IPA uses many special symbols (like turned,
mirrored and/or extended Latin glyphs) not found in any other Latin-based
writing system. For this reason, and already in the first version of the Unicode
Standard (Version 1.0 from 1991), a special block with code points, called
\textsc{IPA Extensions} was included. 

As explained in Section~\ref{pitfall-blocks}, the Unicode Standard code space is
subdivided into character blocks, which generally encode characters from a
single script. However, as is illustrated by the IPA, characters that form a
single writing system may be dispersed across several different character
blocks. With its diverse collection of symbols from various scripts and
diacritics, the IPA is spread across 13 blocks in the Unicode
Standard:\footnote{This number of blocks depends on whether only IPA-sanctioned
symbols are counted or if the phonetic symbols commonly found in the literature
are also included, see \cite[Appendix C]{Moran2012}.}

\begin{itemize}
	\item Basic Latin (30 characters), e.g. <a b c d e> 
	\item Latin-1 Supplement (4 characters): <æ ç ð ø> 
	\item Latin Extended-A (3 characters): <ħ ŋ œ> 
	\item Latin Extended-B (5 characters): <ǀ ǁ ǂ ǃ ȵ> 
	\item IPA Extensions (70 characters), e.g. <ɐ ɑ ɔ> 
	\item Spacing Modifier Letters (20 characters), e.g. <ʰ ʷ ˥> 
	\item Combining Diacritical Marks (33 characters), e.g. <{\large \ \ ̝\ \ ̥\ \ ̪ }> 
	\item Greek and Coptic (3 characters): <β θ χ> 
	\item Phonetic Extensions (2 characters): <{\small \fontspec{CharisSIL}ᴅ ᴴ}> 
	\item Phonetic Extensions Supplement (3 characters): <{\small \fontspec{CharisSIL}ᶑ ᶾ ᶣ}> 
	\item Superscripts and Subscripts (1 character): <ⁿ> 
	\item Arrows (4 characters): <↑ ↓ ↗ ↘>
	\item Latin Extended-C (1 character): <{\small \fontspec{CharisSIL}ⱱ}> 
\end{itemize}

\section{Pitfall: IPA homoglyphs in Unicode}
\label{pitfall-ipa-homoglyphs}

Another problem is the large number of homoglyphs, i.e.~different characters
that have highly similar glyphs (or even completely identical, depending on the
font rendering). For example, a speaker of Russian should ideally not use the
<а> \textsc{cyrillic small letter a} at code point \uni{0430} for IPA
transcriptions, but instead use the <a> \textsc{latin small letter a} at code point
\uni{0061}, although visually they are mostly indistinguishable, and the
Cyrillic character is more easily typed on a Cyrillic keyboard. 

Furthermore, even linguists are unlikely to distinguish between
the <ə> \textsc{latin small letter schwa} at code point \uni{0259} and <ǝ>
\textsc{latin small letter turned e} at \uni{01DD}. Conversely,
non-linguists are unlikely to distinguish any semantic difference between an
open back unrounded vowel <ɑ> \textsc{latin small letter alpha} at
\uni{0251}, and the open front unrounded vowel <a> \textsc{latin small letter
a} at \uni{0061}. But even among linguists this distinction leads to problems.
For example, as pointed out by \citet{Mielke2009}, there is a problem stemming
from the fact that about 75\% of languages are reported to have a five-vowel
system \citep{Maddieson1984}. Historically, linguistic descriptions tend not to
include precise audio recording and measurements of formants, so this may lead
one to ask if the many characters that are used in phonological description
reflects a transcriptional bias. The common use of <a> in transcriptions
could be in part due to the ease of typing the letter on an English keyboard (or
for older descriptions, the typewriter). We found it to be exceedingly rare that
a linguist uses <ɑ> for a low back unrounded vowel.\footnote{One example is
\citet[75]{Vidal2001a}, in which the author states: ``The definition of Pilagá
/a/ as {[}+back{]} results from its behavior in certain phonological contexts.
For instance, uvular and pharyngeal consonants only occur around /a/ and /o/.
Hence, the characterization of /a/ and /o/ as a natural class of (i.e.,
{[}+back{]} vowels), as opposed to /i/ and /e/.''} They simply use <a> as
long as there is no opposition to <ɑ>.\footnote{See Thomason's Language Log
post, ``Why I don't love the International Phonetic Alphabet'' at:
\url{http://itre.cis.upenn.edu/~myl/languagelog/archives/005287.html}.}

Making things even more problematic, there is an old typographic tradition that
the double-story <a> uses a single-story <ɑ> in italics. This leads to the
unfortunate effect that in most well-designed fonts the italics of <a> and <ɑ>
use the same glyph. If this distinction has to be kept upright in italics, the
only solution we can currently offer is to use \textsc{slanted} glyphs
(i.e.~artificially italicized glyphs) instead of real italics (i.e.~special
italics glyphs designed by a typographer).\footnote{For example, the widely used
IPA font Doulos SIL
(\url{http://scripts.sil.org/cms/scripts/page.php?item\_id=DoulosSIL}) does not
have real italics. This leads some word-processing software, like Microsoft
Word, to produce slanted glyphs instead. That particular combination of font and
software application will thus lead to the desired effect distinguishing <a>
from <ɑ>. However, note that when the text is transferred to another font
(i.e.~one that includes real italics) and/or to another software application
(like Apple Pages, which does not perform slanting), then this visual appearance
will be lost. In this case we are thus still in the pre-Unicode situation in
which the choice of font and rendering software actually matters. The ideal
solution from a linguistic point of view would be the introduction of a new IPA
code point for a different kind of which explicitly specifies that it should
still be rendered as a double-story character when italicized. After informal
discussion with various Unicode players, our impression is that this highly
restricted problem is not sufficiently urgent to introduce even more <a>-like
characters in Unicode (which already lead to much confusion, see Section~
\ref{pitfall-homoglyphs}).}

Some other homoglyphs related to encoding IPA in the Unicode Standard are:

\begin{itemize}
	\item The uses of the apostrophe has led to long discussions on the Unicode
       Standard email list. An English keyboard inputs <{\fontspec{Monaco}'}> \textsc{apostrophe}
       at \uni{0027}, although the preferred Unicode apostrophe is the <\ ' >
       \textsc{right single quotation mark} at \uni{2019}.
     \item The glottal stop/glottalization/ejective marker is another completely
        different character <{\large ʼ}>, the \textsc{modifier letter apostrophe} at
        \uni{02BC}, but unfortunately looks mostly highly similar to
        \uni{2019}. 
	\item Another problem is the <ˁ> \textsc{modifier letter reversed
       glottal stop} at \uni{02C1} vs.\@ the <ˤ> \textsc{modifier
       letter small reversed glottal stop} at \uni{02E4}. Both are denoted in
       the Unicode Standard as the \textsc{pharyngealized diacritic} and both
       appear in various resources representing phonetic data online. This is
       thus an example for which the Unicode Standard does not solve the
       linguistic standardization problem. 
	\item There is at least one case in which the character name assigned by the
       Unicode Consortium does not match the IPA's description. In the Unicode
       Standard the <ǃ> at \uni{01C3} is labeled \textsc{latin letter retroflex
       click}, but in IPA is an alveolar or postalveolar click (not retroflex).
       This naming is probably best seen as a simple error in the Unicode
       Standard. This character is of course often simply typed as <!>
       \textsc{exclamation mark} at \uni{0021}.
\end{itemize}       

\section{Pitfall: Homoglyphs in IPA}

It is not just the Unicode Standard that offers multiple options for encoding 
the IPA. Even the IPA itself is not consistent in how specific meaningful 
transcriptions have to be encoded. There are a few cases in which the IPA 
explicitly allows for different options of transcribing the same phonetic 
content

Another example
we commonly encounter is the use of <g> \textsc{latin small letter g} at \uni{0067},
instead of the Unicode Standard IPA character for the voiced velar
stop <ɡ> \textsc{latin small letter script g} at \uni{0261}. One begins to question
whether this issue is at all apparent to the working linguist, or if they simply
use the \uni{0067} because it is easily keyboarded and thus saves time, whereas
the latter must be cumbersomely inserted as a special symbol in most
software.


\begin{itemize}
  \item tone marks (prefer tone symbols)
  \item marking of <g>\footnote{This issue was recently addressed by the International
Phonetic Association, which has taken the stance that both the keyboard
\textsc{latin small letter g} and the \textsc{latin small letter script g} are
valid input characters for the voiced velar plosive. Unfortunately, this
decision further introduces ambiguity for linguists trying to adhere to a strict
Unicode Standard IPA encoding.}
  \item marking of voiceless with ring diacritic
  \item tie bar
\end{itemize}

\section{Pitfall: Ligatures and digraphs}
\label{pitfall-ligatures-digraphs}     
       
One important distinction to acknowledge is the difference between multigraphs
and ligatures. Multigraphs are groups of characters (in the context of IPA e.g.
< tʃ > or < ou >) while ligatures are single characters (e.g. <ʧ> \textsc{latin
small letter tesh digraph} at \uni{02A7}). Ligatures arose in the context of
printing easier-to-read texts, and are included in the Unicode Standard for
reasons of legacy encoding. However, their usage is discouraged by the Unicode
core specification. Specifically related to IPA, various phonetic combinations
of characters (typically affricates) are available as single code-points in the
Unicode Standard, but are designated as \textsc{ligatures} or \textsc{digraphs}
(confusingly both names appear interchangeably). Such glyphs might be used by
software to produce a pleasing display, but they should not be hard-coded into
the text itself. In the context of IPA, characters like the following ligatures
should thus \emph{not} be used. Instead a combination of two characters is
preferred:
      
\begin{itemize} 
	\item <ʣ> \textsc{latin small letter dz digraph} at \uni{02A3} (use <d> + <z> instead) 
	\item <ʧ> \textsc{latin small letter tesh digraph} at \uni{02A7} (use <t> + <ʃ> instead) 
	\item <ʩ> \textsc{latin small letter feng digraph} at \uni{02A9} (use <f> + <ŋ> instead) 
\end{itemize}

However, there are a few Unicode characters that are historically ligatures, but
which are today considered as simple characters in the Unicode Standard and thus
should be used when writing IPA, namely:

\begin{itemize}
	\item <ɮ> \textsc{latin small letter lezh} at \uni{026E} 
	\item <œ> \textsc{latin small ligature oe} at \uni{0153} 
	\item <ɶ> \textsc{latin letter small capital oe} at \uni{0276} 
	\item <æ> \textsc{latin small letter ae} at \uni{00E6} 
\end{itemize}

\section{Pitfall: Missing decomposition}
\label{pitfall-missing-decomposition}

Although many combinations of base character with diacritic are treated as 
canonical equivalent with precomposed characters, there are a few combinations 
in IPA that allow for multiple, apparently identical, encodings that are not 
canonical equivalent. The following elements should not be treated as diacritics 
when encoding IPA in Unicode:
\begin{itemize}
  \item <\ {\large  ̡}\ > \textsc{combining palatalized hook below} at \uni{0321}
  \item <\ \ {\large  ̢}> \textsc{combining retroflex hook below} at \uni{0322}
  \item <\ \ {\large  ̵}> \textsc{combining short stroke overlay} at \uni{0335}
  \item <\ \ {\large  ̷}> \textsc{combining short solidus overlay} at \uni{0337}
  \item <\ \ {\large  ̴}> \textsc{combining tilde overlay} at \uni{0334}
  \item <{\large ˞}> \textsc{modifier letter rhotic hook} at \uni{02DE}
\end{itemize} 

There turn out to be a lot of characters in the IPA that could be conceived as 
using any of these elements, like <ɲ>, <ɳ>, <ɨ>, <ø>, <ɫ> or <ɚ>. However, all 
such characters exist as well as ``precomposed'' combination in Unicode, and these 
precomposed characters should preferably be used. When combinations of a base 
character with diacritic are used, then these combinations are not canonical 
equivalent to the precomposed combinations. This means that any search will not 
find both at the same time.

Reversely, <ç> \textsc{latin small letter c with cedilla} at \uni{00E7} will be
decomposed into <c> \textsc{latin small letter c} at \uni{0063} with 
<\ \ {\large  ̧}> \textsc{combining cedilla} at \uni{0327}, 
also if such a decompotion is not intended, because it is meaningless in IPA.

\section{Pitfall: Different notions of diacritics}
\label{pitfall-the-ipa-notion-of-diacritics-is-not-the-same-as-the-unicode-standards-notion-of-diacritics}

Another pitfall is diacritics. The problem is that the meaning of the term
\textsc{diacritics} as used by the IPA is not the same as it used in the Unicode
Standard. Specifically, diacritics in the IPA-sense are either so-called
\textsc{Combining Diacritical Marks} or \textsc{Spacing Modifier Letters} in the
Unicode Standard. Crucially, Combining Diacritical Marks are by definition
combined with the character before them (to form so-called default grapheme
clusters, see Section 3). In contrast, Spacing Modifier Letters are by
definition \emph{not} combined into grapheme clusters with the preceding
character, but simply treated as separate letters. In the context of the IPA,
the following IPA-diacritics are actually Spacing Modifier Letters in the
Unicode Standard:

\begin{itemize}
	\item Length marks, namely <ː> \textsc{modifier letter triangular colon} at \uni{02D0} and <ˑ> \textsc{modifier letter half triangular colon} at \uni{02D1}. 
	\item Tone letters, like <˥> \textsc{modifier letter extra-high tone bar} at \uni{02E5}, and others like this. 
	\item Superscript letters, like <ʰ> \textsc{modifier letter small h} at \uni{02B0} or <ˤ> \textsc{modifier letter small reversed glottal stop} at \uni{02E4}, and others like this. 
	\item <˞> \textsc{modifier letter rhotic hook} at \uni{02DE}. 
\end{itemize}

Although linguists might expect these characters to belong together with the
character in front of them, at least for <ʰ> \textsc{modifier letter small h} at
\uni{02B0} the Unicode Consortium's decision to treat it as a separate character
is also linguistically correct, because according to the IPA it can be used both
for aspiration (more precisely post-aspiration following the base character) and
pre-aspiration (preceding the base character). Note that there is a mechanism in
Unicode to force separate characters to be combined (namely by using the
\textsc{zero width joiner} at \uni{200D}), but this seems to be a rather
impractical, and probably not enforceable solution to us.

\section{Pitfall: No unique diacritic ordering}
\label{pitfall-no-unique-diacritic-ordering}

Also related to diacritics is the question of ordering. To our knowledge, the
International Phonetic Association does not specify a specific ordering for
diacritics that combine with phonetic base symbols; this exercise is left to the
reasoning of the transcriber. However, such marks have to be explicitly ordered
if sequences of them are to be interoperable and compatible. An example is a
labialized aspirated alveolar plosive: <tʷʰ>. There is nothing holding linguists
back from using <tʰʷ> instead (with exactly the same intended meaning). However,
from a technical standpoint, these two sequences are different, e.g.~if both
sequences are used in a document, searching for <tʷʰ> will not find any
instances of <tʰʷ>, and vice versa. Likewise, a creaky voiced syllabic dental
nasal can be encoded in various orders, e.g. <n̪̰̩>, <n̩̰̪> or <n̩̪̰>.

In accordance with the absence of any specification of ordering in the IPA, the
Unicode Standard likewise does not propose any standard orderings. Both leave it
to the user to be consistent. However, there is one aspect of ordering for which
the Unicode Standard does present a canonical solution, which is uncontroversial
from a linguistic perspective. Diacritics in the Unicode Standard
(i.e.~Combining Diacritical Marks, see above) are classified in Canonical
Combining Classes. In practice, the diacritics are distinguished by their
position relative to the base character.\footnote{See
\url{http://unicode.org/reports/tr44/\#Canonical\_Combining\_Class\_Values for a
detailed description}.} When applying a Unicode normalization (NFC or NFD, see
previous section), the diacritics in different positions are put in a specified
order. This process therefore harmonizes the difference between different
encodings, for example, of a midtone creaky voice vowel <ḛ̄>. This grapheme
cluster can be encoded either as <e> + <̄> + <̰> or as <e> + <̰> + <̄> . To
prevent this twofold encoding, the Unicode Standard specifies the second
ordering as canonical (diacritics below before diacritics above).

When encoding a string according to the Unicode Standard, it is possible to do
this either using the NFC (composition) or NFD (decomposition) normalization.
Decomposition implies that precomposed characters (like <á> \textsc{latin small
letter a with acute} at \uni{00E1}) will be split into its parts. This might
sound preferable for a linguistic analysis, as the different diacritics are
separated from the base characters. However, note that most attached elements
like strokes (e.g.~in the <ɨ>), retroflex hooks (e.g.~in <ʐ>) or rhotic hooks
(e.g.~in <ɝ>) will not be decomposed, but strangely enough a cedilla (like in
<ç>) will be decomposed. In general, Unicode decomposition does not behave like
a feature decomposition as expected from a linguistic perspective. It is thus
important to consider Unicode decomposition only as a technical procedure, and
not assume that it is linguistically sensible.

Facing the problem of specifying a consistent ordering of diacritics while
developing a large database of phonological inventories from the world's
languages, \citet[540]{Moran2012} defines a set of diacritic ordering
conventions.\footnote{The most recent version of these conventions is online:
\url{http://phoible.github.io/conventions/}} The conventions are influenced by
the linguistic literature, though some ad-hoc decisions had to be taken. The
goal was to explicitly define all character sequences so that the vast variety
of phonemes found in descriptions of the world's language were normalized into
consistent character sequences, e.g.~if one language description uses and
another , when both are intended to be phonetically equivalent, then a decision
to normalize to one form was taken. For example, when a character sequence
contains more than one character in Spacing Modifier Letters, the order that is
proposed is the following (where <\textbar{}> indicates \textit{or} and <→>
indicates \textit{precedes}):

\begin{itemize}
	\itemsep1pt\parskip0pt\parsep0pt 
	\item \textsc{Spacing Modifier Letters ordering:} ( unreleased <̚> \textbar{} lateral release <ˡ> \textbar{} nasal release <ⁿ>) → ( palatalized <ʲ>) → ( labialized <ʷ>) → ( velarized <ˠ>) → ( pharyngealized <ˤ>) → ( aspirated <ʰ> \textbar{} ejective <ʼ>) → ( long <ː> \textbar{} half-long <ˑ>) 
\end{itemize}

If a character sequence contains more than one diacritic below the base
character, then the place feature is applied first (dental, laminal, apical,
fronted, backed, lowered, raised), then the laryngeal setting (voiced,
voiceless, creaky voice, breathy voice) and finally the syllabic or non-syllabic
marker (for vowels, ATR gets put on between the place and laryngeal setting).
So:

\begin{itemize}
	\itemsep1pt\parskip0pt\parsep0pt 
	\item \textsc{Combining Diacritical Marks (below) ordering:} ( dental <t̪> \textbar{} laminal <t̻> \textbar{} apical <t̺>) → ( fronted <u̟> \textbar{} backed <e̠>) →( lowered <e̞> \textbar{} raised <e̝>) → ( ATR <e̘ e̙>) → ( voiced <s̬> \textbar{} voiceless <n̥> \textbar{} creaky voice <b̰> \textbar{} breathy voice <b̤>) → ( syllabic <n̩> \textbar{} non-syllabic <e̯>) 
\end{itemize}

Character sequences with diacritics above the base character were not
problematic in \citet{Moran2012} because they include only the centralized,
mid-centralized and nasalized combining characters. \citet{Moran2012} marks
tones as singletons with Space Modifier Letters, e.g. \textless{}˦\textgreater{}
for a phonemic high tone, instead of accent diacritics, alleviating potential
conflicts. Building on the work of \citet{Moran2012}, if a character sequence
contains more than one diacritic above the base character, we propose:

\begin{itemize}
	\itemsep1pt\parskip0pt\parsep0pt 
	\item \textsc{Combining Diacritical Marks (above) ordering:} (centralized <ë> \textbar{} mid-centralized <e̽>) → (extra short <ĕ>) →( tone accents, e.g. <è> ) → ( Spacing Modifier Letters ) → ( tone letters, e.g. <e˦>) 
\end{itemize}
\chapter{Orthography profiles}
\label{orthography-profiles}

\section{Characterizing writing systems}
\label{characterizing-writing-systems}

At this point in the course of rapid ongoing developments, we are left with a
situation in which the Unicode Standard offers a highly detailed and flexible
approach to deal computationally with writing systems, but it has unfortunately
not influenced the linguistic practice very much. In many practical situations,
the Unicode Standard is far too complex for the day-to-day practice in
linguistics because it does not offer practical solutions for the down-to-earth
problems of many linguists. In this section, we propose some simple practical
guidelines and methods to improve on this situation.

Our central aims for linguistics, to be approached with a Unicode-based
solution, are: (i) to improve the consistency of the encoding of sources, (ii)
to transparently document knowledge about the writing system (including
transliteration), and (iii) to do all of that in a way that is easy and quick to
manage for many different sources with many different writing systems. The
central concept in our proposal is the \textsc{orthography profile}, a simple
tab-separated CSV text file, that characterizes and documents a writing system.
We also offer basic implementations in Python and R to assist with the
production of such files, and to apply orthography profiles for consistency
testing, grapheme tokenization and transliteration. Not only can orthography
profiles be helpful in the daily practice of linguistics, they also succinctly
document the orthographic details of a specific source, and, as such, might
fruitfully be published alongside sources (e.g.~in digital archives). Also, in
high-level linguistic analyzes in which the graphemic detail is of central
importance (e.g.~phonotactic or comparative-historical studies), orthography
profiles can transparently document the decisions that have been taken in the
interpretation of the orthography in the sources used.

Given these goals, Unicode locale descriptions (see Section~\ref{terminology})
might seem like the ideal orthography profiles. However, there are various
practical obstacles preventing the use of such locale descriptions in the daily
linguistic practice, namely: (i) the XML-structure is too verbose to easily and
quickly produce or correct manually, (ii) locale descriptions are designed for a
wide scope on information (like date formats or names of weekdays) most of which
is not applicable for documenting writing systems, and (iii) most crucially,
even if someone made the effort to produce a technically correct locale
description for a specific source at hand, then it is nigh impossible to deploy
the description. This is because a locale description has to be submitted to and
accepted by the Unicode Common Locale Data Repository. The repository is
(rightly so) not interested in descriptions that only apply to a limited set of
sources (e.g.~only a single specific dictionary).

The major challenge then is developing an infrastructure to identify the
elements that are individual graphemes in a source, specifically for the
enormous variety of sources using some kind of alphabetic writing system.
Authors of source documents (e.g.~dictionaries, wordlists, corpora) use a
variety of writing systems that range from their own idiosyncratic
transcriptions to already well-established practical or longstanding
orthographies. Although the IPA is one practical choice as a sound-based
normalization for writing systems (which can act as an interlingual pivot to
attain interoperability across writing systems), graphemes in each writing
system must also be identified and standardized if interoperability across
different sources is to be achieved. In most cases, this amounts to more than
simply mapping a grapheme to an IPA segment because graphemes must first be
identified in context (e.g.~is the sequence one sound or two sounds or both?)
and strings must be tokenized, which may include taking orthographic rules into
account (e.g.~between vowels is /n/ and after a vowel but before a consonant is
a nasalized vowel /ṽ/). In our experience, data from each source must be
individually tokenized into graphemes so that its orthographic structure is
identified and its contents can be extracted. To extract data for analysis, a
source-by-source approach is required before an orthography profile can be
created. For example, almost each available lexicon on the world's languages is
idiosyncratic in its orthography and thus requires lexicon-specific approaches
to identify graphemes in the writing system and to map graphemes to phonemes, if
desired.

Thus, our key proposal for the characterization of a writing system is to use a
grapheme tokenization as an inter-orthographic pivot. Basically, any source
document is tokenized by graphemes, and only then a mapping to IPA (or any other
orthographic conversion) is performed. An orthography profile then is a
description of the units and rules that are needed to adequately model a
graphemic tokenization for a language variety as described in a particular
source document. An orthography profile summarizes the Unicode (tailored)
graphemes and orthographic rules used to write a language (the details of the
structure and assumptions of such a profile will be presented in the next
section).

As an example of graphemic tokenization, note the three different levels of
technological and linguistic elements that interact in the hypothetical lexical
form <tsʰṍ̰shi>:

\begin{enumerate}
	\def\labelenumi{\arabic{enumi}.} 
	\item code points (10 text elements): t s ʰ o ̃ ̰ ´ s h i 
	\item grapheme clusters (7 text elements): t s ʰ ṍ̰ s h i 
	\item tailored grapheme clusters (4 text elements): tsʰ ṍ̰ sh i 
\end{enumerate}

In (1), the string <tsʰṍ̰shi> has been tokenized into ten Unicode code points
(using NFD normalization), delimited here by space. Unicode tokenization is
required because sequences of code points can differ in their visual and logical
orders. For example, <õ̰> is ambiguous to whether it is the sequence of + <̰> +
<̃> or + <̃> + <̰>. Although these two variants are visually homoglyphs,
computationally they are different. Unicode normalization should be applied to
this string to reorder the code points into a canonical order, allowing the data
to be treated canonically equivalently for search and comparison. In (2), the
Unicode code points have been logically normalized and visually organized into
grapheme clusters, as specified by the Unicode Standard. The combining character
sequence <õ̰> is normalized and visually grouped together. Note that, the
MODIFIER LETTER SMALL H at \uni{02B0}, is not grouped with. This is because it
belongs to Spacing Modifier Letters category in the Unicode Standard. These
characters are underspecified for the direction in which they modify a host
character. For example, can indicate either pre- or post-aspiration (whereas the
nasalization or creaky diacritic is defined in the Unicode Standard to apply to
a specified base character). Finally, to arrive at the graphemic tokenization in
(3), tailored grapheme clusters are needed (as for example specified in an
orthography profile). For example, this orthography profile would specify that
the sequence of characters, and form a single grapheme, and that and form a
grapheme. The orthography profile could also specify orthographic rules,
e.g.~when tokenization graphemes, in say English, the in the forms and should be
treated as distinct sequences depending on their contexts.

\section{Informal description}
\label{informal-description-of-orthography-profiles}

An orthography profile describes the Unicode code points, characters, graphemes
and orthographic rules in a writing system. An orthography profile is a
language-specific (and often even resource-specific) description of the units
and rules that are needed to adequately model a writing system. An important
assumption of our work is that we assume a resource is encoded in Unicode (or
has been converted to Unicode). Any data source that the Unicode Standard is
unable to capture, will also not be captured by an orthography profile.

Informally, an orthography profile specifies the graphemes (or, in Unicode
parlance, \textsc{tailored grapheme clusters}) that are expected to occur in any
data to be analyzed or checked for consistency. These graphemes are first
identified throughout the whole data (a step which we call
\textsc{tokenization}), and possibly simply returned as such, possibly including
error messages about any parts of the data that are not specified by the
orthography profile. Once the graphemes are identified, they might also be
changed into other graphemes (a step which we call \textsc{transliteration}).
When a grapheme has different possible transliterations, then these differences
should be separated by contextual specification, possibly down to listing
individual exceptional cases.

In practice, we foresee a workflow in which orthography profiles are iteratively
refined, while at the same time inconsistencies and errors in the data to be
tokenized are corrected. In some more complex use-cases there might even be a
need for multiple different orthography profiles to be applied in sequence (see
Section~\ref{use-cases} on various exemplary use-cases). The result of any such
workflow will normally be a cleaned dataset and an explicit description of the
orthographic structure in the form of an orthography profile. Subsequently, the
orthography profiles can be easily distributed in scholarly channels alongside
the cleaned data, for example in supplementary material added to journal papers
or in electronic archives.

\section{Formal specification}
\label{formal-specification-of-orthography-profiles}

The formal specifications of an orthography profile (or simply \textsc{profile}
for short) are the following:

\begin{enumerate}
	\def\labelenumi{\arabic{enumi}.} 
	\item \textsc{A profile is a} \textsc{Unicode UTF-8 encoded text file} (ideally using NFC, no-BOM, and LF; see Section~\ref{pitfall-file-formats}, Pitfall: File Formats) that includes the information pertinent to the orthography. 
	\item \textsc{A profile is a} \textsc{tab-separated CSV file with an obligatory header line}. A minimal profile can have just a single column, in which case there will of course no tabs, but the first line will still be the header. For all columns we assume the name in the header of the CSV file to be crucial. The actual ordering of the columns is unimportant. 
	\item \textsc{Lines starting with a hash \textless{}\#\textgreater{} are ignored.} Comments and metadata can be included inside the file, but only as complete lines in the profile, to be marked by lines starting with hash \textsc{\#} (\textsc{number sign} at \uni{0023}). Hashes somewhere else in the file are to be treated literally, i.e.~hashes are only to be ignored when they occur at the start of a line.\footnote{Comments that belong to specific lines will have to be put in a separate column of the CSV file, e.g.~add a column called \textsc{comments}. Further, if the content of a profile contains a hash at the start of a line, either reorder the columns so the hash does not occur at the start of the line, or add a dummy column in front of the data to not have the data start with a hash.} 
	\item \textsc{Metadata are given in commented lines at the beginning of the text file in a basic \textsc{tag: value} format. }Metadata about the orthographic description given in the orthography profile includes, minimally, (i) author, (ii) date, (iii) title, (iv) a stable language identifier encoded in BCP 47/ISO 639-3, and (v) bibliographic data for resource(s) that illustrate the orthography described in the profile. 
\end{enumerate}

The content of a profile consists of lines, each describing a grapheme of the
orthography, using the following columns:

\begin{enumerate}
	\def\labelenumi{\arabic{enumi}.} 
	\item \textsc{A minimal profile consists of a single column with header \textsc{graphemes}}, listing each of the different graphemes in a separate line. 
	\item \textsc{Optional columns called \textsc{left} and \textsc{right} can be used to specify the left and right context of the grapheme, respectively.} The same grapheme can occur multiple times with different contextual specifications, for example to distinguish different pronunciations depending on the context. 
	\item \textsc{The columns \textsc{grapheme}, \textsc{left} and \textsc{right} can use regular expression metacharacters.} If regular expressions are used, then all literal usage of the special symbols, like full stops <.> or dollar signs <\$> (so-called \textsc{metacharacters}) have to be explicitly escaped by adding a backslash before them (i.e.~use <.> or <\$>). Note that any specification of context automatically expects regular expressions, so it is probably better to always escape all regular expression metacharacters when used literally in the orthography, i.e.~the following symbols will need to be preceded by a backslash: {[} {]} ( ) \{ \} ~+ * . - ! ? \^{} \$ . 
	\item \textsc{An optional column called \textsc{class} can be used to specify classes of graphemes}, for example to define a class of vowels. Users can simply add ad-hoc identifiers in this column to indicate a group of graphemes, which can then be used in the description of the graphemes or the context. The identifiers should of course be chosen such that they do not conflate with any symbols used in the orthography themselves. Note that such classes only refer to the graphemes, not to the context. 
	\item \textsc{Columns describing transliterations for each graphemes can be added and named at will}. Often more than a single possible transliteration will be of interest. Any software application using these profiles should use the names of these columns to select a specific transliteration column. 
	\item \textsc{Any other columns can be added freely, but will mostly be ignored by any software application using the profiles}. As orthography profiles are also intended to be read and interpreted by humans, it is often highly useful to add extra information on the graphemes in further columns, like for example Unicode codepoints, Unicode names, frequency of occurrence, examples of occurrence, explanation of the contextual restrictions, or comments. 
\end{enumerate}

For the automatic processing of the profiles, the following technical standards
will be expected:

\begin{enumerate}
	\def\labelenumi{\arabic{enumi}.} 
	\item \textsc{Each line of a profile will be interpreted as a regular expression. }Software applications using profiles can also offer to interpret a profile in the literal sense to avoid the necessity for the user to escape regular expressions metacharacters in the profile. However, this only is possible when no contexts or classes are described, so this seems only useful in the most basic orthographies. 
	\item \textsc{The \textsc{class} column will be used to produce explicit \textsc{or} chains of regular expressions}, which will then be inserted in the \textsc{graphemes}, \textsc{left} and \textsc{right} columns at the position indicated by the class-identifiers. For example, a class \textsc{V} as a context specification might be replaced by a regular expression like: (a\textbar{}e\textbar{}i\textbar{}o\textbar{}u\textbar{}ei\textbar{}au). Only the graphemes themselves are included here, not any contexts specified for the elements of the class. 
	\item \textsc{The \textsc{left} and \textsc{right} contexts will be included into the regular expressions by using lookbehind and lookahead}. Basically, the actual regular expression syntax of lookbehind and lookahead is simply hidden to the users by allowing them to only specify the contexts themselves. Internally, the contexts in the columns \textsc{left} and \textsc{right} are combined with the column \textsc{graphemes} to form a complex regular expression like: (?\textless{}=left)graphemes(?=right). 
	\item \textsc{The regular expressions will be applied in the order as specified in the profile, from top to bottom.} A software implementation can offer help in figuring out the optimal ordering of the regular expressions, but should then explicitly report on the order used. 
\end{enumerate}

The actual implementation of the profile on some text-string will function as
follows:

\begin{enumerate}
	\def\labelenumi{\arabic{enumi}.} 
	\item \textsc{All graphemes are matched in the text before they are tokenized or transliterated}. In this way, there is no necessity for the user to consider `feeding' and `bleeding' situations, in which the application of a rule either changes the text so another rule suddenly applies (feeding) or prevents another rule to apply (`bleeding'). 
	\item \textsc{The matching of the graphemes can occur either globally or linearly. }From a computer science perspective, the most natural way to match graphemes from a profile in some text is by walking linearly through the text-string from left to right, and at each position go through all graphemes in the profile to see which one matches, then go to the position at the end of the matched grapheme and start over. This is basically how a finite state transducer works, which is a well-established technique in computer science. However, from a linguistic point of view, our experience is that most linguists find it more natural to think from a global perspective. In this approach, the first grapheme in the profile is matched everywhere in the text-string first, before moving to the next grapheme in the profile. Theoretically, these approaches will lead to different results, though in practice of actual natural language orthographies they almost always lead to the same result. Still, we suggest that any software application using orthography profiles should offer both approaches (i.e. \textsc{global} or \textsc{linear}) to the user. The approach used should be documented in the metadata as \textsc{tokenization method}. 
	\item \textsc{The matching of the graphemes can occur either in NFC or NFD. }By default, both the profile and the text-string to be tokenized should be treated as NFC (see section \ref{pitfall-canonical-equivalence}, Pitfall: Canonical equivalence, above). However, in some use-cases it turns out to be practical to treat both text and profile as NFD. This typically happens when very many different combinations of diacritics occur in the data. An NFD-profile can then be used to first check which individual diacritics are used, before turning to the more cumbersome inspection of all combinations. We suggest that any software application using orthography profiles should offer both approaches (i.e. \textsc{NFC} or \textsc{NFD}) to the user. The approach used should be documented in the metadata as \textsc{unicode normalization}. 
	\item \textsc{The text-string is always returned in tokenized form} by separating the matched graphemes by a user-specified symbols-string. Any transliteration will be returned on top of the tokenization. 
	\item \textsc{Leftover characters (i.e.~characters that are not matched by the profile) should be reported to the user as errors.} Typically, the unmatched character are replaced in the tokenization by a user-specified symbol-string. 
\end{enumerate}

Any software application offering to use orthography profile:

\begin{enumerate}
	\def\labelenumi{\arabic{enumi}.} 
	\item \textsc{should offer user-options} to specify:
	\begin{enumerate}
		\def\labelenumii{\arabic{enumii}.} 
		\item \textsc{the name of the column to be used for transliteration} (if any). 
		\item \textsc{the symbol-string to be inserted between graphemes.} Optionally, a warning might be given if the chosen string includes characters from the orthography itself. 
		\item \textsc{the symbol-string to be inserted for unmatched strings} in the tokenized and transliterated output. 
		\item \textsc{the tokenization method}, i.e.~whether the tokenization should proceed \textsc{global} or \textsc{linear}. 
		\item \textsc{unicode normalization}, i.e.~whether the text-string and profile should use \textsc{NFC} or \textsc{NFD}. 
	\end{enumerate}
	\item \textsc{might offer user-options }to:
	\begin{enumerate}
		\def\labelenumii{\arabic{enumii}.} \setcounter{enumii}{5} 
		\item \textsc{assist in the ordering of the graphemes.} In our experience, it makes sense to apply larger graphemes before shorter graphemes, and to apply graphemes with context before graphemes without context. Further, frequently relevant rules might be applied after rarely relevant rules (though frequency is difficult to establish in practice, as it depends on the available data). Also, if this all fails to give any decisive ordering between rules, it seems useful to offer linguists the option to reverse the ordering from any manual specified ordering, because linguists tend to write the more general rule first, before turning to exceptions or special cases. 
		\item \textsc{assist in dealing with upper and lower case characters.} It seems practical to offer some basic case matching, so characters like <a> and <A> are treated equally. This will be useful in many concrete cases, although the user should be warned that case matching does not function universally in the same way across orthographies. Ideally, users should prepare orthography profiles with all lowercase and uppercase variants explicitly mentioned, so by default no case matching should be performed. 
		\item \textsc{treat the profile literal}, i.e.~to not interpret regular expression metacharacters. Matching graphemes literally often leads to strong speed increase, and would allow users to not needing to worry about escaping metacharacters. However, in our experience all actually interesting use-cases of orthography profiles include some contexts, which automatically prevents any literal interpretation, so by default the matching should not be literal. 
	\end{enumerate}
	\item \textsc{should return the following information} to the user:
	\begin{enumerate}
		\def\labelenumii{\arabic{enumii}.} \setcounter{enumii}{8} 
		\item \textsc{the original text-strings to be processed in the used Unicode normalization}, i.e.~in either NFC or NFD as specified by the user. 
		\item \textsc{the tokenized strings}, with additionally any transliterated
        strings, if transliteration is requested. 
		\item \textsc{a survey of all errors encountered}, ideally both in which
        text-strings any errors occurred and which characters in the
        text-strings lead to errors. 
		\item \textsc{a reordered profile}, when any automatic reordering is offered 
	\end{enumerate}
\end{enumerate}

\section{Examples}

[Here should a few abstract short simple examples be added]

\ 

Note that to deal with ambiguous parsing cases, we can use the Unicode approach
using the zero width joiner. This is actually a non-joiner (the name is
confusing): the idea is to add this character into the text to identify cases in
which a sequence of characters is not supposed to be a complex grapheme (even
though the sequence is in the orthography profile)


\chapter{Use cases}
\label{use-cases}

\section{Introduction}

We now present several use cases that have motivated the development of orthography profiles. These include:
\begin{itemize}
	\item tokenization and error checking
	\item normalization of orthographic systems to attain interoperability across different source documents, 
	\item cognate identification for detecting the similarity between words from different languages 
\end{itemize}

An important assumption of our work is that input sources are encoded in Unicode (UTF-8 to be precise). Anything that the Unicode Standard is unable to capture, cannot be captured by an orthography profile. We also use Unicode normalization form NFD to decompose all incoming text input into normalized and (Unicode) logically ordered strings. This process is of fundamental importance when working with text data in the Unicode Standard.

\section{Tokenization and error checking}
The most basic use case is tokenization of language data. Tokenization comes in three types:

- Unicode character tokenization
- grapheme tokenization
- tailored grapheme tokenization

The simplest form of tokenization provided by any Unicode Standard-compliant software will split an input text stream on Unicode character code points. The input text is split on the character bit sequences as they have been encoded by the user. This may be in various Unicode Normalization Forms (see Section \ref{}) and tokenization split function returns a byte stream sequence given a particular encoding (see Section \ref{}).

There is increasing support for the regular expression match, commonly ``\\X'', to identify Unicode graphemes (see Section \ref{}) and perform tokenization on these sets of Unicode characters. Examples are given in Section \ref{}.

The orthography profile provides the mechanism for tokenizing an input stream on tailored grapheme clusters. The orthography's formal specification is given in Section \ref{}. To sum, an orthography profile is a list of tailored grapheme clusters and / or orthographic rules that specify how a specific resource of text input should be tokenized. 

Once text has been normalized, an additional and straightforward process is Unicode grapheme tokenization. More recently this regular expression, often available as short cut ``\textbackslash{X}'', identifies all sequences of ``base'' characters followed by 0 or more combining diacritics. For example, the sequence, % <ŋ͡mṍraː> would tokenize as a space-delimited sequence as: <ŋ͡ m ṍ r a ː>. For a first pass at identifying grapheme clusters, this is a straightforward 
tokenization. However, the linguist will note that both the tie bar (COMBINING DOUBLE INVERTED BREVE, a Unicode Combining Modifier) and the length marker (MODIFIER LETTER TRIANGULAR COLON. a Unicode Letter Modifier) do not appear ``correctly'', i.e.~the tie bar is grouped with the first character in the sequence and the length marker appears singly. This is necessary, as defined by the Unicode Standard, because of these character's semantic properties -- with Unicode we cannot know if, for example, the MODIFIER LETTER TRIANGULAR COLON should appear before or after the base character that it modifies, cf.~the IPA aspiration marker, , the MODIFIER LETTER SMALL H, which is also a Unicode Letter Modifier and for linguists a diacritic that be be used for pre- or post-aspiration. These ambiguities of position in tokenization are not uncommon in IPA, thus orthographic (or source) normalization and tokenization is needed.

\section{Cross-orthographic analysis of writing systems}

Given that orthography profiles are stored in a standard CSV format, we can use tools for converting and working with CSV. One such tool is the command line utility \textsc{csvkit}.\footnote{\url{https://csvkit.readthedocs.org}}

% insert some examples of creating a table of language | grapheme, language | orthographic contexts
% any measure of writing systems complexity?

\begin{comment}

\section{Cross-orthography **normalization*}

Orthographic tokenization requires a specification of the tailored grapheme clusters (linguists: graphemes) to separate orthographic units within an input source. In the simplest case, the Unicode graphemic tokenization with regular expression ``\textbackslash{X}'' captures the grapheme clusters, i.e.~the input data is graphemically tokenized and there are no additional string sequences required to capture each grapheme, e.g.~no specification. This scenario is the most unlikely.

The creation of an orthography typically requires that someone describes knowledge about the input data source's writing system. For example, when orthographically tokenizing a dictionary as input, the sequences of grapheme clusters need to be explicitly stated or a regular expression rule can be specified to transform the input. We illustrate these methods with examples from Thiesen 1998, a bilingual dictionary of Diccionario Bora-Castellano, and with Huber \& Huber 1992, a comparative vocabulary of sixty-nine indigenous languages of Columbia.

Huber \& Huber 1992

% ({[}a\textbar{}á\textbar{}e\textbar{}é\textbar{}i\textbar{}í\textbar{}o\textbar{}ó\textbar{}u\textbar{}ú{]})(n)(\s)({[}a\textbar{}á\textbar{}e\textbar{}é\textbar{}i\textbar{}í\textbar{}o\textbar{}ó\textbar{}u\textbar{}ú{]}), \1 \2 \4

A preliminary orthography profile can be generated with code available online.\footnote{\url{https://github.com/bambooforest/tokenizer}} The Python script creates a unigram model for individual Unicode characters or Unicode grapheme clusters and their frequency in the source input. We find that this gives a broad overview of the input at the character and grapheme levels and it often finds typographic errors in the source input by identifying rarely occurring or one-off characters in the input. The models also provides the basis for the ``quick'' development of an initial orthography profile by saving the user keyboarding time.

Note that these symbols are still phonetically unspecified, i.e.~symbol may or may not represent, say, a front open unrounded vowel.

The orthography profile is designed so that specified strings of characters are captured for tokenization.

\% how the tokenization works

The graphemic cluster specifications in the orthography profile are read into a trie data structure, which is used to tokenize the incoming input source by identifying sentinels (tokenization boundaries) in the input stream, which we then separate grapheme clusters by spaces.

An illustration is given in Figure N.

Comparative wordlists often start as the compilation of different sources with different writing systems. The Dogon comparative wordlist is an example of a compilation of slightly divergent transcription practices, which we will use to illustrate our point and to orthographically normalize an input source by transforming different transcriptions into a single IPA-like transcription.

*Orthographic normalization / cross-source **normalization*

Once we have normalized and tokenized our text elements, we can apply various sorts of source normalization across resources to create interoperable and comparable data.

Sound-based normalization is practical because \ldots{}

We use orthography profiles in our work to describe cross-linguistic difference between writing systems.

\emph{Cognate detection}

The identification of cognates is a non-trivial task, which involves identifying the semantic and phonetic similarity of words.

use grapheme correspondences to approximate cognacy and sound correspondences

to identify cognates and sound changes

Examples

Orthographic similarity != Phonetic similarity != Regular sound change

Cognate identification is further complicated by the fact that languages are written differently, and even more so for the low resource languages that we work with, linguists' phonetic transcriptions are heterogeneous and typically document-specific, i.e.~two linguists working on the same language variety most often develop different practical orthographies by using different symbols and different orthographic rules. Thus the essence of language documentation and description is diverse and ephemeral in nature. The encoding of linguistic data and linguistic data models is diverse and abstracting away from individual analytical preferences requires a document-by-document approach and a mechanism for standardization of writing to form an interlingual pivot so that different descriptions can be normalized into one encoding system to undertake analysis.

\% Orthography profile implementation in Lingpy / generic tokenizer

\% describe Tokenizer's function

\% describe algorithm for tokenizing IPA (Unicode grapheme cluster tokenization and then alignment of Letter Modifiers and Tie Bars)

\% insert an example of orthographic tokenization from PAD

\section{move Letter Modifiers left}\label{move-letter-modifiers-left}

(\s)(\p{Lm}), \2

which catches:

\section{Diacritics: Unicode letter modifiers (Lm)}\label{diacritics-unicode-letter-modifiers-lm}

\section{({[}ː\textbar{}ʰ\textbar{}ʼ\textbar{}ʿ\textbar{}ʾ{]})}\label{ux2d0ux2b0ux2bcux2bfux2be}

from the set of this:

\section{Diacritics: all}\label{diacritics-all}

\section{({[}ː\textbar{}ʰ\textbar{}ʼ\textbar{}ʿ\textbar{}ʾ\textbar{}̜\textbar{}̠\textbar{}̟\textbar{}͡\textbar{}̪\textbar{}̰\textbar{}̣\textbar{}̥\textbar{}̩\textbar{}‿\textbar{}.\textbar{}→\textbar{}˻\textbar{}+\textbar{}̃\textbar{}̫\textbar{}̤{]})}\label{ux2d0ux2b0ux2bcux2bfux2be.}

and allows us to with one regular expression do this:

PAD examples

A side effect of character tokenization is identification of outliers, e.g.

Some examples in Python

Some examples in R

{[}UNICODE USE-CASES{]}

The most basic overall Unicode character property is the \textsc{General Category}, which categorizes Unicode characters into: \emph{Letters, Punctuation, Symbols, Marks, Numbers, Separators, }and* Other*. Each Unicode character property also has a character property value. So each General Category is denoted by a property value by a single letter abbreviation, e.g. ``L'' for Letter. Each General Category property also has subcategories that are also properties with property values. For example, ``Letters'' is broken down into Uppercase Letter (Lu), Lowercase Letter (Ll), Titlecase Letter (Lt), Modifier Letter (Lm) and Other Letter (Lo).\footnote{The subcategory ``Other Letter'' includes characters such as uncased letters as in the Chinese writing system.} By defining character properties for each code point, algorithmic implementations such as text processing applications or regular expression engines that are conformant to the Unicode Standard can be used to recognize whole (sub)categories of characters or ranges of characters, e.g.~to match proper names in English in title case or more generally to match character types, such as scripts, or symbols and integers in email addresses, phone numbers, etc.

Script information is useful for algorithmic processes, such as collation, searching and for multilingual text processing. For example, if a regular expression engine includes functionality to match characters at the level of script, then it provides the ability to identify characters, e.g. \match{Greek}, which tests whether letters in a text belong to the Greek script or not. This functionality can be used to determine the boundaries between different writing systems that appear in the same document. In Section 6 (Use cases of grapheme tokenization) we provide an example of a regular expression match at the level of character property.

Unicode-aware regular expression engines may contain a meta-character, ``\textbackslash{X}'', that matches this level of grapheme cluster, i.e.~a base character followed by some number of accents or diacritical marks. Grapheme clusters are important to determine word or line boundaries, for collation (`ordering') and for user interface interaction (e.g.~mouse selection, backspacing, cursor movement).

\end{comment}


\chapter{Testing code inclusion}


Trying to include code through knitr. 
Different languages can be easily 
included!

There is a bug for non-R engines: the fontsize settings and line spaceing are not correct. I have 
filed an issue, hopefully this will be fixed soon. 

For syntax colouring, install the package `highlight' and for nice tables 
install the package `xtable'.

\begin{knitrout}\scriptsize
\definecolor{shadecolor}{rgb}{1, 1, 1}\color{fgcolor}\begin{kframe}
\begin{verbatim}
echo 'this is a simple bash example'
ls | wc
## this is a simple bash example
##       18      18     306
\end{verbatim}
\end{kframe}
\end{knitrout}

\begin{knitrout}\scriptsize
\definecolor{shadecolor}{rgb}{1, 1, 1}\color{fgcolor}\begin{kframe}
\begin{verbatim}
## note that bash chunks are executed as one, with all output at the end.
## You will have to make different chunks to get output separated.
## And add manual linebreaks!
\end{verbatim}
\end{kframe}
\end{knitrout}

Spacing of lines is fine. I think without background looks better

\begin{knitrout}\scriptsize
\definecolor{shadecolor}{rgb}{1, 1, 1}\color{fgcolor}\begin{kframe}
\begin{verbatim}
# adding comments inside code? Probably not a good idea
(example <- "this is a simple R example")
## [1] "this is a simple R example"

test <- 3 + 4
test
## [1] 7
\end{verbatim}
\end{kframe}
\end{knitrout}

\begin{knitrout}\scriptsize
\definecolor{shadecolor}{rgb}{1, 1, 1}\color{fgcolor}\begin{kframe}
\begin{verbatim}
library(qlcData)
\end{verbatim}


{\ttfamily\noindent\bfseries\color{errorcolor}{\#\# Error in library(qlcData): there is no package called 'qlcData'}}\begin{verbatim}
profile <- write.profile("AΑА")
\end{verbatim}


{\ttfamily\noindent\bfseries\color{errorcolor}{\#\# Error in eval(expr, envir, enclos): could not find function "{}write.profile"{}}}\end{kframe}
\end{knitrout}

\begin{kframe}


{\ttfamily\noindent\bfseries\color{errorcolor}{\#\# Error in print(xtable(profile, caption = "{}test"{}, label = "{}bla"{}), include.rownames = FALSE, : could not find function "{}xtable"{}}}\end{kframe}



You can refer to variables by using \textbackslash Sexpr{}. 
For example, the length of the example string is 26.
Crossreferencing also works, see Table~\ref{bla}.

\begin{knitrout}\scriptsize
\definecolor{shadecolor}{rgb}{1, 1, 1}\color{fgcolor}\begin{kframe}
\begin{verbatim}
print('bla' + 'bla')
print(3+4)
## blabla
## 7
\end{verbatim}
\end{kframe}
\end{knitrout}

You have to add the option ``engine.path='python3'' to get Python 3.

\begin{knitrout}\scriptsize
\definecolor{shadecolor}{rgb}{1, 1, 1}\color{fgcolor}\begin{kframe}
\begin{verbatim}
print('python3 needs an extra tag')
uni = 'aɽɮz'
print(uni)
## python3 needs an extra tag
## aɽɮz
\end{verbatim}
\end{kframe}
\end{knitrout}

Testing: does referencing to variables also work in Python? It does give an error, so no... !


% TODOS:

% Lots of characters within <> were lost in translation...
% some things to note in the translation to LaTeX:
% tildes in URLs broken -- turned into \textasciitilde{}
% <a> etc. seem to sometimes be lost
% section in some cases should be "chapter"


% other stuff to fix / proof read
% references
% sections, e.g. Section 4, Pitfall 4)
% URLs
% `', ``''
% reinsertion of a lot of the glyphs, etc.
% fix ligatures
% all glyphs need to be checked
% all <>s need to be checked and/or readded

\begin{comment}
===
Double quotation marks are generally used for distancing, in particular in the following situations:

1. when a passage from another work is cited in the text (e.g. According to Takahashi (2009: 33), “quotatives were never used in subordinate clauses in Old Japanese”); but block quotations do not have quotation marks;
2. when a technical term is mentioned that the author does not want to adopt, but wants to mention, e.g. This is sometimes called “pseudo-conservatism”, but I will not use this term here, as it could lead to confusion.

Single quotation marks are used exclusively for linguistic meanings, as in the following: Latin habere ‘have’ is not cognate with Old English hafian ‘have’.

===
so, we should normally use double quotation in most of our cases :-). In general though, I would like us to try and remove double quotations as much as possible. Mostly is thus signals uncertainty on our part :-).

\end{comment}

%%%%%%%%%%%%%%%%%%%%%%%%%%%%%%%%%%%%%%%%%%%%%%%%%%%%
%%%                                              %%%
%%%             Backmatter                       %%%
%%%                                              %%%
%%%%%%%%%%%%%%%%%%%%%%%%%%%%%%%%%%%%%%%%%%%%%%%%%%%%

% There is normally no need to change the backmatter section
\backmatter
\phantomsection%this allows hyperlink in ToC to work
{\sloppy\printbibliography[heading=\lsReferencesTitle]}
\cleardoublepage

\phantomsection 
\addcontentsline{toc}{chapter}{\lsIndexTitle} 
\addcontentsline{toc}{section}{\lsNameIndexTitle}
\ohead{\lsNameIndexTitle} 
\printindex 
\cleardoublepage
  
\phantomsection 
\addcontentsline{toc}{section}{\lsLanguageIndexTitle}
\ohead{\lsLanguageIndexTitle} 
\printindex[lan] 
\cleardoublepage
  
\phantomsection 
\addcontentsline{toc}{section}{\lsSubjectIndexTitle}
\ohead{\lsSubjectIndexTitle} 
\printindex[sbj]
\ohead{} 

\end{document}

% you can create your book by running
% xelatex lsp-skeleton.tex
%
% you can also try a simple 
% make
% on the commandline
