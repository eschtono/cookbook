\chapter{Encoding}
\label{encoding}

There are many in-depth histories of the origin and development of writing systems (e.g.~\citet{Robinson1997,Powell2012}), a story that we will not repeat here. However, the history of turning writing into computer readable code is not so often told, so we decided to offer a short survey of the major developments here. This history turns out to be intimately related to the history of telegraphic communication.\footnote{Because of the recent history as summarized in this section, we have used mostly rather ephemeral internet sources. When not references by traditional literature in the bibliography, we have specifically used \url{http://www.unicode.org/history/} and various Wikipedia pages for the information presented here. A useful survey of the historical development of the physical hardware of telegraphy and telecommunication is \citet{Huurdeman2003}. Most books that discuss the development of encoding of telegraphic communication focus of cryptography, e.g.~\citet{Singh1999}, and forego the rather interesting story of `open' encoding that is related here.}

Writing systems have existed for roughly 6000 years, allowing people to exchange messages through time and space. Additionally, to bridge large geographic distances, \textsc{telegraphic systems} of communication (from Greek \emph{τῆλε γράφειν} `distant writing') have a long and widespread history since ancient times. The most widespread telegraphic systems worldwide are so-called whistled languages \citep{Meyer2015}, but also drumming languages \citep{Meyer_etal2012} and smoke, fire, hydraulic or flag signals are forms of telegraphy. 

Telegraphy was reinvigorated in the end of the eighteenth century through the introduction of semaphoric systems by Claude Chapelle (since then using flags, flashing lights, or various specially designed contraptions) to convey messages over large distances. The \textit{innovation} of those systems was that all characters of the written language were replaced one-to-one by visual signals. Since then, all telegraphic systems have taken this principle, namely that any language to be transmitted first has to be turned into some orthographic system, which subsequently is encoded for transmission by the sender, and then turned back into orthographic representation at the receiver side.\footnote{Sound and video-based telecommunication of course takes a different approach by ignoring the written version of language and directly encode sound waves or light patterns.} This of course implies that the usefulness of any such telegraphic encoding completely depends on the sometimes rather haphazard structure of orthographic systems.

In the nineteenth century, \textsc{electric telegraphy} lead to a new approach in which written language characters were encoded by signals sent through a copper wire. Originally, \textsc{bisignal} codes were used, consisting of two different signals. For example, Carl Friedrich Gauss in 1833 used positive and negative current \citep[282]{Mania2008}. More famous and influential, Samuel Morse in 1836 used long and short pulses. 

In those \textsc{bisignal codes} each character from the written language was encoded with a different number of signals (between one and five), so two different separators are needed: one between signals and one between characters. For example, in Morse-code there is a short pause between signals and a long pause between characters. Actually, Morse-code also includes an extra long pause between words. Interestingly, it took a long time to consider the written word boundary---using white-space---as a bona-fide character that should simply be encoded with its own code point. This happened only with the revision of the Baudot-code (see below) by Donald Murray in 1901, in which he introduced a specific white-space code. This principle has been followed ever since.

From those bisignal encodings, true \textsc{binary codes} developed with a fixed length of signals per character. In such systems only a single separator between signals is needed, because the separation between characters can be established by counting until a fixed number of signals.\footnote{Of course, no explicit separator is needed when the timing of the signals is known, which is the principle used in all modern telecommunication systems. An important modern consideration is also how to know where to start counting when you did not catch the start of a message, something that is known in Unicode as \textsc{self synchronization}.} 

In the context of electric telegraphy, such a binary code system was first established by Émile Baudot in 1870, using a fixed combination of five signals for each written character.\footnote{True binary codes have a longer history, going at least back to the Baconian cipher devised by Francis Bacon in 1605. However, the proposal by Baudot was the quintessential proposal leading to all modern systems.} There are $2^5 = 32$ possible combination when using five binary signals; an encoding today designated as \textsc{5-bit}. These codes are sufficient for all Latin letters, but of course they do not suffice for all written symbols, including punctuation and digits. As a solution, the Baudot code uses a so-called \textsc{shift} character, which signifies that from that point onwards---until shifted back---a different encoding is used, allowing for yet another set of 32 codes. In effect, this means that the Baudot code, and the \textsc{International Telegraph Alphabet} (ITA) derived from it, had an extra `bit' of information, so the encoding is actually 6-bit (with $2^6 = 64$ different possible characters). For decades, this encoding was the standard for all telegraphy and it is still in limited use today.

To also allow for different uppercase and lowercase letters and a large variety of control characters to be used in the newly developing technology of computers, the American Standards Association decided to propose a new 7-bit encoding in 1963 (with $2^7 = 128$ different possible characters), known as the \textsc{American Standard Code for Information Interchange} (ASCII), geared towards the encoding of English orthography. 

With the ascent of other orthographies in computer usage, the wish to encode further variation of Latin letters (like German <ß> or various letters with diacritics like <è>) led the Digital Equipment Corporation to introduce an 8-bit \textsc{Multinational Character Set} (MCS, with $2^8 = 256$ different possible characters), first used with the introduction of the {\small VT}220 Terminal in 1983. 

Because 256 characters were clearly not enough for the many different characters needed in the world's writing systems, the ISO/IEC~8859 standard in 1987 extended the MCS to include 16 different 8-bit code pages. For example, part 5 was used for Cyrillic characters, part 6 for Arabic, and part 7 for Greek.\footnote{In effect, because $16 = 2^4$, this means that ISO/IEC~8859 was actually a $8+4=12$-bit encoding, though with very many duplicates by design (e.g.~all ASCII codes were repeated in each 8-bit code page). To be precise, ISO/IEC~8859 used the 7-bit ASCII as the basis for each code page, and defined 16 different 7-bit extensions, leading to $(1+16)\cdot{2^7} = 2,176$ possible characters.}

This system almost immediately was understood to be insufficient and impractical, so various initiatives to extend and reorganize the encoding started in the 1980s. This led, for example, to various proprietary encodings from Microsoft (e.g.~Windows Latin 1) and Apple (e.g.~Mac OS Roman), which one still sometimes encounters today. 

More wide-ranging, various people in the 1980s started to develop true international code sets. In the United States, a group of computer scientists formed the \textsc{Unicode Consortium}, proposing a 16-bit encoding in 1991 (with $2^{16} = 65,536$ different possible characters). At the same time in Europe, the \textsc{International Organization for Standardization} (ISO) was working on ISO~10646 to supplant the ISO/IEC~8859 standard. Their first draft of the \textsc{Universal Character Set} (UCS) in 1990 was 31-bit (with theoretically $2^{31} = 2,147,483,648$ possible characters, but because of some technical restrictions only 679,477,248 were allowed). 

Since 1991, the Unicode Consortium and the ISO jointly develop the \textsc{unicode standard}, or ISO/IEC~10646, leading to the current system including the original 16-bit Unicode proposal as the \textsc{basic multilingual plane}, and 16 additional planes of 16-bit for further extensions (with in total $(1+16) \cdot 2^{16} = 1,114,112$ possible characters). The most recent version of the Unicode Standard (7.0) was published in June 2014 and it defines 112,956 different characters \citep{Unicode2014}.