\chapter{IPA meets Unicode}
\label{ipa-meets-unicode}

% ==========================
\section{The International Phonetic Alphabet (IPA)}
\label{the-international-phonetic-alphabet}
% ==========================

The International Phonetic Alphabet (IPA) is a common standard in linguistics to
transcribe sounds of spoken language into Latin-based characters
\citep{IPA2005}. Although IPA is reasonably easily adhered to with pen and
paper, it is not trivial to encode IPA characters electronically. Similar to the
previous chapter, this chapter will discuss various pitfalls with the encoding
of the IPA.\@ The details of the encoding are unimportant as long as the
transcription is only directed towards phonetically trained eyes. For a linguist
reading an IPA transcription, many of the details that will be discussed in this
chapter might seem like hair-splitting trivialities. However, if IPA
transcriptions are intended to be used across resources (e.g.~searching similar
phenomena across different languages) then it becomes crucial that there are strict
encoding guidelines. Our main goal in this chapter is to present the encoding
issues and propose recommendations for a ``strict'' IPA encoding.

\ 

\noindent For a long time, linguists, like all other computer users, were limited to
ASCII-encoded 7-bit characters, which only includes Latin characters, numbers
and some punctuation and symbols. Restricted to these standard character sets
that lacked IPA support or other language-specific graphemes that they needed,
linguists devised their own solutions.\footnote{Early work addressing the need
for a universal computing environment for writing systems and their
computational complexity is discussed in \citet{Simons1989}. A survey of
practical recommendations for language resources, including notes on encoding,
can be found in \citet{BirdSimons2003}} For example, some chose to represent
unavailable graphemes with substitutes, e.g.~the combination of <ng> to
represent <ŋ>. Tech-savvy linguists redefined selected characters from a
character encoding by mapping custom made fonts to those code points. However,
one linguist's electronic text would not render properly on another linguist's
computer without access to the same font. Further, if two character encodings
defined two character sets differently, then data could not be reliably and
correctly displayed. This is a common example of the non-interoperability of
data and data formats.

To alleviate this problem, during the late 1980s, SAMPA (Speech Assessment
Methods Phonetic Alphabet) was created to represent IPA symbols with ASCII
character sequences, e.g. <p\textbackslash> for [ɸ]. Two problems with SAMPA are
that (i) it is only a partial encoding of the IPA and (ii) it encodes different
languages in separate data tables, instead of using a universal alphabet, like
IPA.\@ SAMPA tables are derived from phonemes appearing in several European
languages that were developed as part of a European Commission-funded project to
address technical problems like electronic mail exchange (what is now simply
called email). SAMPA is essentially a hack to work around displaying IPA
characters, but it provided speech technology and other fields a basis that has
been widely adopted and used in code. So, SAMPA was a collection of tables to be
compared, instead of a large universal table representing all languages. An
extended version of SAMPA, called X-SAMPA, set out to include every symbol in
the IPA chart including all diacritics \citep{WellsND}. X-SAMPA was considered
more universally applicable because it consisted of one table that encoded all
characters that represented phones/segments in IPA across languages. SAMPA and
X-SAMPA have been widely used for speech technology and as an encoding system in
computational linguistics. Eventually, ASCII-encoding of the IPA became
deprecated through the advent of the Unicode Standard. Note however that many
popular software packages used for linguistic analyses still require ASCII
input, e.g.~RuG/L04 and SplitsTree4.\footnote{See
\url{http://www.let.rug.nl/kleiweg/L04/} and \url{http://www.splitstree.org/},
respectively}

There are a few pitfalls to be aware of when using the Unicode Standard to
encode IPA.\@ As we have said before, from a linguistic perspective it might
sometimes look like the Unicode Consortium is making incomprehensible decisions,
but it is important to realize that the consortium has tried and is continuing
to try to be as consistent as possible across a wide range of use cases, and it
does place linguistic traditions above other orthographic possibilities. In
general, we strongly suggest linguists not to complain about any decisions in
the Unicode Standard, but to try and understand the rationale of the Unicode
Consortium (which in our experience is almost always well-conceived) and devise
ways to work with any unexpected behavior. Many of the current problems derive
from the fact that the IPA is clearly historically based on the Latin script,
but different enough from most other Latin-based writing systems to warrant
special attention. This ambivalent status of the IPA glyphs (partly Latin,
partly special) is unfortunately also attested in the treatment of IPA in the
Unicode Standard. In retrospect, it might have been better to consider the IPA
(and other transcription systems) to be a special kind of script within
the Unicode Standard, and treat the obvious similarity to Latin glyphs as a
historical relic. All IPA glyphs would then have their own code points, instead
of the current situation in which some IPA glyphs have special code points,
while others are treated as being identical to the regular Latin characters.
Yet, the current situation, however unfortunate, is unlikely to change, so as
linguists we will have to learn to deal with the specific pitfalls of IPA within
the Unicode Standard. In this section, we will describe these pitfalls in some
detail.

% ==========================
\section{Pitfall: No complete IPA code block}
\label{pitfall-no-complete-ipa-block}
% ==========================

The ambivalent nature of IPA glyphs arises because, on the one hand, the IPA
uses Latin-based glyphs like <a>, <b> or <p>. From this perspective, the IPA
seems to be just another orthographic tradition using Latin characters, all of
which do not get a special treatment within the Unicode Standard (just like
e.g.~the French, German, or Danish orthographic traditions do not have a special
status). On the other hand, the IPA uses many special symbols (like turned <ɐ>,
mirrored <ɘ> and/or extended <ɧ> Latin glyphs) not found in any other Latin-based
writing system. For this reason a special block with code points, called
\textsc{IPA Extensions} was included already in the first version of the Unicode
Standard (Version 1.0 from 1991).

As explained in Section~\ref{pitfall-blocks}, the Unicode Standard code space is
subdivided into character blocks, which generally encode characters from a
single script. However, as is illustrated by the IPA, characters that form a
single writing system may be dispersed across several different character
blocks. With its diverse collection of symbols from various scripts and
diacritics, the IPA is spread across 13 blocks in the Unicode
Standard:\footnote{This number of blocks depends on whether only IPA-sanctioned
symbols are counted or if the phonetic symbols commonly found in the literature
are also included, see~\cite[Appendix~C]{Moran2012}.}

\begin{itemize}
	\item Basic Latin (27 characters): \newline < a~b~c~d~e~f~h~i~j~k~l~m~n~o~p~q~r~s~t~u~v~w~x~y~z~.~| >
	\item Latin-1 Supplement (4 characters): \newline < æ ç ð ø >
	\item Latin Extended-A (3 characters): \newline < ħ ŋ œ >
	\item Latin Extended-B (4 characters): \newline < ǀ ǁ ǂ ǃ >
	\item IPA Extensions (68 characters): \newline < ɐ ɑ ɒ ɓ ɔ ɕ ɖ ɗ ɘ ə ɛ ɜ ɞ ɟ ɠ ɡ ɢ ɣ ɤ ɥ ɦ ɧ ɨ ɪ ɫ ɬ ɭ ɮ ɯ ɰ ɱ ɲ ɳ ɴ ɵ ɶ ɸ ɹ ɺ ɻ ɽ ɾ ʀ ʁ ʂ ʃ ʄ ʈ ʉ ʊ ʋ ʌ ʍ ʎ ʏ ʐ ʑ ʒ ʔ ʕ ʘ ʙ ʛ ʜ ʝ ʟ ʡ ʢ > 
	\item Spacing Modifier Letters (17 characters): \newline < ʰ ʲ ʷ ʼ ˈ ˌ ː ˑ ˞ ˠ ˡ ˤ ˥ ˦ ˧ ˨ ˩ >
	\item Combining Diacritical Marks (29 characters): \newline < {\large \  ̃\ \ ̆\ \ ̈\ ̚\ \ ̘\ \ ̙\ \ \ ̜\ \ ̝\ \ ̞\ \ ̟\ \ ̠\ \ ̤\ \ ̥\ \ ̩\ \ ̪\ \ ̬\ \ ̯\ \ ̰\ \ ̹\ \ ̺\ \ ̻\ \ ̼\ \ ̽\ \ ͜\  ̋\ \ ́\ \ ̄\ \ ̀\ \ ̏\ } >
	\item Greek and Coptic (3 characters): \newline < β θ χ >
%	\item Phonetic Extensions (2 characters): \charis{ᴅ ᴴ} 
	\item Phonetic Extensions Supplement (1 characters): \newline < \charis{ᶑ} >
    \item General Punctuation (2 characters): \newline < ‖ \charis{‿} >
	\item Superscripts and Subscripts (1 character): \newline < ⁿ > 
	\item Arrows (4 characters): \newline < ↑ ↓ ↗ ↘ >
	\item Latin Extended-C (1 character): \newline < \charis{ⱱ} >
\end{itemize}

% ==========================
\section{Pitfall: IPA homoglyphs in Unicode}
\label{pitfall-ipa-homoglyphs}
% ==========================

Another problem is the large number of homoglyphs, i.e.~different characters
that have highly similar glyphs (or even completely identical glyphs, depending
on the font rendering). For example, a user of a Cyrillic computer keyboard
should ideally not use the <а> \textsc{cyrillic small letter a} at code point
\uni{0430} for IPA transcriptions, but instead use the <a> \textsc{latin small
letter a} at code point \uni{0061}, although visually they are mostly
indistinguishable, and the Cyrillic character is more easily typed on a Cyrillic
keyboard. Some problematic homoglyphs related to encoding IPA in the Unicode Standard are 
the following.
\begin{itemize}

  \item The uses of the apostrophe has led to long discussions on the Unicode
     Standard email list. An English keyboard inputs <{\fontspec{Monaco}'}> \textsc{apostrophe}
     at \uni{0027}, although the preferred Unicode apostrophe is the <\ ' >
     \textsc{right single quotation mark} at \uni{2019}.\footnote{Note that many word 
     processors (like Microsoft Word) by default will replace straight quotes by 
     curly quotes, depending on the whitespace around it.} However, the glottal stop/glottalization/ejective marker is yet 
     another completely different character <{\large ʼ}>, the 
     \textsc{modifier letter apostrophe} at \uni{02BC}, which unfortunately looks 
     mostly extremely similar to \uni{2019}. 
  \item Another problem is the <ˁ> \textsc{modifier letter reversed
     glottal stop} at \uni{02C1} vs.\@ the <ˤ> \textsc{modifier
     letter small reversed glottal stop} at \uni{02E4}. Both 
     appear in various resources representing phonetic data online. This is
     thus a clear example for which the Unicode Standard does not solve the
     linguistic standardization problem.
  \item even linguists are unlikely to distinguish between the <ə>
     \textsc{latin small letter schwa} at code point \uni{0259} and <ǝ>
     \textsc{latin small letter turned e} at \uni{01DD}.
  \item  The alveolar click <ǃ> at \uni{01C3} is of course often simply
     typed as <!> \textsc{exclamation mark} at \uni{0021}.\footnote{In the Unicode
     Standard the <ǃ> at \unif{01C3} is labeled \textsc{latin letter retroflex
     click}, but in IPA that glyph is used for an alveolar or postalveolar
     click (not retroflex). This naming is probably best seen as an 
     error in the Unicode Standard.}

\end{itemize} 


Conversely, non-linguists are unlikely to distinguish any semantic difference
between an open back unrounded vowel <ɑ> \textsc{latin small letter alpha} at
\uni{0251}, and the open front unrounded vowel <a> \textsc{latin small letter a}
at \uni{0061}, basically treating them as homoglyphs, although they are
different phonetic symbols. But even among linguists this distinction leads to problems. For example, as
pointed out by \citet{Mielke2009}, there is a problem stemming from the fact
that about 75\% of languages are reported to have a five-vowel system
\citep{Maddieson1984}. Historically, linguistic descriptions tend not to include
precise audio recording and measurements of formants, so this may lead one to
ask if the many <a> characters that are used in phonological description
reflects a transcriptional bias. The common use of <a> in transcriptions could
be in part due to the ease of typing the letter on an English keyboard (or for
older descriptions, the typewriter). We found it to be exceedingly rare that a
linguist uses <ɑ> for a low back unrounded vowel.\footnote{One example is
\citet[75]{Vidal2001a}, in which the author states: ``The definition of Pilagá
/a/ as [+back] results from its behavior in certain phonological contexts. For
instance, uvular and pharyngeal consonants only occur around /a/ and /o/. Hence,
the characterization of /a/ and /o/ as a natural class of (i.e., [+back]
vowels), as opposed to /i/ and /e/.''} They simply use <a> as long as there is
no opposition to <ɑ>.

%\footnote{See Thomason's Language Log post, ``Why I don't love the International Phonetic Alphabet'' at:\url{http://itre.cis.upenn.edu/~myl/languagelog/archives/005287.html}.}

Making things even more problematic, there is an old typographic tradition that
the double-story <a> uses a single-story <ɑ> in italics. This leads to the
unfortunate effect that in most well-designed fonts the italics of <a> and <ɑ>
use the same glyph. If this distinction has to be kept upright in italics, the
only solution we can currently offer is to use \textsc{slanted} glyphs
(i.e.~artificially italicized glyphs) instead of real italics (i.e.~special
italics glyphs designed by a typographer).\footnote{For example, the widely used
IPA font Doulos SIL
(\url{http://scripts.sil.org/cms/scripts/page.php?item\_id=DoulosSIL}) does not
have real italics. This leads some word-processing software, like Microsoft
Word, to produce slanted glyphs instead. That particular combination of font and
software application will thus lead to the desired effect distinguishing <a>
from <ɑ> in italics. However, note that when the text is transferred to another
font (i.e.~one that includes real italics) and/or to another software
application (like Apple Pages, which does not perform slanting), then this
visual appearance will be lost. In this case we are thus still in the
pre-Unicode situation in which the choice of font and rendering software
actually matters. The ideal solution from a linguistic point of view would be
the introduction of a new IPA code point for a different kind of <a>, which
explicitly specifies that it should still be rendered as a double-story
character when italicized. After informal discussion with various Unicode
players, our impression is that this highly restricted problem is not
sufficiently urgent to introduce even more <a> homographs in Unicode (which
already lead to much confusion, see Section~\ref{pitfall-homoglyphs}).}

% ==========================
\section{Pitfall: Homoglyphs in IPA}
\label{pitfall-homoglyphs-in-IPA}
% ==========================

It is not just the Unicode Standard that offers multiple options for encoding
the IPA.\@ Even the IPA specification itself offers some flexibility in how
transcriptions have to be encoded. There are a few cases in which the IPA
explicitly allows for different options of transcribing the same phonetic
content. This is understandable from a transcriber's point of view, but it is
not acceptable for interoperability between resources written in IPA.\@ We
consider it crucial to distinguish between ``lax'' IPA, for which it is
sufficient that any phonetically-trained reader is able to understand the
transcription, and ``strict'' IPA, which should be standardized on a single
unique encoding for each sound, so search will work across resources. We are 
aware of the following double options in the IPA, which will be discussed in 
turn below:

\begin{itemize}
  \item The marking of tone
  \item The marking of <g>
  \item The placement of diacritics
  \item The tie bar to indicate a close bond between sounds
\end{itemize}

The first case in which the IPA allows for different encodings is the question
of how to transcribe tone. There is an old tradition to use diacritics on vowels
to mark different tone levels, e.g. <ȅèée̋>.\footnote{To make things even more
complexe, there are at least two different Unicode homoglyphs for the low and
high level tones, namely <\ ̀~> \textsc{combining grave tone mark} at
\unif{0340} vs.\@ <\ ̀~> \textsc{combining grave accent} at \unif{0300} for low
tone, and <\ ́~> \textsc{combining acute tone mark} at \unif{0341} vs.\@ <\ ́~>
\textsc{combining acute accent} at \unif{0301} for high tone.} The IPA also
proposes the option of tone letters, e.g. <˥˦˧˨˩>, which are used much less, but
are more consistent for contours. Tone letters in the IPA have five different levels, and
sequences of these letters can be used to indicate contours. Well-designed fonts
will even merge a sequence of tone letters into a contour. For example, compare
the font Linux Libertine, which does not merge tone letters
<{\fontspec{LinLibertineO}˥˨˧˩}>, with the font CharisSIL, which merges this
sequence of four tone letters into a single contour <\charis{˥˨˧˩}>. For strict
IPA encoding we propose to standardize on tone letters.

Second, we commonly encounter the use of <g> \textsc{latin small letter g} at
\uni{0067}, instead of the Unicode Standard IPA character for the voiced velar
stop <ɡ> \textsc{latin small letter script g} at \uni{0261}. One begins to
question whether this issue is at all apparent to the working linguist, or if
they simply use the \uni{0067} because it is easily keyboarded and thus saves
time, whereas the latter must be cumbersomely inserted as a special symbol in
most software. The International Phonetic
Association has taken the stance that both the keyboard \textsc{latin small
letter g} and the \textsc{latin small letter script g} are valid input
characters for the voiced velar plosive. Unfortunately, this decision further
introduces ambiguity for linguists trying to adhere to a strict Unicode Standard
IPA encoding. For strict IPA encoding we propose to standardize on the more 
idiosyncratic \textsc{latin small letter script g} at \uni{0261}.

Third, for marking marking of voiceless pronunciation of voiced segments the IPA
uses the ring diacritic. Originally, the ring should be placed below the 
base character, like in <m̥>, using the \textsc{combining ring below} at \uni{0325}. 
However, in letters with long descenders the IPA also allows to put the ring 
above the base, like in <ŋ̊>, using the \textsc{combining ring above} at 
\uni{030A}. Yet, proper font design does not have any problem with rendering 
the ring below the base character, like in <ŋ̥>, so for strict IPA encoding we 
propose to standardize on the ring below.

Finally, the tie bar to indicate affricates, doubly articulated consonants or
diphthongs can be placed above or below the base characters, e.g.
<\charis{t͡s}> or <\charis{t͜s}>. IPA allows both the \textsc{combining double inverted
breve} at \uni{0361} and the \textsc{combining double breve below} at \uni{035C}. The  
choice between the two symbols is purely for legible rendering; there is no difference 
in semantics between the two symbols. In fact, rendering is such a problematic issue 
for tie bars in general, that many linguists simply do not use them. Further, tie bars 
are a special type of character in the sense that they bind two graphemes (singleton base characters 
or tailored grapheme clusters) together. In Section \ref{pitfall-different-notions-of-diacritics}, 
we highlight the division of linguistic diacritics 
into two categories in the Unicode Standard: Spacing Modifier Letters and Combining 
Diacritical Marks.\footnote{The former contain characters like aspiration < ʰ > which is user 
specified (before or after the base character) and the latter which combines 
with the base character.} The tie-bar is somewhere in between these two character 
classes, i.e. when the combining breve (above or below) is told to combine within 
a sequence of Unicode characters, it will bind to the character immediately 
preceding it -- the so-called base character -- regardless if that base 
character is a Spacing Modifier Letter \textit{or} Combining Diacritical Mark. \footnote{In some cases 
conversion of a string sequence into Unicode Normalization Form C alleviates 
the problem by composing sequences of characters like <e> + < ́ > into single character < é > 
(see however diacritics like the rhotic hook in ).} In principle, 
it is possible to combine more than two base characters by repeating the tie bar, 
like in <a͡o͡u>. If really necessary, we consider this possible, even though the 
rendering will never look good.

The tie-bar is a convenient diacritic for visually tokenizing input strings into chunks of 
phonetically salient groups and both the above and below tie-bar were introduced in electronic 
text to deal with difficult to render character sequences. The tie-bar is unnecessary for 
computational tokenization given the design of orthography profiles, described in 
Section \ref{usecase-tokenization}. However, to adhere to strict IPA encoding users 
may use either tie bar, or none whatsoever. 

\begin{comment} 
Finally, the tie bar to indicate affricates, doubly articulated consonants or
diphthongs can be placed either below or on top of the base characters, e.g.
<\charis{t͡s}> or <\charis{t͜s}>. For strict IPA encoding we propose to standardize on the tie bar
above the base characters, using Unicode \textsc{combining double inverted
breve} at \uni{0361}. Note that this Unicode character is placed between the two 
base characters to be combined. In principle, it is possible to combine more 
than two base characters by repeating the tie bar, like in <a͡o͡u>. If really
necessary, we consider this possible, even though the rendering will never look 
good.

For strict IPA encoding we propose to standardize on the tie bar
above the base characters, using Unicode \textsc{combining double inverted
breve} at \uni{0361}. Note that this Unicode character is placed between the two 
base characters to be combined. In principle, it is possible to combine more 
than two base characters by repeating the tie bar, like in <a͡o͡u>. If really
necessary, we consider this possible, even though the rendering will never look 
good.\end{comment}



% ==========================
\section{Pitfall: Ligatures and digraphs}
\label{pitfall-ligatures-digraphs}     
% ==========================   

One important distinction to acknowledge is the difference between multigraphs
and ligatures. Multigraphs are groups of characters (in the context of IPA e.g.
<tʃ> or <ou>) while ligatures are single characters (e.g. <ʧ> \textsc{latin
small letter tesh digraph} at \uni{02A7}). Ligatures arose in the context of
printing easier-to-read texts, and are included in the Unicode Standard for
reasons of legacy encoding. However, their usage is discouraged by the Unicode
core specification. Specifically related to IPA, various phonetic combinations
of characters (typically affricates) are available as single code-points in the
Unicode Standard, but are designated \textsc{digraphs}. Such glyphs might be used by
software to produce a pleasing display, but they should not be hard-coded into
the text itself. In the context of IPA, characters like the following ligatures
should thus \emph{not} be used. Instead a combination of two characters is
preferred:
      
\begin{itemize} 
	\item[] <ʣ> \textsc{latin small letter dz digraph} at \uni{02A3} 
	  (use <dz>) 
    \item[] <ʤ> \textsc{latin small letter dezh digraph} at \uni{02A4}
      (use <dʒ>)
    \item[] <ʥ> \textsc{latin small letter dz digraph with curl} at \uni{02A5}
      (use <dʑ>)
    \item[] <ʦ> \textsc{latin small letter ts digraph} at \uni{02A6} 
      (use <ts>)
	\item[] <ʧ> \textsc{latin small letter tesh digraph} at \uni{02A7} 
	  (use <tʃ>) 
    \item[] <ʨ> \textsc{latin small letter tc digraph with curl} at \uni{02A8}
      (use <tɕ>)
   	\item[] <ʩ> \textsc{latin small letter feng digraph} at \uni{02A9}
	  (use <fŋ>) 
\end{itemize}

However, there are a few Unicode characters that are historically ligatures, but
which are today considered as simple characters in the Unicode Standard and thus
should be used when writing IPA, namely:

\begin{itemize}
	\item[] <ɮ> \textsc{latin small letter lezh} at \uni{026E} 
	\item[] <œ> \textsc{latin small ligature oe} at \uni{0153} 
	\item[] <ɶ> \textsc{latin letter small capital oe} at \uni{0276} 
	\item[] <æ> \textsc{latin small letter ae} at \uni{00E6} 
\end{itemize}

% ==========================
\section{Pitfall: Missing decomposition}
\label{pitfall-missing-decomposition}
% ==========================

Although many combinations of base character with diacritic are treated as
canonical equivalent with precomposed characters, there are a few combinations
in IPA that allow for multiple, apparently identical, encodings that are not
canonical equivalent (see Section~\ref{pitfall-canonical-equivalence}),. The
following elements should not be treated as diacritics when encoding IPA in
Unicode: 
\begin{itemize}
  \item[] <\ {\large  ̡}\ > \textsc{combining palatalized hook below} at \uni{0321}
  \item[] <\ \ {\large  ̢}> \textsc{combining retroflex hook below} at \uni{0322}
  \item[] <\ \ {\large  ̵}> \textsc{combining short stroke overlay} at \uni{0335}
  \item[] <\ \ {\large  ̷}> \textsc{combining short solidus overlay} at \uni{0337}
  \item[] <\ \ {\large  ̴}> \textsc{combining tilde overlay} at \uni{0334}
\end{itemize} 

There turn out to be a lot of characters in the IPA that could be conceived as
using any of these elements, like <ɲ>, <ɳ>, <ɨ>, <ø> or <ɫ>. However, all such
characters exist as well as precomposed combination in Unicode, and these
precomposed characters should preferably be used.\footnote{The IPA does not
describe any character for a voiced retroflex implosive, which would
transparently be \charis{ᶑ}. We propose to add this character to the IPA.} When
combinations of a base character with diacritic are used, then these
combinations are not canonical equivalent to the precomposed combinations. This
means that any search will not find both at the same time.

A similar problem arises with the rhotic hook. There are two precomposed
characters in Unicode with a rhotic hook, which are not canonical equivalent 
with a combination of the vowel with a separately encoded hook:
\begin{itemize}
  \item[] <ɚ> \textsc{latin small letter schwa with hook} at \uni{025A}
  \item[] <ɝ> \textsc{latin small letter reversed open e with hook} at \uni{025D}
\end{itemize}
          
All other combinations of vowels with rhotic hooks will have to be made by using
<{\large ˞}> \textsc{modifier letter rhotic hook} at \uni{02DE}, because there
is not complete set of precomposed characters with rhotic hooks. For that reason
we propose to not use the two precomposed characters with hooks mentioned above,
but always use the separate rhotic hook at \uni{02DE} in IPA.\@

Reversely, note that the <ç> \textsc{latin small letter c with cedilla} at
\uni{00E7} will be decomposed into <c> with <\ \ {\large ̧}> \textsc{combining
cedilla} at \uni{0327} by Unicode canonical decomposition, also if such a
decomposition is not intended in the IPA.\@

% ==========================
\section{Pitfall: Different notions of diacritics}
\label{pitfall-different-notions-of-diacritics}
% ==========================

Another pitfall relates to the question of what are diacritics. The problem is that
the meaning of the term diacritics as used by the IPA is not the same as is used
in the Unicode Standard. Specifically, diacritics in the IPA-sense are either
so-called \textsc{combining diacritical marks} or \textsc{spacing modifier
letters} in the Unicode Standard. Crucially, Combining Diacritical Marks are by
definition combined with the character before them (to form so-called default
grapheme clusters, see Section~\ref{the-unicode-approach}). In contrast, Spacing
Modifier Letters are by definition \emph{not} combined into grapheme clusters
with the preceding character, but simply treated as separate letters. In the
context of the IPA, the following IPA-diacritics are actually Spacing Modifier
Letters in the Unicode Standard:

\begin{itemize}
  
	\item[] Length marks, namely: 
	\begin{itemize}
	  \item[] <ː> \textsc{modifier letter triangular colon} at \uni{02D0}
	  \item[] <ˑ> \textsc{modifier letter half triangular colon} at \uni{02D1}
	\end{itemize}
	 
	\item[] Tone letters, like: 
	\begin{itemize} 
	  \item[] <˥> \textsc{modifier letter extra-high tone bar} at \uni{02E5}
	  \item[] <˧> \textsc{modifier letter mid tone bar} at \uni{02E7}
	  \item[] and others like this
	\end{itemize}
	
	\item[] Superscript letters, like:
	\begin{itemize}
	  \item[] <ʰ> \textsc{modifier letter small h} at \uni{02B0}
	  \item[] <ˤ> \textsc{modifier letter small reversed glottal stop} at \uni{02E4}
	  \item[] <ⁿ> \textsc{superscript latin small letter n} at \uni{207F}
	  \item[] and many more like this
	\end{itemize}
	
	\item[] The rhotic hook:
	\begin{itemize}
	  \item[] <˞> \textsc{modifier letter rhotic hook} at \uni{02DE}
	\end{itemize}
	
\end{itemize}

Although linguists might expect these characters to belong together with the
character in front of them, at least for <ʰ> \textsc{modifier letter small h} at
\uni{02B0} the Unicode Consortium's decision to treat it as a separate character
is also linguistically correct, because according to the IPA it can be used both
for aspiration (more precisely post-aspiration following the base character) and
pre-aspiration (preceding the base character). Note that there exists a mechanism in
Unicode to force separate characters to be combined (namely by using the
\textsc{zero width joiner} at \uni{200D}), but this seems to be a rather
impractical, and probably not an enforceable solution to us.

% ==========================
\section{Pitfall: No unique diacritic ordering}
\label{pitfall-no-unique-diacritic-ordering}
% ==========================

Also related to diacritics is the question of ordering. To our knowledge, the
International Phonetic Association does not specify a specific ordering for
diacritics that combine with phonetic base symbols; this exercise is left to the
reasoning of the transcriber. However, such marks have to be explicitly ordered
if sequences of them are to be interoperable and compatible computationally. An example is a
labialized aspirated alveolar plosive: <tʷʰ>. There is nothing holding linguists
back from using <tʰʷ> instead (with exactly the same intended meaning). However,
from a technical standpoint, these two sequences are different, e.g.~if both
sequences are used in a document, searching for <tʷʰ> will not find any
instances of <tʰʷ>, and vice versa. Likewise, a creaky voiced syllabic dental
nasal can be encoded in various orders, e.g. <n̪̰̩>, <n̩̰̪> or <n̩̪̰>.

\subsubsection*{Canonical combining classes}

In accordance with the absence of any specification of ordering in the IPA, the
Unicode Standard likewise does not propose any standardized orders. Both leave it
to the user to be consistent; this approach naturally invites inconsistency across 
different authored resources.

There is one aspect of ordering for which
the Unicode Standard does present a canonical solution. However, it is uncontroversial
from a linguistic perspective. Diacritics in the Unicode Standard
(i.e.~Combining Diacritical Marks, see above) are classified in Canonical
Combining Classes. In practice, the diacritics are distinguished by their
position relative to the base character.\footnote{See
\url{http://unicode.org/reports/tr44/\#Canonical\_Combining\_Class\_Values} for a
detailed description.} When applying a Unicode normalization (NFC or NFD, see
Section~\ref{pitfall-canonical-equivalence}), the diacritics in different
positions are put in a specified order. This process therefore harmonizes the
difference between different encodings, e.g.\ of a midtone creaky voice
vowel <ḛ̄>. This grapheme cluster can be encoded either as <e> + <\ \ ̄> + <\ \
̰> or as <e> + <\ \ ̰> + <\ \ ̄>. To prevent this twofold encoding, the Unicode
Standard specifies the second ordering as canonical (in this case, diacritics
below are put before diacritics above).

When encoding a string according to the Unicode Standard, it is possible to do
this either using the NFC (composition) or NFD (decomposition) normalization.
Decomposition implies that precomposed characters (like <á> \textsc{latin small
letter a with acute} at \uni{00E1}) will be split into its parts. This might
sound preferable for a linguistic analysis, as the different diacritics are
separated from the base characters. However, note that most attached elements
like strokes (e.g.~in the <ɨ>), retroflex hooks (e.g.~in <ʐ>) or rhotic hooks
(e.g.~in <ɝ>) will not be decomposed, but strangely enough a cedilla (like in
<ç>) will be decomposed (see Section~\ref{pitfall-missing-decomposition}). In
general, Unicode decomposition does not behave like a feature decomposition as
expected from a linguistic perspective. It is thus important to consider Unicode
decomposition only as a technical procedure, and not assume that it is
linguistically sensible.

\subsubsection*{Diacritic ordering}

Facing the problem of specifying a consistent ordering of diacritics while
developing a large database of phonological inventories from the world's
languages, \citet[540]{Moran2012} defines a set of diacritic ordering
conventions. The conventions are influenced by the linguistic literature, though
some ad-hoc decisions had to be taken given the vast variability of phonological segments 
described by linguists. 

If a character sequence contains more than one diacritic below the base
character, then the place features are applied first (linguolabial, dental,
apical, laminal, advanced, retracted), followed by the manner features (raised,
lowered, advanced tongue root, retracted tongue root, frictionalized), then
secondary articulations (more round, less round, derhoticized), laryngeal
settings (creaky, breathy, voiced, devoiced (below), devoiced (above)), and
finally the syllabic or non-syllabic marker (for vowels, advanced and retracted
tongue root markers gets applied between the place and laryngeal setting). So
the order that is proposed is the following (where <\textbar{}> indicates
\textit{or} and <→> indicates \textit{precedes}):

\begin{itemize}
	\item[] \textsc{Combining Diacritical Marks (below) ordering:}
	\begin{itemize}	
	  \item[→] linguolabial <t̼> \textbar{} dental <t̪> \textbar{} apical <t̺> \textbar{} laminal <t̻>
	  \item[→] advanced <u̟> \textbar{} retracted <e̠> 
	  \item[→] raised <e̝> \textbar{} lowered <e̞>
	  \item[→] advanced tongue root <e̘> \textbar{} retracted tongue root <e̙>
	  \item[→] frictionalized <ɾ͓>
	  \item[→] more round <ɔ̹> \textbar{} less round <ɔ̜>
	  \item[→] creaky voice <b̰> \textbar{} breathy voice <b̤> \textbar{} voiced <s̬> \textbar{} devoiced <n̥>
	  \item[→] syllabic <n̩> \textbar{} non-syllabic <e̯>
	\end{itemize}
 \end{itemize}

Further, if a character sequence contains more than one diacritic above the base
character, we propose the following order. Note that this ordering is only
important when the extra short superscript is used. All other combinations will 
in practice never lead to problems.

\begin{itemize}
	\item[] \textsc{Combining Diacritical Marks (above) ordering:}
	\begin{itemize}
	  \item[→] nasalized <ẽ>
	  \item[→] centralized <ë> \textbar{} mid-centralized <e̽>
	  \item[→] extra short <ĕ>
%	  \item[→] (tone accents, e.g. <è>)
	  \item[→] unreleased <p̚>
%	  \item[→] Spacing Modifier Letters, see above	  
 \end{itemize} \end{itemize}

Finally, when a character sequence contains more than one character in Spacing
Modifier Letters, these will be placed after all combining diacritic marks in the
following order:

\begin{itemize}
	\item[] \textsc{Spacing Modifier Letters ordering:}
	\begin{itemize}
	  \item[→] rhotic hook <{\large \ ˞} >
	  \item[→] lateral release <ˡ> \textbar{} nasal release <ⁿ>
	  \item[→] labialized <ʷ>
	  \item[→] palatalized <ʲ>
	  \item[→] velarized <ˠ>
	  \item[→] pharyngealized <ˤ>
	  \item[→] glottalized <ˀ>
	  \item[→] aspirated <ʰ> \textbar{} ejective <ʼ>
	  \item[→] long <ː> \textbar{} half-long <ˑ>
	  \item[→] tone letters <˥ ˦ ˧ ˨ ˩>
	\end{itemize}
\end{itemize}

% ==========================
\section{Recommendations}
\label{ipa-recommendations}
% ==========================

Summarizing the pitfalls discussed in this chapter, we propose to define a 
``strict'' IPA encoding to be used when interoperability of phonetic resources 
is intended. For regular research papers, when only the visual outline is 
important, 

%\begin{table}[htdp]
  
 \topcaption{IPA letters with Unicode encodings}
 \label{tab:ipa_letters}
 \tablehead{\toprule & Code & Unicode name & IPA name \\ \midrule }
 \tabletail{\bottomrule}
  
\begin{center}
\begin{xtabular}{ l l L{4.5cm} L{4.5cm} }
a & \uni{0061} & \textsc{latin small letter a} & open front unrounded \\ 
æ & \uni{00E6} & \textsc{latin small letter ae} & raised open front unrounded \\ 
ɐ & \uni{0250} & \textsc{latin small letter turned a} & lowered schwa \\ 
ɑ & \uni{0251} & \textsc{latin small letter alpha} & open back unrounded \\ 
ɒ & \uni{0252} & \textsc{latin small letter turned alpha} & open back rounded \\ 
b & \uni{0062} & \textsc{latin small letter b} & voiced bilabial plosive \\ 
ʙ & \uni{0299} & \textsc{latin letter small capital b} & voiced bilabial trill \\ 
ɓ & \uni{0253} & \textsc{latin small letter b with hook} & voiced bilabial implosive \\ 
c & \uni{0063} & \textsc{latin small letter c} & voiceless palatal plosive \\ 
ç & \uni{00E7} & \textsc{latin small letter c with cedilla} & voiceless palatal fricative \\ 
ɕ & \uni{0255} & \textsc{latin small letter c with curl} & voiceless alveolo-palatal fricative \\ 
d & \uni{0064} & \textsc{latin small letter d} & voiced alveolar plosive \\ 
ð & \uni{00F0} & \textsc{latin small letter eth} & voiced dental fricative \\ 
ɖ & \uni{0256} & \textsc{latin small letter d with tail} & voiced retroflex plosive \\ 
ɗ & \uni{0257} & \textsc{latin small letter d with hook} & voiced dental/alveolar implosive \\ 
\charis{ᶑ} & \uni{1D91} & \textsc{latin small letter d with hook and tail} & voiced retroflex implosive \\ 
e & \uni{0065} & \textsc{latin small letter e} & close-mid front unrounded \\ 
ə & \uni{0259} & \textsc{latin small letter schwa} & mid-central schwa \\ 
ɛ & \uni{025B} & \textsc{latin small letter open e} & open-mid front unrounded \\ 
ɘ & \uni{0258} & \textsc{latin small letter reversed e} & close-mid central unrounded \\ 
ɜ & \uni{025C} & \textsc{latin small letter reversed open e} & open-mid central unrounded \\ 
ɞ & \uni{025E} & \textsc{latin small letter closed reversed open e} & open-mid central rounded \\ 
ɤ & \uni{0264} & \textsc{latin small letter rams horn} & close-mid back unrounded \\ 
f & \uni{0066} & \textsc{latin small letter f} & voiceless labiodental fricative \\ 
ɡ & \uni{0261} & \textsc{latin small letter script g} & voiced velar plosive \\ 
ɢ & \uni{0262} & \textsc{latin letter small capital g} & voiced uvular plosive \\ 
ɠ & \uni{0260} & \textsc{latin small letter g with hook} & voiced velar implosive \\ 
ʛ & \uni{029B} & \textsc{latin letter small capital g with hook} & voiced uvular implosive \\ 
ɣ & \uni{0263} & \textsc{latin small letter gamma} & voiced velar fricative \\ 
h & \uni{0068} & \textsc{latin small letter h} & voiceless glottal fricative \\ 
ħ & \uni{0127} & \textsc{latin small letter h with stroke} & voiceless pharyngeal fricative \\ 
ʜ & \uni{029C} & \textsc{latin letter small capital h} & voiceless epiglottal fricative \\ 
ɦ & \uni{0266} & \textsc{latin small letter h with hook} & voiced glottal fricative \\ 
ɧ & \uni{0267} & \textsc{latin small letter heng with hook} & voiceless postalveolar+velar fricative \\ 
i & \uni{0069} & \textsc{latin small letter i} & close front unrounded \\ 
ɪ & \uni{026A} & \textsc{latin letter small capital i} & lax close front unrounded \\ 
ɨ & \uni{0268} & \textsc{latin small letter i with stroke} & close central unrounded \\ 
j & \uni{006A} & \textsc{latin small letter j} & voiced palatal approximant \\ 
ʝ & \uni{029D} & \textsc{latin small letter j with crossed tail} & voiced palatal fricative \\ 
ɟ & \uni{025F} & \textsc{latin small letter dotless j with stroke} & voiced palatal plosive \\ 
ʄ & \uni{0284} & \textsc{latin small letter dotless j with stroke and hook} & voiced palatal implosive \\ 
k & \uni{006B} & \textsc{latin small letter k} & voiceless velar plosive \\ 
l & \uni{006C} & \textsc{latin small letter l} & voiced alveolar lateral approximant \\ 
ʟ & \uni{029F} & \textsc{latin letter small capital l} & voiced velar lateral approximant \\ 
ɫ & \uni{026B} & \textsc{latin small letter l with middle tilde} & voiced pharyngealized alveolar lateral approximant \\ 
ɬ & \uni{026C} & \textsc{latin small letter l with belt} & voiceless alveolar lateral fricative \\ 
ɭ & \uni{026D} & \textsc{latin small letter l with retroflex hook} & voiced retroflex lateral approximant \\ 
ɮ & \uni{026E} & \textsc{latin small letter lezh} & voiced alveolar lateral fricative \\ 
ʎ & \uni{028E} & \textsc{latin small letter turned y} & voiced palatal lateral approximant \\ 
m & \uni{006D} & \textsc{latin small letter m} & voiced bilabial nasal \\ 
ɱ & \uni{0271} & \textsc{latin small letter m with hook} & voiced labiodental nasal \\ 
n & \uni{006E} & \textsc{latin small letter n} & voiced alveolar nasal \\ 
ɴ & \uni{0274} & \textsc{latin letter small capital n} & voiced uvular nasal \\ 
ɲ & \uni{0272} & \textsc{latin small letter n with left hook} & voiced palatal nasal \\ 
ɳ & \uni{0273} & \textsc{latin small letter n with retroflex hook} & voiced retroflex nasal \\ 
ŋ & \uni{014B} & \textsc{latin small letter eng} & voiced velar nasal \\ 
o & \uni{006F} & \textsc{latin small letter o} & close-mid back rounded \\ 
ø & \uni{00F8} & \textsc{latin small letter o with stroke} & close-mid front rounded \\ 
œ & \uni{0153} & \textsc{latin small ligature oe} & open-mid front rounded \\ 
ɶ & \uni{0276} & \textsc{latin letter small capital oe} & open front rounded \\ 
ɔ & \uni{0254} & \textsc{latin small letter open o} & open-mid back rounded \\ 
ɵ & \uni{0275} & \textsc{latin small letter barred o} & close-mid central rounded \\ 
p & \uni{0070} & \textsc{latin small letter p} & voiceless bilabial plosive \\ 
ɸ & \uni{0278} & \textsc{latin small letter phi} & voiceless bilabial fricative \\ 
q & \uni{0071} & \textsc{latin small letter q} & voiceless uvular plosive \\ 
r & \uni{0072} & \textsc{latin small letter r} & voiced alveolar trill \\ 
ʀ & \uni{0280} & \textsc{latin letter small capital r} & voiced uvular trill \\ 
ɹ & \uni{0279} & \textsc{latin small letter turned r} & voiced alveolar approximant \\ 
ɺ & \uni{027A} & \textsc{latin small letter turned r with long leg} & voiced alveolar lateral flap \\ 
ɻ & \uni{027B} & \textsc{latin small letter turned r with hook} & voiced retroflex approximant \\ 
ɽ & \uni{027D} & \textsc{latin small letter r with tail} & voiced retroflex tap \\ 
ɾ & \uni{027E} & \textsc{latin small letter r with fishhook} & voiced alveolar tap \\ 
ʁ & \uni{0281} & \textsc{latin letter small capital inverted r} & voiced uvular fricative \\ 
s & \uni{0073} & \textsc{latin small letter s} & voiceless alveolar fricative \\ 
ʂ & \uni{0282} & \textsc{latin small letter s with hook} & voiceless retroflex fricative \\ 
ʃ & \uni{0283} & \textsc{latin small letter esh} & voiceless postalveolar fricative \\ 
t & \uni{0074} & \textsc{latin small letter t} & voiceless alveolar plosive \\ 
ʈ & \uni{0288} & \textsc{latin small letter t with retroflex hook} & voiceless retroflex plosive \\ 
u & \uni{0075} & \textsc{latin small letter u} & close back rounded \\ 
ʉ & \uni{0289} & \textsc{latin small letter u bar} & close central rounded \\ 
ɥ & \uni{0265} & \textsc{latin small letter turned h} & voiced labial-palatal approximant \\ 
ɯ & \uni{026F} & \textsc{latin small letter turned m} & close back unrounded \\ 
ɰ & \uni{0270} & \textsc{latin small letter turned m with long leg} & voiced velar approximant \\ 
ʊ & \uni{028A} & \textsc{latin small letter upsilon} & lax close back rounded \\ 
v & \uni{0076} & \textsc{latin small letter v} & voiced labiodental fricative \\ 
ʋ & \uni{028B} & \textsc{latin small letter v with hook} & voiced labiodental approximant \\ 
\charis{ⱱ} & \uni{2C71} & \textsc{latin small letter v with right hook} & voiced labiodental tap \\ 
ʌ & \uni{028C} & \textsc{latin small letter turned v} & open-mid back unrounded \\ 
w & \uni{0077} & \textsc{latin small letter w} & voiced labial-velar approximant \\ 
ʍ & \uni{028D} & \textsc{latin small letter turned w} & voiceless labial-velar fricative \\ 
x & \uni{0078} & \textsc{latin small letter x} & voiceless velar fricative \\ 
y & \uni{0079} & \textsc{latin small letter y} & close front rounded \\ 
ʏ & \uni{028F} & \textsc{latin letter small capital y} & lax close front rounded \\ 
z & \uni{007A} & \textsc{latin small letter z} & voiced alveolar fricative \\ 
ʐ & \uni{0290} & \textsc{latin small letter z with retroflex hook} & voiced retroflex fricative \\ 
ʑ & \uni{0291} & \textsc{latin small letter z with curl} & voiced alveolo-palatal fricative \\ 
ʒ & \uni{0292} & \textsc{latin small letter ezh} & voiced postalveolar fricative \\ 
ʔ & \uni{0294} & \textsc{latin letter glottal stop} & voiceless glottal plosive \\ 
ʕ & \uni{0295} & \textsc{latin letter pharyngeal voiced fricative} & voiced pharyngeal fricative \\ 
ʡ & \uni{02A1} & \textsc{latin letter glottal stop with stroke} & epiglottal plosive \\ 
ʢ & \uni{02A2} & \textsc{latin letter reversed glottal stop with stroke} & voiced epiglottal fricative \\ 
ǀ & \uni{01C0} & \textsc{latin letter dental click} & voiceless dental click \\ 
ǁ & \uni{01C1} & \textsc{latin letter lateral click} & voiceless alveolar lateral click \\ 
ǂ & \uni{01C2} & \textsc{latin letter alveolar click} & voiceless palatoalveolar click \\ 
ǃ & \uni{01C3} & \textsc{latin letter retroflex click} & voiceless (post)alveolar click \\ 
ʘ & \uni{0298} & \textsc{latin letter bilabial click} & voiceless bilabial click \\ 
β & \uni{03B2} & \textsc{greek small letter beta} & voiced bilabial fricative \\ 
θ & \uni{03B8} & \textsc{greek small letter theta} & voiceless dental fricative \\ 
χ & \uni{03C7} & \textsc{greek small letter chi} & voiceless uvular fricative \\
\end{xtabular}
\end{center}
%\end{table}