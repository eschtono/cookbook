\chapter{Writing Systems}
\label{writing_systems}

% \section{Introduction}
% \label{introduction}

Writing systems arise and develop in a complex mixture of cultural, technological and practical pressures. They tend to be highly conservative, in that people who have learned to read and write in a specific way (however impractical or tedious) are mostly unwilling to change their habits, e.g.~they tend to resist spelling reforms. In all literate societies there exists a strong socio-political mainstream that tries to force unification of writing (for example by strongly enforcing ``right'' from ``wrong'' writing in schools). However, there is also a large community of users who take as many liberties in their writing as they can get away with.

For example, the writing of tone diacritics in Yoruba is often proclaimed to be the right way to write, although many users of Yoruba writing seem to be perfectly fine with leaving them out. As pointed out by the proponents of the official rules, there are some homographs when leaving out the tone diacritics \citet[44]{Olumuyiw2013}. However, writing systems (and the languages they represent) are normally full of homophones, which is normally not a problem at all for speakers of the language. More importantly, writing is not just a purely functional tool, but just as importantly it is a mechanism to signal social affiliation. By showing that you \textit{know the rules} of expressing yourself in writing, others will more easily accept you as a worthy participant in their group. And that just as well holds for obeying to the official rules when writing a job application, as for obeying to the informal rules when writing an SMS to classmates in school. The case of Yoruba writing is an exemplary case, as even after more than a century of efforts to standardize the writing systems, there is still a wide range of variation in daily use \citet{Olumuyiw2013}.

The sometimes cumbersome and sometimes illogical structure, and the enormous variability of existing writing systems is a fact of life scholars have to accept and should try to adapt to as good as possible. Our plea here is a proposal for a formalization to do exactly that.

When considering the worldwide linguistic diversity, including all lesser-studied and endangered languages, there exist numerous different orthographies using symbols from the same scripts. For example, there are hundreds of orthographies using Latin-based alphabetic scripts. All of these orthographies use the same symbols, but these symbols differ in meaning and usage throughout the various orthographies. To be able to computationally use and compare different orthographies, we need a way to specify all orthographic idiosyncrasies in a computer-readable format (a process called `tailoring' in Unicode parlance). We call such specifications \textsc{orthography profiles}. Ideally, these specifications have to be integrated into so-called Unicode locale descriptions, though we will argue that in practice this is often not the most useful solution for the kind of problems arising in the daily practice of linguistics. Consequently, a central goal of this paper is to flesh out the linguistic challenges for locale descriptions, and work out suggestions to improve their structure for usage in a linguistic context. Conversely, we also aim to improve linguists' understanding and appreciation for the accomplishments of the Unicode Consortium in the development of the Unicode Standard.

The necessity to computationally use and compare different orthographies most forcefully arises in the context of language comparison. Concretely, in our current research our goal is to develop quantitative methods for language comparison and historical analysis in order to investigate worldwide linguistic variation and to model the historical and areal processes that underlie linguistic diversity, cf.~\citet{Steiner_etal2011,List2012,List2012a,ListMoran2013,MoranProkic2013}. In this work, it is crucial to be able to flexibly process across numerous resources with different orthographies. In many cases even different resources on the `same' language use different orthographic conventions. Another orthographic challenge that we encounter regularly in our linguistic practice is electronic resources on a particular language that claim to follow a specific orthographic convention (often a resource-specific convention), but on closer inspection such resources are almost always not consistently encoded. Thus a second goal of our orthography profiles is to allow for an easy specification of orthographic conventions, and use such profiles to check consistency and to report errors to be corrected.

A central step in our proposed solution to this problem is the tailored grapheme separation of strings of symbols, a process we call \textsc{grapheme tokenization}. Basically, given some strings of symbols (e.g.~morphemes, words, sentences) in a specific source, our first processing step is to specify how these strings have to be separated into graphemes, considering the specific orthographic conventions used in a particular source document. Our experience is that such a graphemic tokenization can be performed without extensive in-depth knowledge about the phonetic and phonological details of the language in question. For example, the specification that $<$ou$>$ is a grapheme of English is a much easier task than to specify what exactly the phonetic values of this grapheme are in any specific occurrence in English words. Grapheme separation is a task that can be performed relatively reliably and with limited availability of time and resources (compare, for example, the task of creating a complete phonetic or phonological normalization).

Although grapheme tokenization is only one part of the solution, it is an important and highly fruitful processing step. Given a grapheme tokenization, various subsequent tasks become easier, like (a) temporarily reducing the orthography in a processing pipeline, e.g.~only distinguishing high versus low vowels; (b) normalizing orthographies across sources (often including temporary reduction of oppositions), e.g.~specifying an (approximate) mapping to the International Phonetic Alphabet; (c) using co-occurrence statistics across different languages (or different sources in the same language) to estimate the probability of grapheme matches, e.g.~with the goal to find regular sound changes between related languages or transliterations between different sources in the same language.

Before we deal with these proposals, in the first part of this paper (Sections \ref{encoding} through \ref{ipa-meets-unicode}) we give an extended introduction to the notion of encoding (Section \ref{encoding}) and writing systems, both from a linguistic perspective and from the perspective of the Unicode Consortium (Section \ref{terminology}). We consider the Unicode Standard to be a breakthrough (and ongoing) development that fundamentally changed the way we look at writing systems, and we aim to provide here a slightly more in-depth survey of the many techniques that are available in the standard. A good appreciation for the solutions that the Unicode Standard also allows for a thorough understanding of possible pitfalls that one might encounter when using it (Section \ref{unicode-pitfalls}). As an example of the current state-of-the-art, we discuss the rather problematic marriage of the International Phonetic Alphabet (IPA) with the Unicode Standard (Section \ref{ipa-meets-unicode}).

The second part of the paper (Sections \ref{orthography-profiles} and \ref{use-cases}) describes our proposals for how to deal with the Unicode Standard in the daily practice of (comparative) linguists. First, we discuss the challenges of characterizing a writing system. To solve these problems, we propose the notions of orthography profiles, closely related to Unicode locale descriptions (Section \ref{orthography-profiles}). Finally, we discuss practical issues with actual examples (Section \ref{use-cases}). We provide reference implementation of our proposals in R and in Python, available as open-source libraries.

The following conventions are followed in this paper. All phonemic and phonetic representations are given in the International Phonetic Alphabet (IPA), unless noted otherwise \citep{IPA2005}. Standard conventions are used for distinguishing between graphemic < >, phonemic / / and phonetic [ ] representations. For character descriptions, we follow the notational conventions of the Unicode Standard \citep{Unicode2014}. Character names are represented in small capital letters (e.g.~\textsc{latin small letter schwa}) and code points are expressed as U\emph{+n} where \emph{n} is a four to six digit hexadecimal number, e.g.~U+0256, which can be rendered as the glyph <ə>.