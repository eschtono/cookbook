\chapter{Unicode pitfalls}
\label{unicode-pitfalls}

% ==========================
\section{Wrong it is not}
\label{wrong-it-is-not}
% ==========================

In this chapter we describe some of the most common pitfalls that we have
encountered when using the Unicode Standard in our own work, or in discussion
with other linguists. This section is not meant as a criticism of the decisions
made by the Unicode Consortium; on the contrary, we aim to highlight where the
technological aspects of the Unicode Standard diverge from many users'
intuitions. What have sometimes been referred to as problems or inconsistencies
in the Unicode Standard are mostly due to legacy compatibility issues, which can
lead to unexpected behavior by linguists using the standard. However, there are
also some cases in which the Unicode Standard has made decisions that
theoretically could have been taken differently, but for some reason or another
(mostly very good reasons) were accepted as they are now. We call behavior that
executes without error but does something different than the user
expected---often unknowingly---a \textsc{pitfall}.

In this context, it is important to realize that the Unicode Standard was not
developed to solve linguistic problems per se, but to offer a consistent
computational environment for written language. In those cases in which the
Unicode Standard behaves differently as expected, we think it is important not
to dismiss Unicode as ``wrong'' or ``deficient'', because our
experience is that in almost all cases the behavior of the Unicode Standard has
been particularly well thought through. The Unicode Consortium has a more
wide-ranging view of matters and often examines important practical use-cases
that from a linguistic point of view are normally not considered. Our general
guideline for dealing with the Unicode Standard is to accept it as it is, and
not to battle windmills. Alternatively, of course, it is possible to actively
engage in the development of the standard itself, an effort that is highly
appreciated by the Unicode Consortium.

% ==========================
\section{Pitfall: Characters are not glyphs}
\label{pitfall-characters-are-not-glyphs}
% ==========================

A central principle of Unicode is the distinction between character and glyph. A
character is the abstract notion of a symbol in a writing system, while a glyph
is the concrete drawing of such a symbol. In practice, there is a complex
interaction between characters and glyphs. A single Unicode character may of
course be rendered as a single glyph. However, a character may also be a piece
of a glyph, or vice-versa. Actually, all possible relations between glyphs and
characters are attested.

First, a single character may have different contextually determined glyphs. For
example, characters in writing systems like Hebrew and Arabic have different
glyphs depending on where they appear in a word. Some letters in Hebrew change
their form at the end of the word, and in Arabic, primary letters have four
contextually-sensitive variants (isolated, word initial, medial and final).
Second, a single character may be rendered as a sequence of multiple glyphs. For
example, in Tamil one Unicode character may result in a combination of a
consonant and vowel, which are rendered as two adjacent glyphs by fonts that
supports Tamil. Third, a single glyph may be a combination of multiple
characters. For example, the ligature <fi>, a single glyph, is the result of two
characters, <f> and <i>, that have undergone glyph substitution by font
rendering (see also Section~\ref{pitfall-faulty-rendering}). Like
contextually-determined glyphs, ligatures are (intended) artifacts of text
processing instructions. Finally, a single glyph may be a part of a
character, as exemplified by diacritics.

Further, the rendering of a glyph is dependent on the font being used. For
example, the Unicode character \textsc{latin small letter g} appears as <g> and
<{\fontspec{Courier}g}> in the Linux Libertine and Courier fonts, respectively,
because their typefaces are designed differently. Furthermore, font face may
change the visual appearance of a character, for example Times New Roman
two-story <{\fontspec{Times New Roman}a}> changes to a single-story glyph in italics
<\emph{\fontspec{Times New Roman}a}>. This becomes a real problem for some
phonetic typesetting (see Section~\ref{pitfall-ipa-homoglyphs}).

In sum, character-to-glyph mappings are complex technological issues that the
Unicode Consortium has had to address in the development of the Unicode
Standard, but for the lay user they can be utterly confusing because visual
rendering does not (necessarily) indicate logical encoding.

% ==========================
\section{Pitfall: Characters are not graphemes}
\label{pitfall-characters-are-not-graphemes}
% ==========================

The Unicode Standard is a character encoding system, and not a writing system
encoding system. This most forcefully becomes clear with the notion of grapheme.
From a linguistic point of view, graphemes are the basic building blocks of a
writing system (see Section~\ref{linguistic-terminology}). It is extremely
common, up to the point of being universally attested, that writing systems use
combinations of multiple symbols as a single grapheme, like <sch>, <th> or <ei>.
There is no possibility to encode such complex graphemes using the Unicode Standard.

The Unicode Standard deals with complex graphemes only inasmuch they consist of
base characters with diacritics (see
Section~\ref{pitfall-different-notions-of-diacritics} for a discussion of the
notion of diacritic). The Unicode Standard calls such combination ``grapheme
clusters''. Complex graphemes consisting of multiple base characters,
like <sch>, are called ``tailored grapheme clusters'' in Unicode parlance (see
Section~\ref{the-unicode-approach}).

Inspecting the Unicode Standard, there appear to be special Unicode characters
to ``glue'' together characters into larger tailored grapheme clusters,
specifically the \textsc{zero width joiner} at \uni{200D} and the
\textsc{combining grapheme joiner} at \uni{034F}. However, these characters are
confusingly named (cf.~Section~\ref{pitfall-names}). Both codepoints actually do
not join characters, but explicitly separate them. The zero-width joiner (ZWJ)
can be used to solve special problems related to ordering (called ``collation''
in Unicode parlance). The combining grapheme joiner (CGJ) can be used to
separate characters that are not supposed to form ligatures. 

To solve the issue of tailored grapheme clusters, Unicode offers some assistance
in the form of the Unicode Locale Descriptions. However, in the practice of
linguistic research, this is not a real solution. For that reason we propose to
use orthography profiles (see Chapter~\ref{orthography-profiles}). Basically,
both orthography profiles and locale descriptions offer a way to specify
tailored grapheme clusters. For example, for English one could specify that <sh>
is such a cluster. Consequently, this sequence of characters is then always
interpreted as a complex grapheme. For cases in which this is not the right
decision, like in the English word \textit{mishap}, the \textsc{zero width
joiner} at \uni{200D} has to be entered between <s> and <h>.

% ==========================
\section{Pitfall: Missing glyphs}
\label{pitfall-missing-glyphs}
% ==========================

The Unicode Standard is often praised (and deservedly so) for solving many of
the perennial problems with the interchange and display of the world's writing
systems. However, a common complaint from users is that, while the praise may be
true, they mostly just see some boxes on their screen instead of those promised
symbols. The problem of course is that users' computers do not have any glyphs
installed matching the Unicode code points in the file they are trying to
inspect. It is important to realize that internally in the computer everything
still works as expected: any handling of Unicode code points works independently
of how they are displayed on the screen. So, although a user might only see
boxes being displayed, this user should be assured that everything is still in
order.

The central problem behind the missing glyphs is that designing actual glyphs
includes a lot of different considerations and it is a time-consuming process.
Many traditional expectations of how specific characters should look like have
to be taken into account when designing glyphs. Those expectations are often not
well documented, and it is mostly up to the knowledge and experience of the font
designer to try and conform to them as good as possible. Therefore, most
designers produce fonts only including glyphs for certain parts of the Unicode
Standard, namely for those characters they feel comfortable with. At the same
time, the number of characters defined by the Unicode Standard is growing with
each new version, so it is neigh impossible for any designer to produce glyphs
for all characters. The result of this is that, almost necessarily, each font
only includes glyphs for a subset of the characters in the Unicode Standard.

The simple solution to missing glyphs is thus to install additional fonts
providing additional glyphs. For the more exotic characters there is often not
much choice. There are a few particularly large fonts that might be considered.
First, there is the \textsc{Everson Mono} font made by Michael Everson, which
currently includes 9,756 different glyphs (not including Chinese) updated up to
Unicode 7.0.\footnote{Everson Mono is available as shareware at
\url{http://www.evertype.com/emono/}.} Already a bit older is the \textsc{Titus
Cyberbit Basic} font made by Jost Gippert and Carl-Martin Bunz, which includes
10,044 different glyphs (not including Chinese), but not including newer
characters added after Unicode 4.0.\footnote{Titus Cyberbit Basic is available
at \url{http://titus.fkidg1.uni-frankfurt.de/unicode/tituut.asp}.}

Further, we suggest to always install at least one so-called \textsc{fall-back
font}, which provides glyphs that at least show the user some information about
the underlying encoded character. Apple Macintoshes have such a font (which is
invisible to the user), which is designed by Michael Everson and made available
for other systems through the Unicode Consortium.\footnote{The Apple/Everson
fallback font is available for non-Macintosh users at \newline
\url{http://www.unicode.org/policies/lastresortfont\_eula.html}.} Further, the
\textsc{GNU Unifont} is a clever way to produce bitmaps approximating the
intended glyph of each available character, updated to Unicode 7.0.\footnote{The
GNU Unifont is available at \url{http://unifoundry.com/unifont.html}.} Finally,
the Summer Institute of Linguistics provides a \textsc{SIL Unicode BMP Fallback
Font}, currently available up to Unicode version 6.1. This font does not even
attempt to show a real glyph, but only shows the hexadecimal code inside a box
for each character, so a user can at least see the Unicode codepoint of the
character to be displayed.\footnote{The SIL Unicode BMP Fallback Font is
available at \newline \url{http://scripts.sil.org/UnicodeBMPFallbackFont}.}

% ==========================
\section{Pitfall: Faulty rendering}
\label{pitfall-faulty-rendering}
% ==========================

A similar complaint to missing glyphs, discussed previously, is that while there
might be a glyph being displayed, it does not look right. There are two
reasons for unexpected visual display, namely automatic font substitution and
faulty rendering. Like missing glyphs, any such problems are independent from
the Unicode Standard. The Unicode Standard only includes very general
information about characters and leaves the specific visual display to others to
decide on. Any faulty display is thus not to be blamed on the Unicode
Consortium, but on a complex interplay of different mechanisms happening in a
computer to turn Unicode codepoints into visual symbols. We will only sketch a
few aspects of this complex interplay here.

Most modern software applications (like Microsoft Word) offer some approach to
\textsc{automatic font substitution}. This means that when a text is written in
a specific font (e.g.~Times New Roman) and an inserted Unicode character does not
have a glyph within this font, then the software application will automatically
search for another font to display the glyph. The result will be that this
specific glyph will look slightly different from the others. This mechanism
works differently depending on the software application, and mostly only limited
user influence is expected and little feedback is given, which might be rather
frustrating to font-aware users.\footnote{For example, Apple Pages does not give
any feedback that a font is being replaced, and the user does not seem to have
any influence on the choice of replacement (except by manually marking all
occurrences). In contrast, Microsoft Word does indicate the font replacement by
showing the name in the font menu of the font replacement. However, Word simply
changes the font completely, so any text written after the replacement is written in a
different font as before. Both behaviors leave much to be desired.}

The other problem with visual display is related to the so-called \textsc{font
rendering}. Font rendering refers to the process of the actual positioning of
Unicode characters on a page of written text. This positioning is actually a
highly complex problem, and many things can go wrong in the process. Well-known
rendering problems, like proportional glyph size or ligatures are reasonably
well understood. In contrast, the positioning of multiple diacritics relative to
a base character is still a widespread problem, even within the Latin script.
Especially when more than one diacritic is supposed to be placed above (or
below) each other, this often leads to unexpected effects in many modern
software applications. The problems arising in Arabic and in many southeast
Asian scripts (like Devanagari or Burmese) are even more complex. 

To understand where any problems arise it is important to realize that there are
basically three different approaches to font rendering. The most widespread is
Adobe's and Microsoft's \textsc{OpenType} system. This approach makes it
relatively easy for font developers, as the font itself does not include all
details about the precise placement of individual characters. For those details,
additional script-descriptions are necessary. All of those systems can lead to
unexpected behavior.\footnote{For more details about OpenType, see
\url{http://www.adobe.com/products/type/opentype.html} and
\url{http://www.microsoft.com/typography/otspec/}. Additional systems for
complex text layout are, among others, Microsoft's DirectWrite
\url{https://msdn.microsoft.com/library/dd368038.aspx} and the open-source
project HarfBuzz \url{http://www.freedesktop.org/wiki/Software/HarfBuzz/}.}
Alternative systems are \textsc{Apple Advanced Typography} (AAT) and the
open-source \textsc{Graphite} system from the Summer Institute of Linguistics
(SIL).\footnote{More information about AAT can be found at
\url{https://developer.apple.com/fonts/}. \newline SIL's Graphite is described
in detail at
\url{http://scripts.sil.org/cms/scripts/page.php?site\_id=projects\&item\_id=graphite\_home}.}
In both of these systems, a larger burden is placed on the description inside
the font.

There is mostly no real solution to problems arising from faulty font rendering.
Switching to another software application that offers better handling is the
only real alternative, but this is normally not an option for daily work. The 
experience with rendering on the side of the software industry is developing 
quickly, so we can expect the situation only to get better. In the meantime one 
can try to correct faulty layout by tweaking baseline and/or kerning (when such 
option are available).

% ==========================
\section{Pitfall: Blocks}
\label{pitfall-blocks}
% ==========================

The Unicode code space is subdivided into blocks of contiguous code points. For
example, the block called \textsc{Cyrillic} runs from \uni{0400} till
\uni{04FF}. These blocks arose as an attempt at ordering the enormous amount of
characters in Unicode, but the ideas of blocks very quickly ran into problems.
First, the size of a block is fixed, so when a block is full, a new block will
have to be instantiated somewhere further in the code space. For example, this
led to the blocks \textsc{Cyrillic Supplement}, \textsc{Cyrillic Extended-A}
(both of which are also already full) and \textsc{Cyrillic Extended-B}. Second,
when a specific character already exists, then it is not duplicated in another
block, although the name of the block might indicate that a specific symbol
should be available there. In general, names of blocks are just an approximate
indication of the kind of characters that will be in the block.

The problem with blocks arises because finding the right character among the
thousands of Unicode characters is not easy. Many software applications present
blocks as a primary search mechanism, because the block names suggest where to
look for a particular character. However, when a user searches for an IPA
character in the block \textsc{IPA Extensions}, then many IPA characters will not
be found there. For example, the velar nasal <ŋ> is not part of the block
\textsc{IPA Extensions} because it was already included as \textsc{latin small letter
eng} at \uni{014B} in the block \textsc{Latin Extensions-A}.

In general, finding a specific character in the Unicode Standard is often not
trivial. The names of the blocks can help, but they are not (and never were supposed
to be) a foolproof structure. It is not the goal nor aim of the Unicode
Consortium to provide a user interface to the Unicode Standard. If one often
encounters the problem of needing to find a suitable character, there are
various other useful services for end-users available.\footnote{The Unicode
website offers a basic interface to the code charts at
\url{http://www.unicode.org/charts/index.html}. As a more flexible interface, we
particularly like PopChar from Macility, available for both Macintosh and
Windows. There are also various free websites that offer search interfaces
to the Unicode code tables, like \url{http://unicode-search.net} or
\url{http://unicode-search.net}. A further useful approach for searching characters
using shape matching is \url{http://shapecatcher.com}.}

% ==========================
\section{Pitfall: Names}
\label{pitfall-names}
% ==========================

The names of characters in the Unicode Standard are sometimes misnomers and
should not be misinterpreted as definitions. For example, the \textsc{combining
grapheme joiner} at \uni{034F} does not join characters into larger graphemes
(see Section~\ref{pitfall-characters-are-not-graphemes}) and the \textsc{latin
letter retroflex click} \uni{01C3} is actually not the IPA symbol for a
retroflex click, but for an alveolar click (see
Section~\ref{pitfall-ipa-homoglyphs}). In a sense, these names can be seen as
``errors.'' However, it is probably better to realize that such names are just
convenience labels that are not going to be changed. Just like the block names
(Section~\ref{pitfall-blocks}), the character names are often helpful, but they
are not supposed to be definitions.

The actual intended ``meaning'' of a Unicode codepoint is a combination of the
name, the block and the character properties (see
Section~\ref{the-unicode-approach}). Further details about the underlying intentions 
with which a character should be used
are only accessible by perusing the actual decisions of the Unicode Consortium.
All proposals, discussions and decisions of the Unicode Consortium are publicly
available. Unfortunately there is not (yet) any way to easily find everything
that is ever proposed, discussed and decided in relation to a specific
codepoint of interest, so many of the details are often somewhat
hidden.\footnote{All proposals and other documents that are the basis of Unicode
decisions are avaialbe at \url{http://www.unicode.org/L2/all-docs.html}. The
actual decisions that make up the Unicode Standard are documented in the minutes
of the Unicode Technical Committee, available at
\url{http://www.unicode.org/consortium/utc-minutes.html}.}

% ==========================
\section{Pitfall: Homoglyphs}
\label{pitfall-homoglyphs}
% ==========================

Homoglyphs are visually indistinguishable glyphs (or highly similar glyphs) that
have different code points in the Unicode Standard and thus different character
semantics. As a principle, the Unicode Standard does not specify how a character
appears visually on the page or the screen. So in most cases, a different
appearance is caused by the specific design of a font, or by user-settings like
size or boldface. Taking an example already discussed in
Section~\ref{pitfall-homoglyphs}, the following symbols <g {\large \textit{g}}
\textbf{g} {\fontspec{ArialMT} {\small g} \textit{g} \textbf{g}}> are different
glyphs of the same character, i.e.~they may be rendered differently depending on
the typography being used, but they all share the same code point (viz.
\textsc{latin small letter g} at \uni{0067}). In contrast, the symbols
<{\fontspec{EversonMono}AАΑᎪᗅᴀꓮ𐊠𝖠𝙰}> are all different code points,
although they look highly similar---in some cases even sharing exactly the same
glyph in some fonts. All these different A-like characters include the following
code points in the Unicode Standard:

\begin{itemize}
	\item[] <{\fontspec{EversonMono}A}> \textsc{latin capital letter a}, at \uni{0041} 
	\item[] <{\fontspec{EversonMono}А}> \textsc{cyrillic capital letter a}, at \uni{0410} 
	\item[] <{\fontspec{EversonMono}Α}> \textsc{greek capital letter alpha}, at \uni{0391} 
	\item[] <{\fontspec{EversonMono}Ꭺ}> \textsc{cherokee letter go}, at \uni{13AA} 
	\item[] <{\fontspec{EversonMono}ᗅ}> \textsc{canadian syllabics carrier gho}, at \uni{15C5} 
	\item[] <{\fontspec{EversonMono}ᴀ}> \textsc{latin small letter capital a}, at \uni{1D00} 
	\item[] <{\fontspec{EversonMono}ꓮ}> \textsc{lisu letter a}, at \uni{A4EE} 
%	\item[] <{\fontspec{EversonMono}A}> \textsc{fullwidth latin capital letter a}, at \uni{FF21} 
	\item[] <{\fontspec{EversonMono}𐊠}> \textsc{carian letter a}, at \uni{102A0} 
%	\item[] <{\fontspec{EversonMono}𐌀}> \textsc{old italic letter a}, at \uni{10300} 
	\item[] <{\fontspec{EversonMono}𝖠}> \textsc{mathematical sans-serif capital a}, \uni{1D5A0} 
	\item[] <{\fontspec{EversonMono}𝙰}> \textsc{mathematical monospace capital a}, at \uni{1D670} 
\end{itemize}

The existence of such homoglyphs is partly due to legacy compatibility, but for
the most part these characters are simply different characters that happen to
look similar.\footnote{A particularly nice interface to look for homoglyphs is
\url{http://shapecatcher.com}, based on the principle of recognizing shapes
\citep{Belongie2002}.} Yet, they are supposed to behave different from the
perspective of a font designer. For example, when designing a Cyrillic font, the
<A> will have different aesthetics and different traditional expectation
compared to a Latin <A>.

Such homoglyphs are a widespread problem for consistent encoding. Although for
most users it looks like the words <voces> and <νοсеѕ> are almost identical, in
actual fact they do not even share a single code point.\footnote{The first words
consists completely of Latin characters, namely \unif{0076}, \unif{006F},
\unif{0063}, \unif{0065} and \unif{0073}, while the second is a mix of Cyrillic
and Greek characters, namely \unif{03BD}, \unif{03BF}, \unif{0041}, \unif{0435}
and \unif{0455}.} For computers these two words are completely different
entities. Commonly, when users with Cyrillic or Greek keyboards have to type
some Latin-based orthography, they mix similar looking Cyrillic or Greek
characters into their text, because those characters are so much easier to type.
Similarly, when users want to enter an unusual symbol, they normally search by
visual impression in their favorite software application, and just pick
something that looks reasonably alike to what they expect the glyph to look
like.

It is really easy to make errors at text entry and add characters that are 
not supposed to be included. Our proposals for orthography profiles (see
Chapter~\ref{orthography-profiles}) are a method for checking the consistency of 
any text. In situations in which interoperability is important, we consider it 
crucial to add such checks in any workflow.

% ==========================
\section{Pitfall: Canonical equivalence}
\label{pitfall-canonical-equivalence}
% ==========================

For some characters, there is more than one possible encoding in the Unicode
Standard. This is a possible pitfall, as this would mean that for the computer
there exist multiple different entities that for a user are the same. This
would, for example, lead to problems with searching, as the computer would
search for specific encodings, and not find all expected characters. As a
solution, the Unicode Standard includes a notion of \textsc{canonical
equivalence}. Different encodings are explicitly declared as equivalent in the
Unicode Standard code tables. Further, to harmonize all encodings in a specific
piece of text, the Unicode Standard proposes a mechanism of
\textsc{normalization}.

Consider for example the characters and following Unicode code points:
\begin{itemize}
	\def\labelenumi{\arabic{enumi}.} 
	\item <Å> \textsc{latin capital letter a with ring above} \uni{00C5} 
	\item <Å> \textsc{angstrom sign} \uni{212B}
	\item <Å> \textsc{latin capital letter a} \uni{0041}
	+ \textsc{combining ring above} \uni{030A}
\end{itemize}

The character, represented here by glyph <Å>, is encoded in the Unicode Standard
in the first two examples by a single-character sequence; each is assigned a
different code point. In the third example, the glyph is encoded in a
multiple-character sequence that is composed of two character code points. All
three sequences are \textsc{canonically equivalent}, i.e.~they are strings that
represent the same abstract character and because they are not distinguishable
by the user, the Unicode Standard requires them to be treated the same in
regards to their behavior and appearance. Nevertheless, they are encoded
differently. For example, if one were to search an electronic text (with
software that does not apply Unicode Standard normalization) for
\textsc{angstrom sign} (\uni{212B}), then the instances of \textsc{latin 
capital letter a with ring above} (\uni{00C5}) would not be found.

In other words, there are equivalent sequences of Unicode characters that should
be normalized, i.e.~transformed into a unique Unicode-sanctioned representation
of a character sequence called a \textsc{normalization form}. Unicode provides a
Unicode Normalization Algorithm, which essentially puts combining marks
into a specific logical order and it defines decomposition and composition
transformation rules to convert each string into one of four normalization
forms. We will discuss here the two most relevant normalization forms: NFC and
NFD.\@

The first of the three characters above is considered the \textsc{Normalization
Form C (NFC)}, where \textsc{C} stands for composition. When the process of NFC
normalization is applied to the character sequences in 2 and 3, both sequences
are normalized into the \textsc{pre-composed} character sequence in 1. Thus all
three canonical character sequences are standardized into one composition form
in NFC.\@The other central Unicode normalization form is the
\textsc{Normalization Form D (NFD)}, where \textsc{D} stands for decomposition.
When NFD is applied to the three examples above, all three, including
importantly the single-character sequences in 1 and 2, are normalized into the
\textsc{decomposed} multiple-sequence of characters in 3. Again, all three are
then logically equivalent and therefore comparable and syntactically
interoperable.

As illustrated, some characters in the Unicode Standard have alternative
representations (in fact, many do), but the Unicode Normalization Algorithm can
be used to transform certain sequences of characters into canonical
forms to test for equivalency. To determine equivalence, each
character in the Unicode Standard is associated with a combining class, which is
formally defined as a character property called \textsc{canonical combining
class} which is specified in the Unicode Character Database. The combining class
assigned to each code point is a numeric value between 0 and 254 and is used by
the Unicode Canonical Ordering Algorithm to determine which sequences of
characters are canonically equivalent. Normalization forms, as very briefly
described above, can be used to ensure character equivalence by ordering
character sequences so that they can be faithfully compared.

It is very important to note that any software applications that is Unicode
Standard compliant is free to change the character stream from one
representation to another. This means that a software application may compose,
decompose or reorder characters as its developers desire; as long as the
resultant strings are canonically equivalent to the original. This might lead to
unexpected behavior for users. Various players, like the Unicode Consortium, the
W{\large 3}C, or the TEI recommend NFC in most user-directed situations, and
some software applications that we tested indeed seem to automatically convert
strings into NFC.\footnote{See the summary of various recommendation here:
\url{http://www.win.tue.nl/~aeb/linux/uc/nfc_vs_nfd.html}.} This means in
practice that if a user, for example, enters <a> and <\dia{0300}>, i.e.~\textsc{latin
small letter a} at \uni{0061} and \textsc{combining grave accent} at \uni{0300},
this might be automatically converted into <à>, i.e.~\textsc{latin small letter
a with grave} at \uni{00E0}.\footnote{The behavior of software applications can
be quite erratic in this respect. For example, Apple's TextEdit does not do any
conversion on text entry. However, when you copy and paste some text inside the
same document in rich text mode (i.e.~RTF-format), it will be transformed into
NFC on paste. Saving a document does not do any conversion to the glyphs on
screen, but it will save the characters in NFC.}

% ==========================
\section{Pitfall: Absence of canonical equivalence}
\label{pitfall-absence-of-equivalence}
% ==========================

Although in most cases canonical equivalence will take care of alternative
encodings of the same character, there are a some cases in which the Unicode
Standard decided against equivalence. This leads to identical characters that
are not equivalent, like <ø> \textsc{latin small letter o with stroke} at
\uni{00F8} and <o̷> a combination of \textsc{latin small letter o} at \uni{006F}
with \textsc{combining short solidus overlay} at \uni{0037}.
The general rule followed is that extensions of Latin characters that are
connected to the base character are not separated as combining diacritics. For
example, characters like <ŋ ɲ ɳ> or <ɖ ɗ> are obviously derived from <n> and <d>
respectively, but they are treated like new separate characters in the Unicode
Standard. Likewise, characters like <ø> and <ƈ> are not separated into a base 
character <o> and <c> with an attached combining diacritic.

Interestingly, and somewhat illogically, there are three elements, which are
directly attached to their base characters, but which are still treated as
separable in the Unicode Standard. Such characters are decomposed (in NFD
normalization) in a base character with a combining diacritic. However, it is
these cases that should be considered the exceptions to the rule. These three 
elements are the following:

\begin{itemize}

  \item <\dia{0327}>: the \textsc{combining cedilla} at \uni{0327} \newline 
        This diacritic is
        for example attested in the precomposed character <ç> \textsc{latin
        small letter c with cedilla} at \uni{00E7}. This <ç> will thus be
        decomposed in NFC normalization.
  \item <\dia{0328}>: the \textsc{combining ogonek} at \uni{0328} \newline 
        This diacritic is
        for example attested in precomposed <ą> \textsc{latin small letter a
        with ogonek} at \uni{0105}. This <ą> will thus be decomposed in NFC
        normalization.
  \item <\dia{031B}>: the \textsc{combining horn} at \uni{031B} \newline 
        This diacritic is for
        example attested in precomposed <ơ> \textsc{latin small letter o with
        horn} at \uni{01A1}. This <ơ> will thus be decomposed in NFC
        normalization. 

\end{itemize}

There are further combinations that deserve specific care because it is actually
possible to produce identical characters in different ways without them being
canonically equivalent. In these situations, the general rule holds, namely that
characters with attached extras are not decomposed. However, in the following
cases the ``extras'' actually exist as combining diacritics, so there is also 
the possibility to construct a character by using a base character with those 
combining diacritics.

\begin{itemize}
  
  \item First, there are the combining characters designated as ``combining
        overlay'' in the Unicode Standard, like <\dia{0334}>
        \textsc{combining tilde overlay} at \uni{0334} or <\dia{0335}>
        \textsc{combining short stroke overlay} at \uni{0335}. There are many
        characters that look like they are precomposed with such an overlay,
        for example <\charis{ɫ~ᵬ~ᵭ~ᵱ}> or <\charis{ƚ~ɨ~ɉ~ɍ}>, or also the
        example of <ø> given at the start of this section. However, they are 
        not decomposed in NFD normalization.
  \item Second, the same situation also occurs with combining characters
        designated as ``combining hook'', like 
        <{\fontspec{CharisSIL}{\large ◌}}\symbol{"0321}> \textsc{combining
        palatalized hook below} at \uni{0321}. This element seems to occur in
        precomposed characters like <\charis{ᶀ~ᶁ~ᶂ~ᶄ}>. However, they are 
        not decomposed in NFD normalization.
        
\end{itemize}

To harmonize the encoding in these cases it is not sufficient to use Unicode 
normalization. Additional checks are necessary, for example by using orthography 
profiles (see Chapter~\ref{orthography-profiles}).

% ==========================
\section{Pitfall: File formats}
\label{pitfall-file-formats}
% ==========================

Unicode is a character encoding standard, but these characters of course
actually appear inside some kind of computer file. The most basic Unicode-based file
format is pure line-based text, i.e.~strings of Unicode-encoded characters
separated by line breaks (note that these line breaks are what for most people
intuitively corresponds to paragraph breaks). Unfortunately, even within this
apparently basic setting there exist a multitude of variants. In general, these
different possibilities are well-understood in the software industry, and
nowadays they normally do not lead to any problems for the end user. However,
there are some situations in which a user is suddenly confronted with cryptic
questions in the user interface involving abbreviations like LF, CR, BE, LE or
BOM.\@ Most prominently this occurs with exporting or importing data in several
software applications from Microsoft. Basically, there are two different issues
involved. First, the encoding of line breaks and, second, the encoding of the
Unicode characters into code units and the related issue of endianness.

\subsubsection*{Line breaks}

The issue with \textsc{line breaks} originated with the instructions necessary
to direct a printing head of a physical printer to a new line. This involves two
movements, known as \textsc{carriage return} (CR, returning the printing head to
the start of the line on the page) and \textsc{line feed} (LF, moving the
printing head to the next line on the page). Physically, these are two different
events, but conceptually together they form one action. In the history of
computing, various encodings of line breaks have been used (e.g.~CR+LF, LF+CR,
only LF, or only CR). Currently, all Unix and Unix-derived systems use only LF
as code for a line break, while software from Microsoft still uses a combination
of CR+LF.\@ Today, most software applications recognize both options, and
are able to deal with either encoding of line breaks (until rather recently this
was not the case, and using the wrong line breaks would lead to unexpected
errors). Our impression is that there is a strong tendency in software
development to standardize on the simpler ``only LF'' encoding for line
breaks, and we suggest that everybody use this encoding whenever possible.

\subsubsection*{Code units}

The issue with \textsc{code units} stems from the question how to separate a
stream of binary ones and zero, i.e.~bits, into chunks representing Unicode
characters. A code unit is the sequence of bits used to encode a single
character in an encoding. Depending on different use cases, the Unicode Standard
offers three different approaches, called UTF-32, UTF-16 and UTF-8.\footnote{The
letters UTF stand for \textsc{Unicode Transformation Format}, but the notion of
``transformation'' is a legacy notion that does not have meaning anymore.
Nevertheless, the designation UTF (in capitals) has become an official
standard designation, but should probably best be read as simply ``Unicode
Format.''} The details of this issue is extensively explained in section 2.5 of
the Unicode Core Specification~\citet{Unicode2014}. 

Basically, \textsc{UTF-32} encodes each character in 32 bits (32 \textit{bi}nary
uni\textit{ts}, i.e.~32 zeros or ones) and is the most disk-space-consuming
variant of the three. However, it is the most efficient encoding
processing-wise, because the computer simply has to separate each character
after 32 bits. 

In contrast, \textsc{UTF-16} uses only 16 bits per character, which is
sufficient for the large majority of Unicode characters, but not for all of
them. A special system of \textsc{surrogates} is defined within the Unicode Standard to
deal with these additional characters. The effect is a more disk-space efficient
encoding (approximately half the size), while adding a limited computational
overhead to manage the surrogates. 

Finally, \textsc{UTF-8} is a more complex system that dynamically encodes each
character with the minimally necessary number of bits, choosing either 8, 16 or
32 bits depending on the character. This represents again a strong reduction in
space (particularly due to the high frequency of data using erstwhile ASCII
characters, which need only 8 bits) at the expense of even more computation
necessary to process such strings. However, because of the ever growing
computational power of modern machines, the processing overhead is in most
practical situations a non-issue, while saving on space is still useful,
particularly for sending texts over the Internet. As an effect, UTF-8 has become
the dominant encoding on the World Wide Web. We suggest that everybody uses
UTF-8 as their default encoding.

A related problem is a general issue about how to store information in computer
memory, which is known as \textsc{endianness}. The details of this issue go
beyond the scope of this book. It suffices to realize that there is a
difference between \textsc{big-endian} (BE) storage and \textsc{little-endian}
(LE) storage. The Unicode Standard offers a possibility to explicitly indicate
what kind of storage is used by starting a file with a so-called \textsc{byte order
mark} (BOM). However, the Unicode Standard does not require the usage of BOM,
preferring other non-Unicode methods to signal to computers which kind of
endianness is used. This issue only arises with UTF-32 and UTF-16 encodings.
When using the preferred UTF-8, using a BOM is theoretically possible, but
strongly dispreferred according to the Unicode Standard. We suggest that
everyone tries to prevent the inclusion of BOM in your data.

% ==========================
\section{Recommendations}
\label{recommendations}
% ==========================

Summarizing the pitfalls discussed in this chapter, we propose the following 
recommendations:

\begin{itemize}
   \item To prevent strange boxes instead of nice glyphs, always install a few
         fonts with a large glyph collection and at least one fall-back font
         (see Section~\ref{pitfall-missing-glyphs}).
   \item Unexpected visual impressions of symbols does not necessarily mean that
         the actual encoding is wrong. It is mostly a problem of faulty
         rendering (see Section~\ref{pitfall-faulty-rendering}).
   \item Do not trust the names of codepoints as a definition of the character
         (see Section~\ref{pitfall-names}). Also do not trust Unicode blocks as
         a strategy to find specific characters (see
         Section~\ref{pitfall-blocks})
   \item To ensure consistent encoding of texts, apply Unicode normalization
         (NFC or NFD, see Section~\ref{pitfall-canonical-equivalence}).
   \item To prevent remaining inconsistencies after normalization, for example 
         stemming from homoglyphs (see Section~\ref{pitfall-homoglyphs}) 
         or from missing canonical equivalence in the Unicode Standard
         (see Section~\ref{pitfall-absence-of-equivalence})
         use orthography profiles (see Chapter~\ref{orthography-profiles}).
   \item To deal with ``tailored'' grapheme clusters
         (Section~\ref{pitfall-characters-are-not-graphemes}), use Unicode Locale 
         Descriptions, or orthography profiles 
         (see Chapter~\ref{orthography-profiles}).
   \item As a preferred file format, use Unicode Format UTF-8 in 
         Normalization Form Composition (NFC) with LF line endings, 
         but without byte order mark (BOM), whenever possible (see 
         Section~\ref{pitfall-file-formats}). This last nicely cryptic 
         recommendation has T-shirt potential:
  
\end{itemize}

\begin{center}
  I prefer it
  
  \textbf{UTF-8 NFC LF no BOM}
\end{center}


