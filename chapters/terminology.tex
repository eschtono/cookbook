\chapter{Terminology}
\label{terminology}

\section{The linguistic terminological tradition}
\label{the-linguistic-tradition}

In this section we provide a very brief overview of the linguistic terminology concerning writing systems before turning to the slightly different computational terminology in the next section on the Unicode Standard. 

Linguistically speaking, a \textsc{writing system} is a symbolic system that uses visible or tactile signs to represent language in a systematic way. The term \textsc{writing system} has two mutually exclusive meanings. First, it may refer to the way a particular language is written. In this sense the term refers to the writing system of a particular language, as, for example, in \emph{``the Serbian writing system uses two scripts: Latin and Cyrillic''}. Second, the term \textsc{writing system} may also refer to a type of symbolic system as used among the world's languages to represent the language, as, for example, in \emph{``alphabetic writing system''}. In this latter sense the term refers to how scripts have been classified according to the way that they encode language, as in, for example, \emph{``the Latin and Cyrillic scripts are both alphabetic writing systems''}. To avoid confusion, this second notion of \textsc{writing system} would more aptly have been called \textsc{script system}. 

The term \textsc{script} refers to a collection of distinct symbols as employed by one or more orthographies. For example, both Serbian and Russian are written with subsets of the Cyrillic script. A single language, like Serbian or Japanese, can also be written using orthographies based on different scripts. Over the years linguists have typologized script systems in a variety of ways, with the tripartite classification of logographic, syllabic, and alphabetic remaining the most popular, even though there are at least half a dozen different types of script systems that can be distinguished \citep{Daniels1990,Daniels1996}.

Only focussing here on the first sense of writing systems, we distinguish two different kinds of writing systems used to write a particular language, namely transcriptions and orthographies. First, \textsc{transcription} is a scientific procedure (and also the result of that procedure) for graphically representing the sounds of human speech at the phonetic level. It incorporates a set of unambiguous symbols to represent speech sounds, including conventions that specify how these symbols should be combined. A transcription system is a specific system of symbols and rules used for transcription of the sounds of a spoken language variety. In principle, a transcription system should be language-independent, in that it should be applicable to all spoken human languages. The International Phonetic Alphabet (IPA) is a commonly used transcription system that provides a medium for transcribing languages at the phonetic level. However, there is a long history of alternative kinds of transcription systems (see \citet{Kemp1994}) and today various alternatives are in widespread use (e.g.~X-SAMPA and Cyrillic-based phonetic transcription systems).\footnote{Most users of IPA do not follow the standard explicitly and many dialects based on it have emerged, e.g.~the Africanist and Americanist transcription systems.} Note that IPA symbols are also often used to represent language on a phonemic level. However, it is important to realize that in this usage the IPA symbols are not a transcription system, but rather an orthography (though with strong links to the pronunciation). Further, a transcription system does not need to be highly detailed, but it can also be a system of broad sound classes. Although such an approximative transcription is not normally used in linguistics, it is widespread in technological approaches (e.g.~\citet{SoundexBeiderMorse2010}\footnote{\url{http://stevemorse.org/phonetics/bmpm.htm}, \url{http://stevemorse.org/phonetics/bmpm2.htm}} and it is sometimes fruitfully used in automatic approaches to historical linguistics \citep{Dolgopolsky1986,ASJP16}.\footnote{\url{http://asjp.clld.org/}}

Second, an \textsc{orthography} specifies the symbols, punctuations, and the rules in which a specific language is written in a standardized way. Orthographies are often based on a phonemic analysis, but they almost always include idiosyncrasies because of historical developments (like sound changes or loans) and because of the widely-followed principle of lexical integrity (i.e.~the attempt to write the same lexical root in a consistent way, also when synchronic phonemic rules change the pronunciation, for example with final devoicing in many Germanic languages). All orthographies are language-specific (and often even resource-specific), although individual symbols or rules might be shared between languages. 

A \textsc{practical orthography} is a strongly phoneme-based writing system designed for practical use by speakers. The mapping relation between phonemes and graphemes in practical orthographies is purposely shallow, i.e.~there is mostly a systematic and faithful mapping from a phoneme to a grapheme. Practical orthographies are intended to jumpstart written materials development by correlating a writing system with the sound units of a language (cf.~\citet{MeinhofJones1928}). Symbols from the IPA are often used by linguists in the development of practical orthographies for languages without writing systems, though this usage of IPA symbols should not be confused with transcription (as defined above). 

Further, a \textsc{transliteration} is a mapping between two different orthographies. It is the process of ``recording the graphic symbols of one writing system in terms of the corresponding graphic symbols of a second writing system'' \citep[396]{Kemp2006}. In straightforward cases, such a transliteration is simply a matter of replacing one symbol with another. However, there are widespread complications, like one-to-many or many-to-many mappings, which are not always easy, or even possible, to solve without listing all cases individually (cf.~\citet[chp 2]{Moran2012}).

Breaking it down further, a script consists of \textsc{graphemes}, and graphemes consist of \textsc{characters}. In the linguistic terminology of writing systems, a \textsc{character} is a general term for any self-contained element in a writing system.\footnote{There is a second interpretation of the term \textsc{character}, i.e.~a conventional term for a unit in the Chinese writing system \citep{Daniels1996}. This interpretation will not be further explored in this paper.} Although in literate societies most people have a strong intuition about what the characters are in their particular orthography or orthographies, it turns out that the separation of an orthography into separate characters is far from trivial. The widespread intuitive notion of a character is strongly biased towards educational traditions, like the alphabet taught at schools, and technological possibilities, like the available type pieces in a printer's job case, the keys on a typewriter, or the symbols displayed in Microsoft Word's symbol browser. In practice, characters often consist of multiple building blocks, each of which could be considered a character in its own right. For example, although a Chinese character may be considered to be a single basic unanalyzable unit, at a more fine-grained level of analysis the internal structure of Chinese characters is often comprised of smaller semantic and phonetic units that should be considered characters \citep{Sproat2000}. In alphabetic scripts, this problem is most forcefully exemplified by diacritics. 

A \textsc{diacritic} is a mark, or series of marks, that may be above, below, or through other characters \citep{Gaultney2002}. Diacritics are sometimes used to distinguish homophonous words, but they are more often used to indicate a modified pronunciation \citep[xli]{DanielsBright1996}. The central question is whether, for example, <e>, <è>, <a> and <à> should be considered four characters, or different combinations of three characters. In general, multiple characters together can form another character, and it is not always possible to decide on principled grounds what should be the basic building blocks of an orthography.

For that reason, it is better to analyze an orthography as a collection of graphemes. A \textsc{grapheme} is the basic, minimally distinctive symbol of a particular writing system, alike to the phoneme is an abstract representation of a distinct sound in a specific language. The term \textsc{grapheme} was modeled after the term \textsc{phoneme} and represents a contrastive graphical unit in a writing system (see \citet{Kohrt1986} for a historical overview of the term grapheme). Most importantly, a single grapheme regularly consists of multiple characters, like <th>, <ou> and <gh> in English (note that each character in these graphemes is also a separate grapheme in English). Such complex graphemes are often used to represent single phonemes. So, a combination of characters is used to represent a single phoneme. Note that the opposite is also found in writing systems, in cases in which a single character represents a combination of two or more phonemes. For example, <x> in English orthography represents a combination of the phonemes /k/ and /s/. 

Further, conditioned or free variants of a grapheme are called \textsc{allographs}. For example, the distinctive forms of Greek sigma are conditioned, with <σ> being used word-internally and <ς> being used at the end of a word. In sum, there are many-to-many relationships between phonemes and graphemes as they are expressed in the myriad of language- and resource-specific orthographies.

This exposition of the linguistic terminology involved in describing writing systems has been purposely brief. We have highlighted some of the linguistic notions that are pertinent, yet sometimes confused with, the technological definitions developed for the computational processing of the world's writing systems, which we describe in the next section.

\section{The Unicode approach}
\label{the-unicode-approach}

The conceptualization and terminology of writing systems was rejuvenated through the development of the Unicode Standard, with major input from Mark Davis, co-founder and long-term president of the Unicode Consortium. For many years, \textit{exotic} writing systems and phonetic transcription systems on personal computers were constrained by the American Standard Code for Information Interchange (ASCII) character encoding scheme (based on the Latin script), which only allowed for a strongly limited number of different symbols to be encoded. This implied that users could either use and adopt the (extended) Latin alphabet or they could assign new symbols to the small number of code points in the ASCII encoding scheme to be rendered by a specifically designed font \citep{BirdSimons2003}. In this situation, it was necessary to specify the font together with each document to ensure the rightful display of its content. To alleviate this problem of assigning different symbols to the same code points, in the late 80's and early 90's the Unicode Consortium set itself the ambitious goal of developing a single universal character encoding to provide a unique number, a code point, for every character in the world's writing systems. Nowadays, the Unicode Standard is the default encoding of the technologies that support the World Wide Web and for all modern operating systems, software and programming languages.

The Unicode Standard represents a massive step forward because it aims to eradicate the distinction between \textsc{universal} (ASCII) versus \textsc{language/resource-particular} (Font) by adding as much as possible language-specific information into the universal standard. However, there are still language/resource-specific specifications necessary for the proper usage of Unicode, as will be discussed below. Within the Unicode structure many of these specifications can be captured by so-called \textsc{Locales} or \textsc{Locale Descriptions}, so we are moving to a new distinction of \textsc{universal} (Unicode Standard) versus \textsc{language-particular} (Locale Description). The major gain is a much larger compatibility on the universal level (because Unicode standardizes a much larger portion of writing system diversity), and much better possibilities for automatic processing on the language-particular level (because Locale Descriptions are computer readable specifications).

Each version of the Unicode Standard (as of writing at version 7) consists of a set of specifications and guidelines that include (i) a core specification, (ii) code charts, (iii) standard annexes and (iv) a character database.\footnote{All documents of the Unicode Standard are available at \url{http://www.unicode.org/versions/latest/}. For a quick survey of the use of terminology inside the Unicode Standard, their glossary is particularly useful, available at \url{http://www.unicode.org/glossary/}. For a general introduction to the principles of Unicode, Chapter 2 of the core specification, called \textsc{General Structure}, is particularly insightful. Different from many other documents of the Unicode Standard, this general introduction is relatively easy to read and illustrated with many interesting examples from various orthography traditions all over the world.} The \textsc{core specification} is a book directed toward human readers that describes the formal standard for encoding multilingual text. The \textsc{code charts} provide a humanly readable online reference to the character contents of the Unicode Standard in the form of PDF files. The \textsc{Unicode Standard Annexes (UAX)} are a set of technical standards that describe the implementation of the Unicode Standard for software development, Web standards, programming languages, etc. The \textsc{Unicode Character Database (UCD)} is a set of computer-readable text files that describe the character properties, including a set of rich character and writing system semantics, for each character in the Unicode Standard. In this section, we introduce the basic Unicode concepts, but we will leave out many details. Please consult the above mentioned full documentation for a more detailed discussion. Further note that the Unicode Standard is exactly that, namely a standard. It normatively describes notions and rules to be followed. In the actual practice of applying this standard in a computational setting, a specific implementation is necessary. The most widely used implementation of the Unicode Standard is the \textsc{International Components for Unicode (ICU)}, which offers C/C++ and Java libraries trying to implement to Unicode Standard.\footnote{\url{http://icu-project.org}}

The Unicode Standard is a \textsc{character encoding system} which goal it is to support the interchange and processing of written characters and text in a computational setting. Underlyingly, the character encoding is represented by a range of numerical values called a \textsc{code space}, which is used to encode a set of characters. A \textsc{code point} is a unique non-negative integer within a code space (i.e.~within a certain numerical range). In the Unicode Standard character encoding system, an \textsc{abstract character}, for example a \textsc{LATIN SMALL LETTER P}, is mapped to a particular code point, such as the decimal value 112, normally represented in hexadecimal and in Unicode parlance as (U+)0070.\footnote{Hexadecimal (base-16) 0070 is equivalent to decimal (base-10) 112, which can be calculated by considering that $(0\cdot16^3) + (0\cdot16^2) + (7\cdot16^1) + (0\cdot16^0) = 7\cdot16 = 112$. Underlyingly, computers will of course treat this code point binary (base-2) as 11100000, as can be seen by calculating that $(1\cdot2^7) + (1\cdot2^6) + (1\cdot2^5) + (0\cdot2^4) + (0\cdot2^3) + (0\cdot2^2) + (0\cdot2^1) + (0\cdot2^0) = 64 + 32 + 16 = 112$.} That encoded abstract character is rendered on a computer screen (or printed page) as a \textsc{glyph}, e.g. <p>, depending on the \textsc{font} and the context in which that character appears.

In Unicode Standard terminology, an (abstract) \textsc{character} is the basic encoding unit. The term \textsc{character} can be quite confusing due to its alternative definitions across different scientific disciplines and because in general the word \textsc{character} means many different things to different people. It is therefore often preferable to refer to Unicode characters simply as \textsc{code points}, because there is a one-to-one mapping between Unicode characters and their numeric representation. The \textit{Unicode2012} defines a character as ``the smallest component of written language that has semantic value. It refers to the abstract meaning and/or general shape, rather than a specific shape, though in code tables some form of visual representation is essential for the reader's understanding.'' Unicode defines characters as abstractions of orthographic symbols, and it does not define visualizations for these characters (although it does presents examples). In contrast, a \textsc{glyph} is a concrete graphical representation of a character as it appears when rendered (or rasterized) and displayed on an electronic device or on printed paper. For example, <reinsert Gs here> are different glyphs of the same character, i.e.~they may be rendered differently depending on the typography being used, but they all share the same code point. From the perspective of Unicode they are \textit{the same thing}. In this approach, a \textsc{font} is then simply a collection of glyphs linked to code points. Allography is not specified in Unicode (expect for a few exceptional cases, due to legacy encoding issues), but can be specified in a font as a \textsc{contextual variant} (aka presentation form).

Each code point in the Unicode Standard is associated with a set of \textsc{character properties} as defined by the Unicode character property model.\footnote{The character property model is described in \url{http://www.unicode.org/reports/tr23/}, but the actual properties are described in \url{http://www.unicode.org/reports/tr44/}. A simplified overview of the properties is available at \url{http://userguide.icu-project.org/strings/properties}. The actual code tables listing all properties for all Unicode code points are available at \url{http://www.unicode.org/Public/UCD/latest/ucd/}.} Basically, those properties are just a long list of values for each character. For example, code point U+0047 has the following properties (among many others):
\begin{itemize}
	\item Name: LATIN CAPITAL LETTER G 
	\item Alphabetic: YES 
	\item Uppercase: YES 
	\item Script: LATIN 
	\item Extender: NO 
	\item Simple\_Lowercase\_Mapping: 0067 
\end{itemize}

These properties contain the basic information of the Unicode Standard and they are necessary to define the correct behavior and conformance required for interoperability in and across different software implementations (as defined in the Unicode Standard Annexes). The character properties assigned to each code point is based on each character's behavior in the real-world writing traditions. For example, the corresponding lowercase character to U+0047 is U+0067 (though note that the relation between uppercase and lowercase is in many situations much more complex than this, and Unicode has further specifications for those cases). Another use of properties is to define the script of a character. The Unicode Standard defines the term \textsc{script} as, ``A collection of letters and other written signs used to represent textual information in one or more writing systems. For example, Russian is written with a subset of the Cyrillic script; Ukrainian is written with a different subset. The Japanese writing system uses several scripts.'' In practice, script is simply defined for each character as the explicit \textsc{Script} property in the Unicode Character Database.

One frequently references property is the \textsc{block} property, which is often used in software applications to impose some structure to the large number of Unicode characters. Each character in Unicode belongs to a specific block. These blocks are basically an organizational structure to alleviate the administrative burden of keeping Unicode up-to-date. Blocks consist of characters that in some way belong together, so that characters are easier to find. Some blocks are connected with a specific script, like the Hebrew block or the Gujarati block. However, blocks are predefined ranges of code points, and often there will come a point after which the range is completely filled. Any extra characters will have to be assigned somewhere else. There is, for example, a block \textsc{Arabic}, which contains most Arabic symbols. However, there is also a block \textsc{Arabic Supplement}, \textsc{Arabic Presentation Forms-A} and \textsc{Arabic Presentation Form B}. The situation with Latin symbols is even more extra. In general, the names for block should be taken as a definitional statement. For example, many IPA symbols are not located in the aptly-names block \textsc{IPA extensions}, but in other blocks (see Section \ref{ipa-meets-unicode}).

There are many cases in which a sequence of characters (i.e.~a sequence of more than one code point) represents what a user perceives as an individual unit in a particular orthographic writing system. For this reason the Unicode Standard differentiates between \textsc{abstract character} and \textsc{user-perceived character}. Sequences of multiple code points that correspond to a single user-perceived characters are called grapheme clusters in Unicode parlance. Grapheme clusters come in two flavors: (default) grapheme clusters and tailored grapheme clusters.

The (default) \textsc{grapheme clusters} are locale-independent graphemes, i.e.~they always apply when a particular combination of characters occurs independent of the writing system in which they are used. These character combinations are defined in the Unicode Standard as functioning as one \textsc{text element}.\footnote{The Unicode Standard defines text element as: ``A minimum unit of text in relation to a particular text process, in the context of a given writing system. In general, the mapping between text elements and code points is many-to-many.''} The simplest example of a grapheme cluster is a base character followed by a letter modifier character. For example, the sequence + <̃\textgreater{} (LATIN SMALL LETTER N at U+006E followed by COMBINING TILDE at U+0303) combines visually into , a user-perceived character in writing systems like that of Spanish. So, what the user perceives as a single character actually involves a multi-code-point sequence. Note that this specific sequence can also be represented with a single \textsc{precomposed} code point, LATIN SMALL LETTER N WITH TILDE at U+00F1, but this is not the case for all multi-code point character sequences. The problem that there multiple encodings possible for the same text element has been acknowledged early on in the Unicode Standard (e.g.~for , the sequence U+006E U+0303 should in all situations be treated identically to the precomposed U+00F1), and a system of \textsc{canonical equivalence} is available for such situations. Basically, the Unicode Standard offers different kind of normalizations to either decompose all precomposed characters (called \textsc{NFD}, \textsc{Normalization Form Canonical Decomposition}), or compose as much as possible combinations (called \textsc{NFC}, \textsc{Normalization Form Canonical Composition}). In current practice of software development, NFC seems to be preferred in most situations and is widely proposed as the preferred canonical form.

More difficult for text processing, because less standardized, is what the Unicode Standard terms \textsc{tailored grapheme clusters}. Tailored grapheme clusters are locale-dependent graphemes, i.e.~such combination of characters do not function as text elements in all situations. For example, the sequence + for the Slovak digraph or the sequence in the Sisaala practical orthography (pronounced as IPA /tʃ/; \cite{Moran2006}). These grapheme clusters are \textsc{tailored} in the sense that they must be specified on a language-by-language or a writing system-by-writing system basis. The Unicode Standard provides technological specifications for creating locale specific data in so-called \textsc{Unicode Locale Descriptions}, i.e.~a set of specification that defines a set of language-specific elements (e.g.~tailored grapheme clusters, collation order, capitalization-equivalence), as well as other special information, like how to format numbers, dates, or currencies. Locale descriptions are saved in the \textsc{Common Locale Data Repository (CLDR)},\footnote{\url{http://cldr.unicode.org/}} a repository of language-specific definitions of writing system properties, each of which describes specific usages of characters. Each locale can be encoded in a document using the \textsc{Locale Data Markup Language (LDML)}. LDML is an XML format and vocabulary for the exchange of structured locale data.

Unicode Locale Descriptions allow users to define language- or resource-specific writing systems or orthographies. The Unicode Consortium defines \textsc{writing system} only very loosely, as it is not a central concept in the Unicode Standard. A writing system is, ``A set of rules for using one or more scripts to write a particular language. Examples include the American English writing system, the British English writing system, the French writing system, and the Japanese writing system.''. However, there are various drawbacks of locale descriptions for the daily practice of linguistic work in a multilingual setting (see Section \ref{use-cases}).